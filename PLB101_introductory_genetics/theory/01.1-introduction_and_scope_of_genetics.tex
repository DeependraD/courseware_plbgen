\PassOptionsToPackage{unicode=true}{hyperref} % options for packages loaded elsewhere
\PassOptionsToPackage{hyphens}{url}
\documentclass[11pt,ignorenonframetext,aspectratio=169]{beamer}
\IfFileExists{pgfpages.sty}{\usepackage{pgfpages}}{}
\setbeamertemplate{caption}[numbered]
\setbeamertemplate{caption label separator}{: }
\setbeamercolor{caption name}{fg=normal text.fg}
\beamertemplatenavigationsymbolsempty
\usepackage{lmodern}
\usepackage{amssymb,amsmath}
\usepackage{ifxetex,ifluatex}
\usepackage{fixltx2e} % provides \textsubscript
\ifnum 0\ifxetex 1\fi\ifluatex 1\fi=0 % if pdftex
  \usepackage[T1]{fontenc}
  \usepackage[utf8]{inputenc}
\else % if luatex or xelatex
  \ifxetex
    \usepackage{mathspec}
  \else
    \usepackage{fontspec}
\fi
\defaultfontfeatures{Ligatures=TeX,Scale=MatchLowercase}







\fi







% use upquote if available, for straight quotes in verbatim environments
\IfFileExists{upquote.sty}{\usepackage{upquote}}{}
% use microtype if available
\IfFileExists{microtype.sty}{%
  \usepackage{microtype}
  \UseMicrotypeSet[protrusion]{basicmath} % disable protrusion for tt fonts
}{}


\newif\ifbibliography


\hypersetup{
      pdftitle={Introduction to genetics},
        pdfauthor={Deependra Dhakal},
          pdfborder={0 0 0},
    breaklinks=true}
%\urlstyle{same}  % Use monospace font for urls







% Prevent slide breaks in the middle of a paragraph:
\widowpenalties 1 10000
\raggedbottom

  \AtBeginPart{
    \let\insertpartnumber\relax
    \let\partname\relax
    \frame{\partpage}
  }
  \AtBeginSection{
    \ifbibliography
    \else
      \let\insertsectionnumber\relax
      \let\sectionname\relax
      \frame{\sectionpage}
    \fi
  }
  \AtBeginSubsection{
    \let\insertsubsectionnumber\relax
    \let\subsectionname\relax
    \frame{\subsectionpage}
  }



\setlength{\parindent}{0pt}
\setlength{\parskip}{6pt plus 2pt minus 1pt}
\setlength{\emergencystretch}{3em}  % prevent overfull lines
\providecommand{\tightlist}{%
  \setlength{\itemsep}{0pt}\setlength{\parskip}{0pt}}

  \setcounter{secnumdepth}{0}


  \usepackage{setspace}
  \usepackage{wasysym}
  % \usepackage{fontenc}
  \usepackage{booktabs,siunitx}
  \usepackage{longtable}
  \usepackage{array}
  \usepackage{multirow}
  \usepackage{wrapfig}
  \usepackage{float}
  \usepackage{colortbl}
  \usepackage{pdflscape}
  \usepackage{tabu}
  \usepackage{threeparttable}
  \usepackage{threeparttablex}
  \usepackage[normalem]{ulem}
  \usepackage{makecell}
  \usepackage{xcolor}
  \usepackage{tikz} % required for image opacity change
  \usepackage[absolute,overlay]{textpos} % for text formatting
  \usepackage[skip=0.333\baselineskip]{caption}
  % \usepackage{newtxtext,newtxmath}% better than txfonts   

  \sisetup{per-mode=symbol}

  % % Added by CII
  % \usepackage[format=hang,labelfont=bf,margin=0.5cm,justification=centering]{caption}
  % \captionsetup{font=small,width=0.9\linewidth,labelfont=small,textfont={small}}
  % % End of CII addition

  \usepackage{subcaption}
  % \newcommand{\subfloat}[2][need a sub-caption]{\subcaptionbox{#1}{#2}}

  \captionsetup[sub]{font=footnotesize,labelfont=footnotesize,textfont=footnotesize}
  % \captionsetup[subfigure]{font=small,labelfont=small,textfont=small}
  % \captionsetup[subfloat]{font=scriptsize,labelfont=scriptsize,textfont=scriptsize}

  % this font option is amenable for beamer, although these are global settings
  \setbeamerfont{caption}{size=\tiny}
  % \setbeamerfont{subcaption}{size=\tiny} % this does not chage subfloat fonts
  % \setbeamerfont{subfloat}{size=\tiny} % this does not change subfloat fonts
   
  % use single line spacing ?
  \singlespacing

  % use customize theme and colortheme
  % \usetheme[subsectionpage=progressbar,background=light]{metropolis}
  % \usetheme[]{Warsaw} % other options are: rose, Berlin
  % \usetheme[subsectionpage=progressbar,background=light]{metropolis} % it should be integrated to template itself by replacing preset theme
  \setbeamercolor{title}{fg=blue!85!black,bg=red!20!white}
  \setbeamertemplate{section in toc shaded}[default][50]
  % \usecolortheme{crane}
  \usecolortheme{seahorse}
  % beamer progress bar customization
  \usetikzlibrary{calc}

  \definecolor{pbblue}{HTML}{0A75A8}% filling color for the progress bar
  \definecolor{pbgray}{HTML}{575757}% background color for the progress bar

  \makeatletter
  \def\progressbar@progressbar{} % the progress bar
  \newcount\progressbar@tmpcounta% auxiliary counter
  \newcount\progressbar@tmpcountb% auxiliary counter
  \newdimen\progressbar@pbht %progressbar height
  \newdimen\progressbar@pbwd %progressbar width
  \newdimen\progressbar@tmpdim % auxiliary dimension

  \progressbar@pbwd=\linewidth
  \progressbar@pbht=1.0ex

  % the progress bar
  \def\progressbar@progressbar{%

      \progressbar@tmpcounta=\insertframenumber
      \progressbar@tmpcountb=\inserttotalframenumber
      \progressbar@tmpdim=\progressbar@pbwd
      \multiply\progressbar@tmpdim by \progressbar@tmpcounta
      \divide\progressbar@tmpdim by \progressbar@tmpcountb

    \begin{tikzpicture}[rounded corners=2pt,very thin]

      \shade[top color=pbgray!20,bottom color=pbgray!20,middle color=pbgray!50]
        (0pt, 0pt) rectangle ++ (\progressbar@pbwd, \progressbar@pbht);

        \shade[draw=pbblue,top color=pbblue!50,bottom color=pbblue!50,middle color=pbblue] %
          (0pt, 0pt) rectangle ++ (\progressbar@tmpdim, \progressbar@pbht);

      \draw[color=normal text.fg!50]  
        (0pt, 0pt) rectangle (\progressbar@pbwd, \progressbar@pbht) 
          node[pos=0.5,color=normal text.fg] {\textnormal{%
               \pgfmathparse{\insertframenumber*100/\inserttotalframenumber}%
               \pgfmathprintnumber[fixed,precision=2]{\pgfmathresult}\,\%%
          }%
      };
    \end{tikzpicture}%
  }

  \addtobeamertemplate{footline}{} % modified from 'headline' by dd_rookie
  {%
    \begin{beamercolorbox}[wd=\paperwidth,ht=4ex,center,dp=1ex]{white}%
      \progressbar@progressbar%
    \end{beamercolorbox}%
  }
  \makeatother

  \title[]{Introduction to genetics}

  \subtitle{An overview}

  \author[
        Deependra Dhakal
    ]{Deependra Dhakal}

  \institute[
    ]{
    College of Natural Resource Management, Tikapur, Kailali\\
Agriculture and Forestry University\\
\textit{ddhakal.rookie@gmail.com}\\
\url{https://rookie.rbind.io}
    }

\date[
      Academic year 2022-2023
  ]{
      Academic year 2022-2023
        }

\begin{document}

% Hide progress bar and footline on titlepage
  \begin{frame}[plain]
  \titlepage
  \end{frame}


  \begin{frame}
  \frametitle{Outline} % modified by dd_rookie
  \tableofcontents[hideallsubsections,pausesections] % modified by dd_rookie
  \end{frame}

\hypertarget{introduction}{%
\section{Introduction}\label{introduction}}

\begin{frame}{Background}
\protect\hypertarget{background}{}
\begin{itemize}[<+->]
\tightlist
\item
  A scientific discipline
\item
  The major drive of the green revolution
\item
  Issues of government regimes -- politics of masses during Hitler's
  Germany.
\item
  Can traditional agriculture satisfy human needs in distant future ?

  \begin{itemize}[<+->]
  \tightlist
  \item
    Modern variety
  \item
    GM crops and their safety
  \end{itemize}
\item
  Medical drugs, genetic engineering
\item
  Appreciation
\end{itemize}
\end{frame}

\begin{frame}{}
\protect\hypertarget{section}{}
\begin{columns}[T,onlytextwidth]
  
  \column{0.5\textwidth}

\begin{figure}
\includegraphics[width=0.7\linewidth]{../images/wild_flowers_abroad} \end{figure}
\begin{figure}
\includegraphics[width=0.7\linewidth]{../images/jacaranda_kathmandu} \end{figure}

  \column{0.5\textwidth}

\begin{figure}
\includegraphics[width=0.96\linewidth]{../images/wheat_trial_plots} \caption{Crop improvement}\label{fig:genetics-applications-2}
\end{figure}

\end{columns}
\end{frame}

\begin{frame}{}
\protect\hypertarget{section-1}{}
\begin{columns}[T,onlytextwidth]
  
  \column{0.5\textwidth}

\begin{figure}
\includegraphics[width=0.38\linewidth]{../images/green_revolution_borlaug} \caption{Norman Borlaug, a leader in the development of new strains of wheat that led to the Green Revolution. Borlaug was awarded the Nobel Peace Prize in 1970.}\label{fig:green-revolution-borlaug}
\end{figure}

  \column{0.5\textwidth}

\begin{figure}
\includegraphics[width=0.8\linewidth]{../images/green_revolution_crop} \caption{Modern, high-yielding rice plant (left) and traditional rice plant (right)}\label{fig:green-revolution-crop}
\end{figure}

\end{columns}
\end{frame}

\begin{frame}{Definition}
\protect\hypertarget{definition}{}
\begin{block}{Study of heredity}
\protect\hypertarget{study-of-heredity}{}
\begin{itemize}[<+->]
\tightlist
\item
  Why cats always have kittens and humans always have babies ?
\item
  Why do children resemble their parents ?
\item
  Why two people are never same ?
\end{itemize}
\end{block}

\begin{block}{Etymology}
\protect\hypertarget{etymology}{}
\begin{itemize}[<+->]
\tightlist
\item
  ``gene''
\item
  At any level of study genes are central
\item
  Genes have unique nature to perform their biological roles --
  replication, information bearing/generation of form and mutation
\end{itemize}
\end{block}
\end{frame}

\begin{frame}{Terminologies}
\protect\hypertarget{terminologies}{}
\begin{columns}[T,onlytextwidth]
  
  \column{0.3\textwidth}

\begin{itemize}
\item Selection
\item DNA
\item RNA
\item PCR
\item Viruses (Coronaviruses, H1N1)
\item BLAST
\item TATA
\item Hox1, Pht1
\item ELISA
\item FISH
\item CRISPR
\end{itemize}

  \column{0.7\textwidth}

\begin{figure}
\includegraphics[width=0.8\linewidth]{../images/chicken_selection} \caption{Terminologies and implications. Selection}\label{fig:terminologies-1}
\end{figure}

\end{columns}
\end{frame}

\begin{frame}{}
\protect\hypertarget{section-2}{}
\begin{center}\includegraphics[width=0.3\linewidth]{../images/viruses} \end{center}
\end{frame}

\begin{frame}{Branches}
\protect\hypertarget{branches}{}
\begin{figure}
\includegraphics[width=0.45\linewidth]{../images/genetics_subdivision} \caption{Subdivision of genetics into three interrelated fields}\label{fig:genetics-subdivision}
\end{figure}
\end{frame}

\hypertarget{past-and-present}{%
\section{Past and present}\label{past-and-present}}

\begin{frame}{Gregor Mendel}
\protect\hypertarget{gregor-mendel}{}
\begin{itemize}
\tightlist
\item
  Blending theory of inheritance
\item
  Mendelian theory of inheritance
\end{itemize}
\end{frame}

\begin{frame}{Mendel's Experiment}
\protect\hypertarget{mendels-experiment}{}
\begin{columns}[T,onlytextwidth]
  
  \column{0.5\textwidth}

\begin{figure}
\includegraphics[width=0.35\linewidth]{../images/mendels_experiment} \caption{The mating scheme for Mendel’s experiment involving the crossing of purple- and white- flowered varieties of pea plants. The purple and white circles signify the gene variants for purple vs. white flower color. Gametes carry one gene copy; the plants each carry two gene copies. The $  imes$ signifies a cross-pollination between the purple -- and white-flowered plants.}\label{fig:mendels-experiment}
\end{figure}
  
  \column{0.5\textwidth}

\begin{figure}
\includegraphics[width=0.5\linewidth]{../images/mendels_pea} \caption{\textbf{The seven phenotypic pairs studied by Mendel; For each character Mendel studied two contrasting phenotypes} \newline }\label{fig:mendels-pea}
\end{figure}

\end{columns}
\end{frame}

\begin{frame}{The crossing}
\protect\hypertarget{the-crossing}{}
\begin{figure}
\includegraphics[width=0.8\linewidth]{../images/crossing_selfing} \caption{Cross-pollination and selfing are two types of crosses}\label{fig:crossing-pea}
\end{figure}
\end{frame}

\begin{frame}{Model organisms}
\protect\hypertarget{model-organisms}{}
\begin{columns}[T,onlytextwidth]
  
  \column{0.33\textwidth}
  
\begin{figure}
\includegraphics[width=0.8\linewidth]{../images/model_organisms_drosophila} \caption{\textit{Drosophila melanogaster}}\label{fig:model-organisms-drosophila}
\end{figure}

  \column{0.33\textwidth}

\begin{figure}
\includegraphics[width=0.8\linewidth]{../images/model_organisms_coli} \caption{\textit{Escherichia coli}}\label{fig:model-organisms-ecoli}
\end{figure}

  \column{0.33\textwidth}

\begin{figure}
\includegraphics[width=0.8\linewidth]{../images/model_organisms_elegans} \caption{\textit{Caenorhabditis elegans}}\label{fig:model-organisms-caenorhabditis}
\end{figure}

\end{columns}
\end{frame}

\begin{frame}{}
\protect\hypertarget{section-3}{}
\begin{columns}[T,onlytextwidth]
  
  \column{0.33\textwidth}

\begin{figure}
\includegraphics[width=0.8\linewidth]{../images/model_organisms_arabidopsis} \caption{\textit{Arabidopsis thaliana}}\label{fig:model-organisms-arabidopsis}
\end{figure}

  \column{0.33\textwidth}

\begin{figure}
\includegraphics[width=0.8\linewidth]{../images/model_organisms_cerevisiae} \caption{\textit{Saccharomyces cerevisiae}}\label{fig:model-organisms-cerevisiae}
\end{figure}

  \column{0.33\textwidth}

\begin{figure}
\includegraphics[width=0.8\linewidth]{../images/model_organisms_mus} \caption{\textit{Mus musculus}}\label{fig:model-organisms-mus}
\end{figure}

\end{columns}
\end{frame}

\begin{frame}{Fundamental concepts of genetics}
\protect\hypertarget{fundamental-concepts-of-genetics}{}
\begin{itemize}[<+->]
\tightlist
\item
  Cells are of two basic types: \alert{Eukaryotic} and
  \alert{prokaryotic}
\item
  The gene is the fundamental unit of heredity
\item
  Genes come in multiple forms called alleles
\item
  Genes confer phenotypes
\item
  Genetic information is carried in DNA and RNA
\item
  Genes are located on chromosomes
\item
  Chromosomes separate through the process of mitosis and meiosis
\item
  Genetic information is transferred from DNA to RNA to protein
\item
  Mutations are permanent, heritable changes in genetic information
\item
  Some traits are affected by multiple factors
\item
  Evolution is genetic change
\end{itemize}
\end{frame}

\hypertarget{bibliography}{%
\section{Bibliography}\label{bibliography}}




\end{document}
