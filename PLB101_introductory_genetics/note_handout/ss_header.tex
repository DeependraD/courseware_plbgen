% \usepackage{exam}

\newcommand{\bquestions}{\begin{questions}}
\newcommand{\equestions}{\end{questions}}
\newcommand{\bsolution}{\begin{solution}}
\newcommand{\esolution}{\end{solution}}
\newcommand{\bparts}{\begin{parts}}
\newcommand{\eparts}{\end{parts}}
\newcommand{\stpart}{\part} % this is absolutely necessary to make new part command
\newcommand{\bsubparts}{\begin{subparts}}
\newcommand{\esubparts}{\end{subparts}}
\newcommand{\stsubpart}{\subpart}

% solution environment is a minipage and cannot support float so, always use "HOLD_position" 
\usepackage{float} % this package is essential for solution with code
% eqnarray is better avoided, most suggestion lead to \align provided in amsmath
% also split is used when there are very long lines of equation
% for different expression of the same equation use line separator \\ and use \notag or \nonumber
\usepackage{eqnarray, amsmath}

% \usepackage{background}
% \usepackage{eso-pic}
% \usepackage{contour}
% 
% \backgroundsetup{
%   angle=90,
%   opacity=0.8,
%   scale=1.2,
%   color=red,
%   nodeanchor=south west,
%   position={current page.south east},
%   contents={}{{\footnotesize Generated by }{\Large Deependra Dhakal }{(\footnotesize to be shared and distributed freely!)}},
%   hshift=40pt,% to move the text vertically
%   vshift=+10pt% to move the text horizontally
% }
% 
% \pagestyle{empty}

\newcommand{\under}[1]{%
  \vphantom{#1}%
  \underaccent{\bar}{\smash[b]{#1}}%
  }

\newcommand{\probP}{\text{I\kern-0.12em P}}
