% Options for packages loaded elsewhere
\PassOptionsToPackage{unicode}{hyperref}
\PassOptionsToPackage{hyphens}{url}
%
\documentclass[
  answers,addpoints,12pt]{exam}
\usepackage{amsmath,amssymb}
\usepackage{lmodern}
\usepackage{iftex}
\ifPDFTeX
  \usepackage[T1]{fontenc}
  \usepackage[utf8]{inputenc}
  \usepackage{textcomp} % provide euro and other symbols
\else % if luatex or xetex
  \usepackage{unicode-math}
  \defaultfontfeatures{Scale=MatchLowercase}
  \defaultfontfeatures[\rmfamily]{Ligatures=TeX,Scale=1}
\fi
% Use upquote if available, for straight quotes in verbatim environments
\IfFileExists{upquote.sty}{\usepackage{upquote}}{}
\IfFileExists{microtype.sty}{% use microtype if available
  \usepackage[]{microtype}
  \UseMicrotypeSet[protrusion]{basicmath} % disable protrusion for tt fonts
}{}
\makeatletter
\@ifundefined{KOMAClassName}{% if non-KOMA class
  \IfFileExists{parskip.sty}{%
    \usepackage{parskip}
  }{% else
    \setlength{\parindent}{0pt}
    \setlength{\parskip}{6pt plus 2pt minus 1pt}}
}{% if KOMA class
  \KOMAoptions{parskip=half}}
\makeatother
\usepackage{xcolor}
\usepackage[top=1.5cm,bottom=1.5cm,left=1.5cm,right=1.5cm]{geometry}
\usepackage{longtable,booktabs,array}
\usepackage{calc} % for calculating minipage widths
% Correct order of tables after \paragraph or \subparagraph
\usepackage{etoolbox}
\makeatletter
\patchcmd\longtable{\par}{\if@noskipsec\mbox{}\fi\par}{}{}
\makeatother
% Allow footnotes in longtable head/foot
\IfFileExists{footnotehyper.sty}{\usepackage{footnotehyper}}{\usepackage{footnote}}
\makesavenoteenv{longtable}
\usepackage{graphicx}
\makeatletter
\def\maxwidth{\ifdim\Gin@nat@width>\linewidth\linewidth\else\Gin@nat@width\fi}
\def\maxheight{\ifdim\Gin@nat@height>\textheight\textheight\else\Gin@nat@height\fi}
\makeatother
% Scale images if necessary, so that they will not overflow the page
% margins by default, and it is still possible to overwrite the defaults
% using explicit options in \includegraphics[width, height, ...]{}
\setkeys{Gin}{width=\maxwidth,height=\maxheight,keepaspectratio}
% Set default figure placement to htbp
\makeatletter
\def\fps@figure{htbp}
\makeatother
\setlength{\emergencystretch}{3em} % prevent overfull lines
\providecommand{\tightlist}{%
  \setlength{\itemsep}{0pt}\setlength{\parskip}{0pt}}
\setcounter{secnumdepth}{-\maxdimen} % remove section numbering
% \usepackage{exam}

\newcommand{\bquestions}{\begin{questions}}
\newcommand{\equestions}{\end{questions}}
\newcommand{\bsolution}{\begin{solution}}
\newcommand{\esolution}{\end{solution}}
\newcommand{\bparts}{\begin{parts}}
\newcommand{\eparts}{\end{parts}}
\newcommand{\stpart}{\part} % this is absolutely necessary to make new part command
\newcommand{\bsubparts}{\begin{subparts}}
\newcommand{\esubparts}{\end{subparts}}
\newcommand{\stsubpart}{\subpart}

% solution environment is a minipage and cannot support float so, always use "HOLD_position" 
\usepackage{float} % this package is essential for solution with code
% eqnarray is better avoided, most suggestion lead to \align provided in amsmath
% also split is used when there are very long lines of equation
% for different expression of the same equation use line separator \\ and use \notag or \nonumber
\usepackage{eqnarray, amsmath}

% \usepackage{background}
% \usepackage{eso-pic}
% \usepackage{contour}
% 
% \backgroundsetup{
%   angle=90,
%   opacity=0.8,
%   scale=1.2,
%   color=red,
%   nodeanchor=south west,
%   position={current page.south east},
%   contents={}{{\footnotesize Generated by }{\Large Deependra Dhakal }{(\footnotesize to be shared and distributed freely!)}},
%   hshift=40pt,% to move the text vertically
%   vshift=+10pt% to move the text horizontally
% }
% 
% \pagestyle{empty}
\usepackage{wasysym}
\ifLuaTeX
  \usepackage{selnolig}  % disable illegal ligatures
\fi
\IfFileExists{bookmark.sty}{\usepackage{bookmark}}{\usepackage{hyperref}}
\IfFileExists{xurl.sty}{\usepackage{xurl}}{} % add URL line breaks if available
\urlstyle{same} % disable monospaced font for URLs
\hypersetup{
  pdftitle={Essay questions},
  hidelinks,
  pdfcreator={LaTeX via pandoc}}

\title{Essay questions}
\author{Deependra Dhakal}
\date{}

\begin{document}
\maketitle

\hypertarget{permutation}{%
\section{Permutation}\label{permutation}}

\bquestions

\question[4] \label{quest:fifth}

What are the reasons for obtaining hermaphrodite goat population in Chitwan ?

\bsolution

Hermaphrodity in Goat poulation may be due to one or multiple reasons given below:

\begin{enumerate}
\def\labelenumi{\arabic{enumi}.}
\tightlist
\item
  Recessive mutations that remained hidden in population for long time and may have suddenly expressed due to inbreeding or close breeding with exotic breeds.
\item
  Some goat breeds, for example Saneen, are more likely to develop hermaphrodite progeny phenotypes than other breeds.
\item
  Due to aged (old) goat stock kidding a newborn. The old age may accumulate mutations which can affect the gene for hermaphroditism.
\item
  Due to the nature of hermaphroditism recognized in goat breeds -- pseudohermaphrodite. This is a condition in which only external genitalia development is affected. This could also be caused by environmental factors that affect sex expression loci in goats.
\end{enumerate}

\esolution

\question \label{quest:stat1-2}

Wheat do you mean by the random sampling ? Differentiate between stratified sampling and cluster sampling.

\bsolution (Question \ref{quest:stat1-2})

Sampling is the selection of a subset (a statistical sample) of individuals from within a statistical population to estimate characteristics of the whole population.

Most basic form of sampling is the simple random sampling, in which every sample of size n has the same chance of being selected. In addition to simple random sampling, there are other sampling plans that involve randomization and therefore provide a probabilistic basis for inference making.

When the population consists of two or more subpopulations, called strata, a sampling plan that ensures that each subpopulation is represented in the sample is called a stratified random sample.

Another form of random sampling is used when the available units are groups of elements, called clusters. For example, a household is a cluster of individuals living together. A ward, in similar way, might be a convenient sampling unit and might be considered a cluster for a given sampling plan. A cluster sample is hence, a simple random sample of clusters from the available clusters in the population.

\esolution

\equestions

\end{document}
