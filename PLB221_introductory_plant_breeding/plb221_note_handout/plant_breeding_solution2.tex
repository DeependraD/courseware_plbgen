% Options for packages loaded elsewhere
\PassOptionsToPackage{unicode}{hyperref}
\PassOptionsToPackage{hyphens}{url}
%
\documentclass[
  answers,addpoints,12pt]{exam}
\usepackage{lmodern}
\usepackage{amssymb,amsmath}
\usepackage{ifxetex,ifluatex}
\ifnum 0\ifxetex 1\fi\ifluatex 1\fi=0 % if pdftex
  \usepackage[T1]{fontenc}
  \usepackage[utf8]{inputenc}
  \usepackage{textcomp} % provide euro and other symbols
\else % if luatex or xetex
  \usepackage{unicode-math}
  \defaultfontfeatures{Scale=MatchLowercase}
  \defaultfontfeatures[\rmfamily]{Ligatures=TeX,Scale=1}
\fi
% Use upquote if available, for straight quotes in verbatim environments
\IfFileExists{upquote.sty}{\usepackage{upquote}}{}
\IfFileExists{microtype.sty}{% use microtype if available
  \usepackage[]{microtype}
  \UseMicrotypeSet[protrusion]{basicmath} % disable protrusion for tt fonts
}{}
\makeatletter
\@ifundefined{KOMAClassName}{% if non-KOMA class
  \IfFileExists{parskip.sty}{%
    \usepackage{parskip}
  }{% else
    \setlength{\parindent}{0pt}
    \setlength{\parskip}{6pt plus 2pt minus 1pt}}
}{% if KOMA class
  \KOMAoptions{parskip=half}}
\makeatother
\usepackage{xcolor}
\IfFileExists{xurl.sty}{\usepackage{xurl}}{} % add URL line breaks if available
\IfFileExists{bookmark.sty}{\usepackage{bookmark}}{\usepackage{hyperref}}
\hypersetup{
  pdftitle={Numerical solutions to some problems in Plant Breeding},
  hidelinks,
  pdfcreator={LaTeX via pandoc}}
\urlstyle{same} % disable monospaced font for URLs
\usepackage[top=1.5cm,bottom=1.5cm,left=1.5cm,right=1.5cm]{geometry}
\usepackage{longtable,booktabs}
% Correct order of tables after \paragraph or \subparagraph
\usepackage{etoolbox}
\makeatletter
\patchcmd\longtable{\par}{\if@noskipsec\mbox{}\fi\par}{}{}
\makeatother
% Allow footnotes in longtable head/foot
\IfFileExists{footnotehyper.sty}{\usepackage{footnotehyper}}{\usepackage{footnote}}
\makesavenoteenv{longtable}
\usepackage{graphicx}
\makeatletter
\def\maxwidth{\ifdim\Gin@nat@width>\linewidth\linewidth\else\Gin@nat@width\fi}
\def\maxheight{\ifdim\Gin@nat@height>\textheight\textheight\else\Gin@nat@height\fi}
\makeatother
% Scale images if necessary, so that they will not overflow the page
% margins by default, and it is still possible to overwrite the defaults
% using explicit options in \includegraphics[width, height, ...]{}
\setkeys{Gin}{width=\maxwidth,height=\maxheight,keepaspectratio}
% Set default figure placement to htbp
\makeatletter
\def\fps@figure{htbp}
\makeatother
\setlength{\emergencystretch}{3em} % prevent overfull lines
\providecommand{\tightlist}{%
  \setlength{\itemsep}{0pt}\setlength{\parskip}{0pt}}
\setcounter{secnumdepth}{5}
% \usepackage{exam}

\newcommand{\bquestions}{\begin{questions}}
\newcommand{\equestions}{\end{questions}}
\newcommand{\bsolution}{\begin{solution}}
\newcommand{\esolution}{\end{solution}}
\newcommand{\bparts}{\begin{parts}}
\newcommand{\eparts}{\end{parts}}
\newcommand{\stpart}{\part} % this is absolutely necessary to make new part command
\newcommand{\bsubparts}{\begin{subparts}}
\newcommand{\esubparts}{\end{subparts}}
\newcommand{\stsubpart}{\subpart}

% solution environment is a minipage and cannot support float so, always use "HOLD_position" 
\usepackage{float} % this package is essential for solution with code
% eqnarray is better avoided, most suggestion lead to \align provided in amsmath
% also split is used when there are very long lines of equation
% for different expression of the same equation use line separator \\ and use \notag or \nonumber
\usepackage{eqnarray, amsmath}

\usepackage{booktabs}
\usepackage{longtable}
\usepackage{array}
\usepackage{multirow}
\usepackage{wrapfig}
\usepackage{float}
\usepackage{colortbl}
\usepackage{pdflscape}
\usepackage{tabu}
\usepackage{threeparttable}
\usepackage{threeparttablex}
\usepackage[normalem]{ulem}
\usepackage{makecell}

\title{Numerical solutions to some problems in Plant Breeding}
\usepackage{etoolbox}
\makeatletter
\providecommand{\subtitle}[1]{% add subtitle to \maketitle
  \apptocmd{\@title}{\par {\large #1 \par}}{}{}
}
\makeatother
\subtitle{Part II}
\author{Deependra Dhakal\\
Gokuleshwor Agriculture and Animal Science College, Baitadi, Nepal}
\date{9/18/2020}

\begin{document}
\maketitle

\hypertarget{solution-to-exam-qustions}{%
\section{Solution to Exam Qustions}\label{solution-to-exam-qustions}}

\bquestions

\question[4] \label{quest:first}

How do you produce hybrid seed using one self-incompatible (P1) and another self-compatible (P2) parents ?

\bsolution (Question \ref{quest:first})

System of self-incompatibility prevents self fertilization based on the descrimination between self- and non-self pollen. It is known to be controlled by single S-locus with multi-allelic nature demonstrated among different groups of crops (more than 100 families and in approximately 40\% of species). Early instances of overcoming incompatibility is recorded in sweet cherry (\textit{Prunus avium L.}) cultivars through irradiation. However, using the same technique of SI during 1940s and 1950s, Japanese seed companies' introduced hybrid Cabbage gained heights of popularity.

The S-locus encodes both male and female specificity determinants (S-determinants) whose products are predicted to interact and trigger the self/non-self discrimination process. Most types of SI can be classified as sporophytic or gametophytic based on time of gene action in the stamen. The pollen phenotype is determined by the S-genotype of the diploid pollen-parent in sporophytic SI and by the genotype of the individual microspore in gametophytic SI.

In crops exhibiting SI (and even in SC crops), cultivars that serve as pollen donors (``pollenizers'') are usually interspersed throughout orchards since fruit set depends largely on cross-pollinations. Pollenizers are commonly used in canola (\textit{Brassica L.}), sunflower, strawberry and apple.

To achieve hybrid viour or heterosis, two parents with different genetic backgrounds (usually pure lines) need to be crossed. Heterotic F1 progeny show elevated yields as well as other agriculturally desirable traits. Since most cultivated crops are SC, producing hybrid seed requires an efficient system to control pollination to prevent the female parent from self-fertilization. SI has been reported to be preferred over male sterility in crop species with entomophilous pollination since pollen-collecting bees rarely visit male-sterile plants. Nonetheless, SI also may have disadvantages, for example, F1 hybrids of two SI parents are also SI and this may be undesirable for crop production. SI hybrids in such cases have hindered seed production (e.g., oilseed rape/canola) or fruit production. This is in regard of the fact that hybrid production requires maintanence of one or two lines continually over generations. If either of the two parents cannot be self fertilized, nevertheless it needs to be maintained, it is done by transiently breaking down the incompatibility by using chemical or physical techniques.

A system of hybrid breeding utilizing one SI line as female parent (P1) and the other self compatible line as male parent (P2) is shown below:

\begin{figure}[H]

{\centering \includegraphics[width=0.8\linewidth]{plant_breeding_solution2_files/figure-latex/diagram-si-system1-1} 

}

\caption{Maintenance of parental lines. P1 is Self incompatible (SI) while P2 is self compatible (SC). Propagating P1 requires transiently overcoming SI to allow selfing.}\label{fig:diagram-si-system1}
\end{figure}

\begin{figure}[H]

{\centering \includegraphics[width=0.6\textwidth]{plant_breeding_solution2_files/figure-latex/diagram-si-system2-1} 

}

\caption{Production of single cross hybrid using P1 and P2}\label{fig:diagram-si-system2}
\end{figure}

\esolution

\question[4] \label{quest:second}

What are the differences between half-sib and full-sib selection procedure ?

\bsolution

\textbf{Half sib selection}

\begin{itemize}
\tightlist
\item
  Lines composited to form a new population are selected from progeny performance rather than phenotypic appearance.
\item
  Effective in changing gene frequency for highly heritable characters that could be selected visually.
\item
  Not effective for characteristics with low heritability.
\item
  Like mass selection, it is based on maternal plant selection, without pollen control.
\item
  Heritability estimates made from single parent only, and thus are reduced by one-half.
\item
  Includes half sib selection using top cross progeny test, open-pollinated progeny test, some forms of maternal line selection, and the polycross progeny test.
\end{itemize}

\textbf{Full sib selection}

\begin{itemize}
\tightlist
\item
  Crosses are made between selected pair of plants from the source population.
\item
  The seeds of crosses are used both for progeny tests and for reconstituting the new population.
\item
  Measures the combining ability from mating specific individuals -- i.e., both specific as well as general combining abilities are known for the parents used crosses.
\item
  Suitable for improvement of traits with lower heritability.
\item
  Heritability estimates are obtained using information from both parents.
\item
  Includes full sib selection using inbred parents and selfed progeny selection.
\end{itemize}

\esolution

\question[4] \label{quest:third}

In a random mating population, the mean plant height and variance are 120 cm and 121 \(cm^2\) respectively. A plant breeder selected the top 5\% plants from the base population and found mean plant height 110 cm in the next generation. Find the genetic gain, selection differential and the heritability of this trait.

\bsolution

\underline{Numerical}

Given,

Mean (\(X_0\)) = 120

Variation (\(\sigma^2\)) = 121

Mean of next (progeny) generation (\(X_p\)) = 110

Selection intensity (\(k\)) = 0.05

Here,

Response to selection (a.k.a. Genetic gain; R) = -10 (\(\because\) height has decreased in progeny, the gain is negative in value).

Selection differential (S) = \(-\sigma i\) = \(\sqrt{121} \times 2.06\) = -22.66 (\(\because\) selection has reduced the mean phenotype, selection differential is expressed as negative in value).

To estimate heritability (\(h^2\)), we have:

\begin{align}
R &= i \sigma h^{2} \notag \\
|110-120|  &= \sqrt{121} \times 2.06 h^{2} \notag \\
h^2 &= \frac{10}{2.06 \times 11} \notag \\
  &= 0.44 \notag
\end{align}

\esolution

\question[4] \label{quest:fourth}

With the help of following table, answer the following questions.

\begin{table}[H]
\centering
\begin{tabular}{lrrrr}
\toprule
Variety & Virus concentration & Yellowing & Yield with virus & Yield without virus\\
\midrule
A & 100 & 8 & 80 & 90\\
B & 60 & 1 & 100 & 120\\
C & 50 & 3 & 40 & 90\\
\bottomrule
\end{tabular}
\end{table}

\begin{enumerate}
\def\labelenumi{\arabic{enumi}.}
\tightlist
\item
  Which variety is the most susceptible and why ?
\item
  Which variety is the most resistant and why ?
\item
  Which variety is the most tolerant and why ?
\item
  Which variety is the most sensitive and why ?
\end{enumerate}

\bsolution

\begin{enumerate}
\def\labelenumi{\arabic{enumi}.}
\item
  Cultivar A is the most susceptible due to it having the maximum virus concentration.
\item
  Cultivar C is the most resistant because of the least concentration of the virus.
\item
\end{enumerate}

\begin{table}[H]
\centering
\begin{tabular}{lrrl}
\toprule
Variety & Sensitivity & Tolerance & Indicator\\
\midrule
A & 0.001 & 0.999 & Highest tolerance\\
B & 0.003 & 0.997 & \\
C & 0.011 & 0.989 & Highest sensitivity\\
\bottomrule
\end{tabular}
\end{table}

\esolution

\equestions

\end{document}
