% Options for packages loaded elsewhere
\PassOptionsToPackage{unicode}{hyperref}
\PassOptionsToPackage{hyphens}{url}
%
\documentclass[
  answers,addpoints,12pt]{exam}
\usepackage{lmodern}
\usepackage{amssymb,amsmath}
\usepackage{ifxetex,ifluatex}
\ifnum 0\ifxetex 1\fi\ifluatex 1\fi=0 % if pdftex
  \usepackage[T1]{fontenc}
  \usepackage[utf8]{inputenc}
  \usepackage{textcomp} % provide euro and other symbols
\else % if luatex or xetex
  \usepackage{unicode-math}
  \defaultfontfeatures{Scale=MatchLowercase}
  \defaultfontfeatures[\rmfamily]{Ligatures=TeX,Scale=1}
\fi
% Use upquote if available, for straight quotes in verbatim environments
\IfFileExists{upquote.sty}{\usepackage{upquote}}{}
\IfFileExists{microtype.sty}{% use microtype if available
  \usepackage[]{microtype}
  \UseMicrotypeSet[protrusion]{basicmath} % disable protrusion for tt fonts
}{}
\makeatletter
\@ifundefined{KOMAClassName}{% if non-KOMA class
  \IfFileExists{parskip.sty}{%
    \usepackage{parskip}
  }{% else
    \setlength{\parindent}{0pt}
    \setlength{\parskip}{6pt plus 2pt minus 1pt}}
}{% if KOMA class
  \KOMAoptions{parskip=half}}
\makeatother
\usepackage{xcolor}
\IfFileExists{xurl.sty}{\usepackage{xurl}}{} % add URL line breaks if available
\IfFileExists{bookmark.sty}{\usepackage{bookmark}}{\usepackage{hyperref}}
\hypersetup{
  pdftitle={Numerical solutions to some problems in Plant Breeding},
  hidelinks,
  pdfcreator={LaTeX via pandoc}}
\urlstyle{same} % disable monospaced font for URLs
\usepackage[top=1.5cm,bottom=1.5cm,left=1.5cm,right=1.5cm]{geometry}
\usepackage{longtable,booktabs}
% Correct order of tables after \paragraph or \subparagraph
\usepackage{etoolbox}
\makeatletter
\patchcmd\longtable{\par}{\if@noskipsec\mbox{}\fi\par}{}{}
\makeatother
% Allow footnotes in longtable head/foot
\IfFileExists{footnotehyper.sty}{\usepackage{footnotehyper}}{\usepackage{footnote}}
\makesavenoteenv{longtable}
\usepackage{graphicx}
\makeatletter
\def\maxwidth{\ifdim\Gin@nat@width>\linewidth\linewidth\else\Gin@nat@width\fi}
\def\maxheight{\ifdim\Gin@nat@height>\textheight\textheight\else\Gin@nat@height\fi}
\makeatother
% Scale images if necessary, so that they will not overflow the page
% margins by default, and it is still possible to overwrite the defaults
% using explicit options in \includegraphics[width, height, ...]{}
\setkeys{Gin}{width=\maxwidth,height=\maxheight,keepaspectratio}
% Set default figure placement to htbp
\makeatletter
\def\fps@figure{htbp}
\makeatother
\setlength{\emergencystretch}{3em} % prevent overfull lines
\providecommand{\tightlist}{%
  \setlength{\itemsep}{0pt}\setlength{\parskip}{0pt}}
\setcounter{secnumdepth}{5}
% \usepackage{exam}

\newcommand{\bquestions}{\begin{questions}}
\newcommand{\equestions}{\end{questions}}
\newcommand{\bsolution}{\begin{solution}}
\newcommand{\esolution}{\end{solution}}
\newcommand{\bparts}{\begin{parts}}
\newcommand{\eparts}{\end{parts}}
\newcommand{\stpart}{\part} % this is absolutely necessary to make new part command
\newcommand{\bsubparts}{\begin{subparts}}
\newcommand{\esubparts}{\end{subparts}}
\newcommand{\stsubpart}{\subpart}

% solution environment is a minipage and cannot support float so, always use "HOLD_position" 
\usepackage{float} % this package is essential for solution with code

% eqnarray is better avoided, most suggestion lead to \align provided in amsmath
% also split is used when there are very long lines of equation
% for different expression of the same equation use line separator \\ and use \notag or \nonumber
\usepackage{eqnarray, amsmath}

\usepackage{booktabs}
\usepackage{longtable}
\usepackage{array}
\usepackage{multirow}
\usepackage{wrapfig}
\usepackage{float}
\usepackage{colortbl}
\usepackage{pdflscape}
\usepackage{tabu}
\usepackage{threeparttable}
\usepackage{threeparttablex}
\usepackage[normalem]{ulem}
\usepackage{makecell}
\usepackage{xcolor}
\usepackage{amsmath}
\usepackage{caption}

\title{Numerical solutions to some problems in Plant Breeding}
\author{Deependra Dhakal\\
Gokuleshwor Agriculture and Animal Science College, Baitadi, Nepal}
\date{9/18/2020}

\begin{document}
\maketitle

\hypertarget{solution-to-exam-qustions}{%
\section{Solution to Exam Qustions}\label{solution-to-exam-qustions}}

\bquestions

\question[4] \label{quest:first}

Mention different factors that affect gene and genotype frequencies. A population consisting of 10,000 individual, 25 individuals are of ``aa'' genotype. If the population is in Hardy-Weinberg equilibrium, find the gene and genotype frequencies of that population.

\bsolution (Question \ref{quest:first})

Refer to lecture note on Genetic Composition of Cross Pollinated Crops for factors affecting gene and genotype frequencies.

\underline{Numerical}

Total population (\(N\)) = 10000

Number of individuals of \(aa\) genotype = 25

Since the population is in HW equilibrium, gene and genotype frequencies of the population stay constant over generations, i.e.

Gene frequencies = \(p^2 + q^2 = 1\), and

Genotype frequencies = \(p^2AA + 2pqAa + q^2aa = 1\)

So,

\(\rightarrow\) genotype frequency of homozygous recessive \(aa\)(\(q^2\)) = \(\frac{25}{10000}\) = 0.0025

\(\rightarrow\) gene frequency of minor allele \(a\)(\(q\)) = \(\sqrt{\frac{25}{10000}}\) = 0.05

\(\rightarrow\) gene frequency of major allele \(A\)(\(p\)) = \(1 - p\) = 0.95

\(\rightarrow\) genotype frequency of heterogygous \(Aa\)(\(2pq\)) = 0.095

\(\rightarrow\) genotype frequency of homozygous dominant \(AA\)(\(p^2\)) = 0.9025

\esolution

\question[4] Calculate gametic and phenotypic frequencies of GGGg individuals after selfing. Also, write types of genotype and phenotype.

\bsolution

Let us assume tetraploid from only bivalents during pairing. If we start with a GGGg plant and self it we can deduce the frequency of possible gametes for the given genotype as:

\begin{figure}[H]
\includegraphics[width=0.5\linewidth]{plant_breeding_solution1_files/figure-latex/diagram-gamete-1} \caption{Pairing of chromosomes of gametes in a tetraploid of genotype AAAa}\label{fig:diagram-gamete}
\end{figure}

Here, following proportion of gametes are formed:

\begin{table}[H]

\caption{\label{tab:gamete-proportion}Proportion of gametes formed by segregation of bivalents in a tetraploid of genotype AAAa}
\centering
\begin{tabular}[t]{lrr}
\toprule
Gamete & Number & Proportion\\
\midrule
Aa & 3 & 0.5\\
AA & 3 & 0.5\\
\bottomrule
\end{tabular}
\end{table}

Now, if selfing were to be performed in the individual of genotype of AAAa, following outcome of progeny genotypes is possible:

\captionsetup[table]{labelformat=empty,skip=1pt}
\begin{longtable}{lll}
\toprule
& \multicolumn{2}{c}{Male} \\ 
 \cmidrule(lr){2-3}
 & AA & Aa \\ 
\midrule
\multicolumn{1}{l}{Female} \\ 
\midrule
AA & AAAA & AAAa \\ 
Aa & AaAA & AaAa \\ 
\bottomrule
\end{longtable}

It could be observed that out of 3 genotypes are possible and (based on dominance hypothesis) only one phenotype results in the progeny.

\esolution

\question[4] What are advantages of partial diallel cross ? Make a partial diallel scheme involving 8 parents.
\label{quest:diallel}

\bsolution (Question \ref{quest:diallel})

Features of a diallel cross design are:

\begin{itemize}
\item Not all the crosses are made.
\item There are no reciprocals or selfs.
\end{itemize}

The goal is to reduce the breeding workload for a given sample of parents by making less than \(\frac{n(n-1)}{2}\) crosses for n parents.

With the partial scheme of diallel cross, same number of parents could be tested in a framework with fewer crosses. The number of crosses in a partial diallel scheme is given by,

\[
x = \frac{n \times s}{2}
\]

By analogy, we could have said that large number of lines could be tested with lesser crossings under diallel scheme. Rearranging the above relationship, \(\large n = \frac{2x}{s}\).

Further, there is restriction as to how \(s\) should is selected, for example, is \(s = n -1\) it gives half diallel cross. \(s\) is a whole number greater than or equal to 2, and \(k\) is a whole number -- \(k = \frac{n + 1 - s}{2}\).

Here (Table \ref{tab:partial-diallel-8p-table}) is for example diallel cross involving 8 parental lines and 5 set of cross involving each parent,

\begingroup\fontsize{8}{10}\selectfont

\begin{longtable}[t]{rllllllll}
\caption{\label{tab:partial-diallel-8p-table}Partial diallel cross involving 8 parents combined in 5 set of crossess each.}\\
\toprule
p & 1 & 2 & 3 & 4 & 5 & 6 & 7 & 8\\
\midrule
\rowcolor{gray!6}  1 &  &  & 3 x 1 & 4 x 1 & 5 x 1 & 6 x 1 & 7 x 1 & \\
2 &  &  &  & 4 x 2 & 5 x 2 & 6 x 2 & 7 x 2 & 8 x 2\\
\rowcolor{gray!6}  3 &  &  &  &  & 5 x 3 & 6 x 3 & 7 x 3 & 8 x 3\\
4 &  &  &  &  &  & 6 x 4 & 7 x 4 & 8 x 4\\
\rowcolor{gray!6}  5 &  &  &  &  &  &  & 7 x 5 & 8 x 5\\
\addlinespace
6 &  &  &  &  &  &  &  & 8 x 6\\
\rowcolor{gray!6}  7 &  &  &  &  &  &  &  & \\
8 &  &  &  &  &  &  &  & \\
\bottomrule
\end{longtable}
\endgroup{}

\esolution

\question[4] For a quantitative trait in RMP, mean is 100 and variation is 240. The regression of the offspring on mid-parent value is 0.25. Truncation selection is practiced with a selection differential of 32. What is the expected mean in the next generation ?

\bsolution

\underline{Numerical}

Given,

Mean (\(\mu_0\)) = 100

Variation (\(\sigma^2\)) = 240

Regression of offspring on mid-parent value (\(b\)) = 0.25

Selection differential (\(\mu_s - \mu_0\)) = 32

We know that the performance of a selected population after applying selection differential:

\begin{align}
S &= \sqrt{\sigma^2} i \notag \\
  &= \sqrt{240} i \notag \\
i &= \frac{S}{\sqrt{240}} \notag
\end{align}

Thus, \(\rightarrow\) \(i\) = 2.0655911. And we know that \(b = h^2_{ns}\), so:

\(\rightarrow\) \(R = \sigma i h^2\) = 8

Hence, the expected mean of next (progeny) generation (\(P\)) = \(\mu_0 + \sigma i h^2\) = 108.

\esolution
\equestions

\end{document}
