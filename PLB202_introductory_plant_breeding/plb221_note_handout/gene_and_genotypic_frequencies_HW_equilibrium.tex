\documentclass[nofonts,]{tufte-handout}

% ams
\usepackage{amssymb,amsmath}

\usepackage{ifxetex,ifluatex}
\usepackage{fixltx2e} % provides \textsubscript
\ifnum 0\ifxetex 1\fi\ifluatex 1\fi=0 % if pdftex
  \usepackage[T1]{fontenc}
  \usepackage[utf8]{inputenc}
\else % if luatex or xelatex
  \makeatletter
  \@ifpackageloaded{fontspec}{}{\usepackage{fontspec}}
  \makeatother
  \defaultfontfeatures{Ligatures=TeX,Scale=MatchLowercase}
  \makeatletter
  \@ifpackageloaded{soul}{
     \renewcommand\allcapsspacing[1]{{\addfontfeature{LetterSpace=15}#1}}
     \renewcommand\smallcapsspacing[1]{{\addfontfeature{LetterSpace=10}#1}}
   }{}
  \makeatother
\fi

% graphix
\usepackage{graphicx}
\setkeys{Gin}{width=\linewidth,totalheight=\textheight,keepaspectratio}

% booktabs
\usepackage{booktabs}

% url
\usepackage{url}

% hyperref
\usepackage{hyperref}

% units.
\usepackage{units}


\setcounter{secnumdepth}{-1}

% citations
\usepackage{natbib}
\bibliographystyle{plainnat}

%% tint override
\setcitestyle{round} 

% pandoc syntax highlighting

% longtable

% multiplecol
\usepackage{multicol}

% strikeout
\usepackage[normalem]{ulem}

% morefloats
\usepackage{morefloats}


% tightlist macro required by pandoc >= 1.14
\providecommand{\tightlist}{%
  \setlength{\itemsep}{0pt}\setlength{\parskip}{0pt}}

% title / author / date
\title{Gene and alleles and equilibrium of population}
\author{Deependra Dhakal}
\date{2019-06-12}

%% -- tint overrides
%% fonts, using roboto (condensed) as default
\usepackage[sfdefault,condensed]{roboto}
%% also nice: \usepackage[default]{lato}

%% colored links, setting 'borrowed' from RJournal.sty with 'Thanks, Achim!'
\RequirePackage{color}
\definecolor{link}{rgb}{0.1,0.1,0.8} %% blue with some grey
\hypersetup{
  colorlinks,%
  citecolor=link,%
  filecolor=link,%
  linkcolor=link,%
  urlcolor=link
}

%% macros
\makeatletter

%% -- tint does not use italics or allcaps in title
\renewcommand{\maketitle}{%     
  \newpage
  \global\@topnum\z@% prevent floats from being placed at the top of the page
  \begingroup
    \setlength{\parindent}{0pt}%
    \setlength{\parskip}{4pt}%
    \let\@@title\@empty
    \let\@@author\@empty
    \let\@@date\@empty
    \ifthenelse{\boolean{@tufte@sfsidenotes}}{%
      %\gdef\@@title{\sffamily\LARGE\allcaps{\@title}\par}%
      %\gdef\@@author{\sffamily\Large\allcaps{\@author}\par}%
      %\gdef\@@date{\sffamily\Large\allcaps{\@date}\par}%
      \gdef\@@title{\begingroup\fontseries{b}\selectfont\LARGE{\@title}\par}%
      \gdef\@@author{\begingroup\fontseries{l}\selectfont\Large{\@author}\par}%
      \gdef\@@date{\begingroup\fontseries{l}\selectfont\Large{\@date}\par}%
    }{%
      %\gdef\@@title{\LARGE\itshape\@title\par}%
      %\gdef\@@author{\Large\itshape\@author\par}%
      %\gdef\@@date{\Large\itshape\@date\par}%
      \gdef\@@title{\begingroup\fontseries{b}\selectfont\LARGE\@title\par\endgroup}%
      \gdef\@@author{\begingroup\fontseries{l}\selectfont\Large\@author\par\endgroup}%
      \gdef\@@date{\begingroup\fontseries{l}\selectfont\Large\@date\par\endgroup}%
    }%
    \@@title
    \@@author
    \@@date
  \endgroup
  \thispagestyle{plain}% suppress the running head
  \tuftebreak% add some space before the text begins
  \@afterindentfalse\@afterheading% suppress indentation of the next paragraph
}

%% -- tint does not use italics or allcaps in section/subsection/paragraph
\titleformat{\section}%
  [hang]% shape
  %{\normalfont\Large\itshape}% format applied to label+text
  {\fontseries{b}\selectfont\Large}% format applied to label+text
  {\thesection}% label
  {1em}% horizontal separation between label and title body
  {}% before the title body
  []% after the title body

\titleformat{\subsection}%
  [hang]% shape
  %{\normalfont\large\itshape}% format applied to label+text
  {\fontseries{m}\selectfont\large}% format applied to label+text
  {\thesubsection}% label
  {1em}% horizontal separation between label and title body
  {}% before the title body
  []% after the title body

\titleformat{\paragraph}%
  [runin]% shape
  %{\normalfont\itshape}% format applied to label+text
  {\fontseries{l}\selectfont}% format applied to label+text
  {\theparagraph}% label
  {1em}% horizontal separation between label and title body
  {}% before the title body
  []% after the title body

%% -- tint does not use italics here either
% Formatting for main TOC (printed in front matter)
% {section} [left] {above} {before w/label} {before w/o label} {filler + page} [after]
\ifthenelse{\boolean{@tufte@toc}}{%
  \titlecontents{part}% FIXME
    [0em] % distance from left margin
    %{\vspace{1.5\baselineskip}\begin{fullwidth}\LARGE\rmfamily\itshape} % above (global formatting of entry)
    {\vspace{1.5\baselineskip}\begin{fullwidth}\fontseries{m}\selectfont\LARGE} % above (global formatting of entry)
    {\contentslabel{2em}} % before w/label (label = ``II'')
    {} % before w/o label
    {\rmfamily\upshape\qquad\thecontentspage} % filler + page (leaders and page num)
    [\end{fullwidth}] % after
  \titlecontents{chapter}%
    [0em] % distance from left margin
    %{\vspace{1.5\baselineskip}\begin{fullwidth}\LARGE\rmfamily\itshape} % above (global formatting of entry)
    {\vspace{1.5\baselineskip}\begin{fullwidth}\fontseries{m}\selectfont\LARGE} % above (global formatting of entry)
    {\hspace*{0em}\contentslabel{2em}} % before w/label (label = ``2'')
    {\hspace*{0em}} % before w/o label
    %{\rmfamily\upshape\qquad\thecontentspage} % filler + page (leaders and page num)
    {\upshape\qquad\thecontentspage} % filler + page (leaders and page num)
    [\end{fullwidth}] % after
  \titlecontents{section}% FIXME
    [0em] % distance from left margin
    %{\vspace{0\baselineskip}\begin{fullwidth}\Large\rmfamily\itshape} % above (global formatting of entry)
    {\vspace{0\baselineskip}\begin{fullwidth}\fontseries{m}\selectfont\Large} % above (global formatting of entry)
    {\hspace*{2em}\contentslabel{2em}} % before w/label (label = ``2.6'')
    {\hspace*{2em}} % before w/o label
    %{\rmfamily\upshape\qquad\thecontentspage} % filler + page (leaders and page num)
    {\upshape\qquad\thecontentspage} % filler + page (leaders and page num)
    [\end{fullwidth}] % after
  \titlecontents{subsection}% FIXME
    [0em] % distance from left margin
    %{\vspace{0\baselineskip}\begin{fullwidth}\large\rmfamily\itshape} % above (global formatting of entry)
    {\vspace{0\baselineskip}\begin{fullwidth}\fontseries{m}\selectfont\large} % above (global formatting of entry)
    {\hspace*{4em}\contentslabel{4em}} % before w/label (label = ``2.6.1'')
    {\hspace*{4em}} % before w/o label
    %{\rmfamily\upshape\qquad\thecontentspage} % filler + page (leaders and page num)
    {\upshape\qquad\thecontentspage} % filler + page (leaders and page num)
    [\end{fullwidth}] % after
  \titlecontents{paragraph}% FIXME
    [0em] % distance from left margin
    %{\vspace{0\baselineskip}\begin{fullwidth}\normalsize\rmfamily\itshape} % above (global formatting of entry)
    {\vspace{0\baselineskip}\begin{fullwidth}\fontseries{m}\selectfont\normalsize\rmfamily} % above (global formatting of entry)
    {\hspace*{6em}\contentslabel{2em}} % before w/label (label = ``2.6.0.0.1'')
    {\hspace*{6em}} % before w/o label
    %{\rmfamily\upshape\qquad\thecontentspage} % filler + page (leaders and page num)
    {\upshape\qquad\thecontentspage} % filler + page (leaders and page num)
    [\end{fullwidth}] % after
}{}

  
\makeatother


\usepackage{booktabs}
\usepackage{longtable}
\usepackage{array}
\usepackage{multirow}
\usepackage{wrapfig}
\usepackage{float}
\usepackage{colortbl}
\usepackage{pdflscape}
\usepackage{tabu}
\usepackage{threeparttable}
\usepackage{threeparttablex}
\usepackage[normalem]{ulem}
\usepackage{makecell}
\usepackage{xcolor}

\begin{document}

\maketitle




\clearpage

\hypertarget{gene-and-genotypes}{%
\section{Gene and genotypes}\label{gene-and-genotypes}}

\hypertarget{what-is-allelegene}{%
\subsection{What is allele/gene ?}\label{what-is-allelegene}}

An allele/gene is the bit of DNA at the place called locus (the place on
a chromosome where an allele resides). An allele is instantiation of a
locus. But by orthology, a locus is not template for an allele.
Similarly, a locus is not tangible, rather a map describing where to
find a tangible thing, an allele on a chromosome. A diploid individual
has two alleles at a particular autosomal locus.

\hypertarget{mechanisms-by-which-alleles-at-same-locus-changes}{%
\subsection{Mechanisms by which alleles at same locus
changes}\label{mechanisms-by-which-alleles-at-same-locus-changes}}

\begin{enumerate}
\def\labelenumi{\arabic{enumi}.}
\item
  By origin: Same locus but different chromosome.
\item
  By state: It is indicative of the context they are put in. i.e.~DNA
  sequence or amino acid sequence. Same amino acid sequence in some
  alleles may arise due to different DNA sequences (Redundancy of
  genetic code).
\item
  By descent: In practice, we are often concerned with relatively short
  time in past and are content to say that two alleles differ by descent
  if they do not share common ancestor in say, the past 10 generations.
  Two alleles different by descent may or may not be different by state
  because of mutation.
\end{enumerate}

Converse of the mechanisms which cause differences in alleles are termed
as identical by origin, stage or descent. Diploid individuals are said
to be heterozygous at a locus if two alleles at that locus are different
by state. If we are studying proteins, we may call an individual
homozygous at a locus when the protein sequence of the two alleles are
identical, even if their DNA sequences differ.

\begin{figure}
\includegraphics{gene_and_genotypic_frequencies_HW_equilibrium_files/figure-latex/unnamed-chunk-1-1} \caption[An example of pedigree tree showing single pedigree family with 14 subjects]{An example of pedigree tree showing single pedigree family with 14 subjects}\label{fig:unnamed-chunk-1}
\end{figure}

\hypertarget{kreitmans-sample}{%
\subsection{Kreitman's sample}\label{kreitmans-sample}}

Kreitman's sample contain II alleles that differ by origin. How many
alleles differ by state? If we were interested in the full DNA sequence,
the sample contains six alleles that differ by state. If we were
interested in the proteins, then sample contains only two alleles that
differ by state. Of the two protein alleles, the one with a lysine at
position 192 makes up \(\frac{6}{10} = 0.55\) of the alleles. The usual
way to say this is that the allele frequency is an estimate of the
population allele frequency. It's not a particularly precise estimate
because of the small sample size. A rough approximation to the 95\%
confidence interval for a proportion is,

\[
\hat{p} \pm 1.96 \sqrt{\frac{\hat{p}(1-\hat{p})}{n}}
\]

Where \(\hat{p}\) is the estimate of proportion, 0.55 in our case and
\(n\) is the sample size. Thus, the probability that the populatoin
allele frequency falls within the interval (0.26, 0.84) is 0.95. If more
precise estimate is needed, the sample size would have to be increased.

\hypertarget{genotype-and-allele-frequencies}{%
\subsection{Genotype and allele
frequencies}\label{genotype-and-allele-frequencies}}

In A loci, suppose, two alleles \(A_1\) and \(A_2\) are present in a
diploid organism the genotype and genotypic frequency of segregating
population will be;

\[
\begin{aligned}
Genotype \hspace{20pt} & A_1A_1 & A_1A_2 \hspace{20pt} & A_2A_2 \\
Relative frequency \hspace{20pt} & x_{11} & x_{12} \hspace{20pt} & x_{22}
\end{aligned}
\]

As, relative frequencies must add to 1,

\[
x_{11} + x_{12} + x_{22} = 1
\]

The order of subscripting heterozygous is arbitrary.

Frequency of A\_1 allele in the population is,

\[
p = x_{11} + \frac{1}{2}x_{12}
\]

and frequency of A\_2 allele is,

\[
q = 1-p = x_{22} + \frac{1}{2}x_{12}
\]

Measure of each allele frequency can be thought of as independent
events. For e.g., for allele \(p\) to be selected;

\[
p = \left(x_{11} \times \frac{1}{P(p_{A_1A_1})}\right) + \left(x_{12} \times \frac{1}{2}\right) + (x_{22} \times 0)
\]

Where, \(P(p, A_1A_1)\) is the probability of getting \(p\) allele from
\(A_1A_1\) genotype, for loci with more than two alleles, frequency of
\(i^{th}\) allele will be called \(p_i\). Frequency of \(A_iA_j\)
genotype will be called \(x_ij\) for heterozygotes, \(i\neq j\) and, by
convention, \(i<j\). If there are \(n\) alleles,

\[
\begin{aligned}
1 &= x_{11} + x_{22} + x_{33} + ... + x_{nn} + x_{12} + x_{13} + x_{(n-1)n} \\
  &= \sum^n_{i=1}\sum^n_{j\geq i}{x_{ij}}
\end{aligned}
\]

The frequency of \(i^th\) allele is

\[
p_i = x_{ii} + \frac{1}{2}\sum^{i-1}_{j = 1}{x_{ji}} + \frac{1}{2}\sum^n_{j = i+1}{x_{ij}}
\]

\hypertarget{problems}{%
\subsection{Problems}\label{problems}}

\begin{enumerate}
\def\labelenumi{\arabic{enumi}.}
\tightlist
\item
  How many different genotypes are there at a locus with n alleles that
  differ by state?
\end{enumerate}

When there are n alleles, there are n homozygous genotypes, \(A_iA_i\),
\(i = 1, 2, ...., n\). If we first view an \(A_iA_j\) heterozygote as
distinct from an \(A_jA_i\) heterozygote, there are \(n(n-1)\) such
heterozygotes. The actual number of heterozygotes will be one half this
number, or \(\frac{n(n-1)}{2}\). Thus, the total number of genotypes is
\(\frac{n+n(n-1)}{2}= \frac{n(n+1)}{2}\).

\begin{enumerate}
\def\labelenumi{\arabic{enumi}.}
\tightlist
\item
  Derive the hardy weinberg law for a sex-linked locus. Let the initial
  frequency A, in female be \(p_f\) and in males be \(p_m\). Follow the
  two allele frequencies in successive generations untill you understand
  the allele frequency dynamics. Then, jump ahead and find the
  equilibrium genotype frequencies in females and males. Finally, graph
  the male and female allele frequencies over several generations for a
  population that is started with all \(A_1A_1\) females (\(p_f = 1\))
  and A\_2 males (\(p_m = 0\))
\end{enumerate}

As males get their X-chromosomes from their mother, the frequency of
A\_1 in males is always equal to the frequency in females in the
previous generation. As a female gets one X from her mother and one from
her father, the allele freq in females is always the average of the male
and female frequencies in the previous generation. Thus, the allele
frequencies over the first three generation are as follows.

\begin{tabular}{rlll}
\toprule
Generation & Females & Males & Female-male\\
\midrule
1 & $p_f$ & $p_m$ & $p_f - p_m$\\
2 & $\frac{p_f + p_m}{2}$ & $p_m$ & $-\frac{p_f -p_m}{2}$\\
3 & $\frac{p_f + \frac{p_f + p_m}{2}}{2}$ & $\frac{p_f+ p_m}{2}$ & $\frac{p_f -p_m}{2}$\\
\bottomrule
\end{tabular}

Two important things emerge from the table. First, the overall allele
frequecy,

\[
p = \frac{2}{3}p_f + \frac{1}{3}p_m
\]

does not change over time. (Convince yourself that this is so by
calculating p in generations 2 and 3). Second, the difference between
the allele frequencies in females and males is halved each generation,
as recorded in table. Taken together, these two observations show that
eventually the allele frequencies in male and females will converge to
\(p\). At that time, the genotype frequencies in females will be at
Hardy-Weinberg frequencies.

\hypertarget{example-case}{%
\subsection{Example case}\label{example-case}}

A population is consisted of 200 plants. Out of them, 100 plants are of
Aa, 50 plants are of AA and 50 plants are of aa genotypes. This is a
random mating population and in this population the frequencies of these
three genotypes are at H-W equilibrium state. After \(5^{th}\)
generations of random mating, plants having genotypes AA, Aa and aa are
found in 500, 300 and 200 numbers respectively. Are they still in H-W
equilibrium? Test the result with the help of \(\chi^2\) goodness of fit
test.

\(\Large \rightarrow\) Here, the population of 200 plants is stated to
be in H-W equilibrium; we already have equilibrium frequencies. Hence a
\(\chi^2\) test for would show whether or not both the populations are
same or have diverged from H-W equilibrium state (i.e.~observed frequncy
of population after 5th generation is same or different than expected
population frequency at initial condition). For facilitating comparison,
we convert the given frequencies of observed genotypes (that of
\(5^{th}\) generation) to the add upto current population count (200
individual).

Thus observed frequencies are AA: 100; Aa: 60 and aa: 40.

Note, however, we commonly compute the expected frequency based on the
expected ratios. Therefore it also imperative to show the expected
frequency as the proportion of total count of observed frequency.

Now we construct contingency table, as shown in Table
\ref{tab:hw-equilibrium-independence-chi}.

\begin{longtable}{llrrr}
\caption{\label{tab:hw-equilibrium-independence-chi}2x3 contingency table of frequency of genotypes at equilibrium generation and at 5th generation of mating}\\
\toprule
\multicolumn{2}{c}{  } & \multicolumn{3}{c}{Genotype frequency} \\
\cmidrule(l{3pt}r{3pt}){3-5}
  &   & Dominant (AA) & Homozygous dominant (Aa) & Recessive (aa)\\
\midrule
 & $1^{st}$ & 100 & 50 & 50\\
\cmidrule{2-5}
\multirow{-2}{*}{\raggedright\arraybackslash Generation} & $5^{th}$ & 100 & 60 & 40\\
\bottomrule
\end{longtable}

Here since the number of df is 2, we do not apply the Yate's correction.
After computation, we find \(\chi^2\) = 2.020202 with probability of
0.3641822 which is well within the confidence band of 0.95 to 0.05. We
fail to reject the null hypothesis that two observations were taken from
same populations. Thus, we conclude that even after \(5^{th}\)
generation of mating the population continues to be in HW equilibrium
state.

\bibliography{skeleton.bib,bibliography.bib}



\end{document}
