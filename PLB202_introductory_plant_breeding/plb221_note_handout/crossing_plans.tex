\documentclass[nofonts,]{tufte-handout}

% ams
\usepackage{amssymb,amsmath}

\usepackage{ifxetex,ifluatex}
\usepackage{fixltx2e} % provides \textsubscript
\ifnum 0\ifxetex 1\fi\ifluatex 1\fi=0 % if pdftex
  \usepackage[T1]{fontenc}
  \usepackage[utf8]{inputenc}
\else % if luatex or xelatex
  \makeatletter
  \@ifpackageloaded{fontspec}{}{\usepackage{fontspec}}
  \makeatother
  \defaultfontfeatures{Ligatures=TeX,Scale=MatchLowercase}
  \makeatletter
  \@ifpackageloaded{soul}{
     \renewcommand\allcapsspacing[1]{{\addfontfeature{LetterSpace=15}#1}}
     \renewcommand\smallcapsspacing[1]{{\addfontfeature{LetterSpace=10}#1}}
   }{}
  \makeatother
\fi

% graphix
\usepackage{graphicx}
\setkeys{Gin}{width=\linewidth,totalheight=\textheight,keepaspectratio}

% booktabs
\usepackage{booktabs}

% url
\usepackage{url}

% hyperref
\usepackage{hyperref}

% units.
\usepackage{units}


\setcounter{secnumdepth}{-1}

% citations
\usepackage{natbib}
\bibliographystyle{plainnat}

%% tint override
\setcitestyle{round} 

% pandoc syntax highlighting

% longtable
\usepackage{longtable,booktabs}

% multiplecol
\usepackage{multicol}

% strikeout
\usepackage[normalem]{ulem}

% morefloats
\usepackage{morefloats}


% tightlist macro required by pandoc >= 1.14
\providecommand{\tightlist}{%
  \setlength{\itemsep}{0pt}\setlength{\parskip}{0pt}}

% title / author / date
\title{Crossing plans}
\author{Deependra Dhakal}
\date{2020-07-23}

%% -- tint overrides
%% fonts, using roboto (condensed) as default
\usepackage[sfdefault,condensed]{roboto}
%% also nice: \usepackage[default]{lato}

%% colored links, setting 'borrowed' from RJournal.sty with 'Thanks, Achim!'
\RequirePackage{color}
\definecolor{link}{rgb}{0.1,0.1,0.8} %% blue with some grey
\hypersetup{
  colorlinks,%
  citecolor=link,%
  filecolor=link,%
  linkcolor=link,%
  urlcolor=link
}

%% macros
\makeatletter

%% -- tint does not use italics or allcaps in title
\renewcommand{\maketitle}{%     
  \newpage
  \global\@topnum\z@% prevent floats from being placed at the top of the page
  \begingroup
    \setlength{\parindent}{0pt}%
    \setlength{\parskip}{4pt}%
    \let\@@title\@empty
    \let\@@author\@empty
    \let\@@date\@empty
    \ifthenelse{\boolean{@tufte@sfsidenotes}}{%
      %\gdef\@@title{\sffamily\LARGE\allcaps{\@title}\par}%
      %\gdef\@@author{\sffamily\Large\allcaps{\@author}\par}%
      %\gdef\@@date{\sffamily\Large\allcaps{\@date}\par}%
      \gdef\@@title{\begingroup\fontseries{b}\selectfont\LARGE{\@title}\par}%
      \gdef\@@author{\begingroup\fontseries{l}\selectfont\Large{\@author}\par}%
      \gdef\@@date{\begingroup\fontseries{l}\selectfont\Large{\@date}\par}%
    }{%
      %\gdef\@@title{\LARGE\itshape\@title\par}%
      %\gdef\@@author{\Large\itshape\@author\par}%
      %\gdef\@@date{\Large\itshape\@date\par}%
      \gdef\@@title{\begingroup\fontseries{b}\selectfont\LARGE\@title\par\endgroup}%
      \gdef\@@author{\begingroup\fontseries{l}\selectfont\Large\@author\par\endgroup}%
      \gdef\@@date{\begingroup\fontseries{l}\selectfont\Large\@date\par\endgroup}%
    }%
    \@@title
    \@@author
    \@@date
  \endgroup
  \thispagestyle{plain}% suppress the running head
  \tuftebreak% add some space before the text begins
  \@afterindentfalse\@afterheading% suppress indentation of the next paragraph
}

%% -- tint does not use italics or allcaps in section/subsection/paragraph
\titleformat{\section}%
  [hang]% shape
  %{\normalfont\Large\itshape}% format applied to label+text
  {\fontseries{b}\selectfont\Large}% format applied to label+text
  {\thesection}% label
  {1em}% horizontal separation between label and title body
  {}% before the title body
  []% after the title body

\titleformat{\subsection}%
  [hang]% shape
  %{\normalfont\large\itshape}% format applied to label+text
  {\fontseries{m}\selectfont\large}% format applied to label+text
  {\thesubsection}% label
  {1em}% horizontal separation between label and title body
  {}% before the title body
  []% after the title body

\titleformat{\paragraph}%
  [runin]% shape
  %{\normalfont\itshape}% format applied to label+text
  {\fontseries{l}\selectfont}% format applied to label+text
  {\theparagraph}% label
  {1em}% horizontal separation between label and title body
  {}% before the title body
  []% after the title body

%% -- tint does not use italics here either
% Formatting for main TOC (printed in front matter)
% {section} [left] {above} {before w/label} {before w/o label} {filler + page} [after]
\ifthenelse{\boolean{@tufte@toc}}{%
  \titlecontents{part}% FIXME
    [0em] % distance from left margin
    %{\vspace{1.5\baselineskip}\begin{fullwidth}\LARGE\rmfamily\itshape} % above (global formatting of entry)
    {\vspace{1.5\baselineskip}\begin{fullwidth}\fontseries{m}\selectfont\LARGE} % above (global formatting of entry)
    {\contentslabel{2em}} % before w/label (label = ``II'')
    {} % before w/o label
    {\rmfamily\upshape\qquad\thecontentspage} % filler + page (leaders and page num)
    [\end{fullwidth}] % after
  \titlecontents{chapter}%
    [0em] % distance from left margin
    %{\vspace{1.5\baselineskip}\begin{fullwidth}\LARGE\rmfamily\itshape} % above (global formatting of entry)
    {\vspace{1.5\baselineskip}\begin{fullwidth}\fontseries{m}\selectfont\LARGE} % above (global formatting of entry)
    {\hspace*{0em}\contentslabel{2em}} % before w/label (label = ``2'')
    {\hspace*{0em}} % before w/o label
    %{\rmfamily\upshape\qquad\thecontentspage} % filler + page (leaders and page num)
    {\upshape\qquad\thecontentspage} % filler + page (leaders and page num)
    [\end{fullwidth}] % after
  \titlecontents{section}% FIXME
    [0em] % distance from left margin
    %{\vspace{0\baselineskip}\begin{fullwidth}\Large\rmfamily\itshape} % above (global formatting of entry)
    {\vspace{0\baselineskip}\begin{fullwidth}\fontseries{m}\selectfont\Large} % above (global formatting of entry)
    {\hspace*{2em}\contentslabel{2em}} % before w/label (label = ``2.6'')
    {\hspace*{2em}} % before w/o label
    %{\rmfamily\upshape\qquad\thecontentspage} % filler + page (leaders and page num)
    {\upshape\qquad\thecontentspage} % filler + page (leaders and page num)
    [\end{fullwidth}] % after
  \titlecontents{subsection}% FIXME
    [0em] % distance from left margin
    %{\vspace{0\baselineskip}\begin{fullwidth}\large\rmfamily\itshape} % above (global formatting of entry)
    {\vspace{0\baselineskip}\begin{fullwidth}\fontseries{m}\selectfont\large} % above (global formatting of entry)
    {\hspace*{4em}\contentslabel{4em}} % before w/label (label = ``2.6.1'')
    {\hspace*{4em}} % before w/o label
    %{\rmfamily\upshape\qquad\thecontentspage} % filler + page (leaders and page num)
    {\upshape\qquad\thecontentspage} % filler + page (leaders and page num)
    [\end{fullwidth}] % after
  \titlecontents{paragraph}% FIXME
    [0em] % distance from left margin
    %{\vspace{0\baselineskip}\begin{fullwidth}\normalsize\rmfamily\itshape} % above (global formatting of entry)
    {\vspace{0\baselineskip}\begin{fullwidth}\fontseries{m}\selectfont\normalsize\rmfamily} % above (global formatting of entry)
    {\hspace*{6em}\contentslabel{2em}} % before w/label (label = ``2.6.0.0.1'')
    {\hspace*{6em}} % before w/o label
    %{\rmfamily\upshape\qquad\thecontentspage} % filler + page (leaders and page num)
    {\upshape\qquad\thecontentspage} % filler + page (leaders and page num)
    [\end{fullwidth}] % after
}{}

  
\makeatother


\usepackage{booktabs}
\usepackage{longtable}
\usepackage{array}
\usepackage{multirow}
\usepackage{wrapfig}
\usepackage{float}
\usepackage{colortbl}
\usepackage{pdflscape}
\usepackage{tabu}
\usepackage{threeparttable}
\usepackage{threeparttablex}
\usepackage[normalem]{ulem}
\usepackage{makecell}
\usepackage{xcolor}

\begin{document}

\maketitle




\clearpage

\hypertarget{mating-designs-brown2014plantbreeding}{%
\section{\texorpdfstring{Mating designs
\citep{brown2014plantbreeding}}{Mating designs {[}@brown2014plantbreeding{]}}}\label{mating-designs-brown2014plantbreeding}}

The concept of mating designs is based on early work of Louis de
Vilmorin from the proposition he made that the only means to determine
the value of an individual plant (or genotype) was to grow and evaluate
its progeny -- A procedure known as Progeny testing.

Mating designs allow for partitioning of phenotypic effects -- as due to
genotype, environment or interacting effects among genes and alleles.
Using one or more of these mating schemes, identification of heterotic
groups, estimation of general and specific combining abilities and
testing of environmental interactions could be done. Progenies resulting
from a well designed mating are used for the dissection of trait
genetics.

In order to understand genetics of traits and to make effective choice
of parents, two contrasting methods of selection should be understood
first. Bluntly, selection methods can be stated as either being forward
or backward. Forward selection is synonymous to \textbf{within-family
selection} whereas the concept of backward selection embodies
\textbf{selection among families}. Ideally, forward selection works best
for highly heritable traits -- for those traits regulated by few small
genes, as opposed to those involving large number of genes with small
cumulative effects.

Selection gets more complicated as data on several different individuals
belonging to some families is available. {[}What we study versus what
reality is; Insert soil layer image here{]}. Broadly, three distinct
modes of selection could be practiced:

\begin{enumerate}
\def\labelenumi{\arabic{enumi}.}
\tightlist
\item
  Strict within family selection
\item
  Selection on within family deviation
\item
  Combined family and within family selection; family selection index
\end{enumerate}

When the interest is to exploit the state of heterosis arising from
certain combination of parental individuals, the genetic factors
contributing well to superior phenotype should be underpinned. The whole
process of determining favorable combination among parental individuals
should be met with phenotypic data from many progeny, which is
retrospective in purpose -- thus the name backward selection. Family
selection has different variants and serve variying purposes as well.

\begin{itemize}
\tightlist
\item
  Half-sib selection is used to select superier individuals for their
  GCA.
\item
  Full-sib selection is used to make distinct parental matings in order
  to induce hybrid vigor by capturing specific combinining abilities
\end{itemize}

The concept of combining abilities was first laid out by Sprague and
Tatum in 1942 \citep{sprague1942general} in order to generate variance
estimates without too much of underlying genetic assumptions. The
combining ability test procedure involves making crossess of several
different combinations from a set of parents and ascribing the resultant
variances statistically to either the genetic additiveness of parental
charactersistics or the interacting parental genetic combinations. Thus,
the phenotype (\(y_i\)) of a cross progeny can be modeled as linear
combination of additive (\(A_i\)), dominance (\(D_i\)) and environmental
(\(e_i\)) effects, is:

\[y_i = \mu + A_i + D_i + e_i\]

\hypertarget{combining-ability}{%
\subsection{Combining ability}\label{combining-ability}}

Crossing each line with several other lines produces an additional
measure in the mean performance of each line in all crosses. This mean
performance of a line, when expressed as a deviation from the mean of
all crosses, gives what is called the general combining ability
(\textbf{GCA}) of the lines. The GCA is calculated as the average of all
\(F_1\)s having this particular line as one parent, the value being
expressed as a deviation from the overall mean of crosses. Each cross
has an expected value (the sum of GCAs of its two parental lines).

The mean genotypic value of offspring from a particular cross may
deviate from value expected considering the population mean and the sum
of the parental GCA effects. This deviation is the specific combining
ability (\textbf{SCA}) for that cross.

The differences of GCA are due to the additive and additive x additive
interactions in the base population. The differences in SCA are
attributable to nonadditive genetic variance. SCA effects are derived
from inter-allelic/intra-loci interactions. Furthermore, the SCA is
expected to increase in variance more rapidly as inbreeding in the
population reaches high levels. GCA is the average performance of a
plant in a cross with different tester lines, while SCA measures the
performance of a plant in a specific combination in comparison with
other cross combinations.

We can define the mean genotypic value (\(G_{AB}\)) for the full-sib
family produced by crossing parents A and B as the sum of the overall
mean \(\mu\), the GCAs of the two parents and the SCA value:

\[G_{AB} = \mu + GCA_A + GCA_B + SCA_{AB}\]

The types of interactions that can be obtained (SCA effects) depend upon
the mating scheme used to produce the crosses, the most common being the
diallel mating design, developed by B. Griffing (1956). Methods such as
top cross and poly-cross are also not uncommon. A classical method to
estimate dominance genetic variance (D) is to estimate the variance
associated with SCA effects of many crosses. The expected value of the
observed SCA variance component is 1/4 of the dominance genetic variance
in the reference population.

The GCA of each line is calculated as follows:

\[
\mathrm{G_x} = \left[\frac{T_x}{n-2}\right]-\left[\frac{\sum T}{n(n-2)}\right]
\]

Where \(x\) represents a specific line. Using fabricated dataset given
in Table \ref{tab:fabricated-diallel} following procedures outlines how
GCA for Parent 2 (P2) (\(GCA_{P2}\)) can be calculated.

\begin{equation}
\begin{split}
\mathrm{G_b} & = \left[\frac{T_x}{n-2}\right]-\left[\frac{\sum T}{n(n-2)}\right] \\
 & = \left[\frac{39.7}{8}\right]-\left[\frac{324.7}{10\times 8}\right] \\
 & = 4.96-4.06 \\
 & = 0.9
\end{split}
\end{equation}

\hypertarget{diallel-scheme}{%
\subsection{Diallel scheme}\label{diallel-scheme}}

The diallel scheme is simply a more sophisticated application of
Vilmorin's progeny test. The term \emph{diallel cross} was first used by
Danish geneticist J. Schmidt in animal breeding work. The diallel cross
and its variations are as follows:

\begingroup\fontsize{10}{12}\selectfont

\begin{longtable}[t]{>{\raggedright\arraybackslash}p{3em}>{\raggedleft\arraybackslash}p{3em}>{\raggedleft\arraybackslash}p{3em}>{\raggedleft\arraybackslash}p{3em}>{\raggedleft\arraybackslash}p{3em}>{\raggedleft\arraybackslash}p{3em}>{\raggedleft\arraybackslash}p{3em}>{\raggedleft\arraybackslash}p{3em}>{\raggedleft\arraybackslash}p{3em}>{\raggedleft\arraybackslash}p{3em}>{\raggedleft\arraybackslash}p{3em}}
\caption{\label{tab:fabricated-diallel}Fabricated data from a diallel cross scheme using 10 parents}\\
\toprule
Parents & P1 & P2 & P3 & P4 & P5 & P6 & P7 & P8 & P9 & P10\\
\midrule
\rowcolor{gray!6}  P1 &  &  &  &  &  &  &  &  &  & \\
P2 & 2.64 &  &  &  &  &  &  &  &  & \\
\rowcolor{gray!6}  P3 & 3.30 & 4.9 &  &  &  &  &  &  &  & \\
P4 & -0.69 & 8.3 & 5.7 &  &  &  &  &  &  & \\
\rowcolor{gray!6}  P5 & 0.13 & 5.4 & 5.9 & 6.4 &  &  &  &  &  & \\
\addlinespace
P6 & -0.21 & 2.9 & 4.6 & 2.9 & 2.38 &  &  &  &  & \\
\rowcolor{gray!6}  P7 & 0.33 & 2.4 & 5.3 & 4.2 & 1.72 & 0.57 &  &  &  & \\
P8 & 3.70 & 2.1 & 5.0 & 4.2 & 0.13 & 2.99 & 5.3 &  &  & \\
\rowcolor{gray!6}  P9 & 3.94 & 6.0 & 5.4 & 7.6 & 3.39 & 0.35 & 6.2 & 4.2 &  & \\
P10 & 1.83 & 5.2 & 5.2 & 5.6 & 4.26 & 0.47 & 3.8 & 3.3 & 3.1 & \\
\bottomrule
\end{longtable}
\endgroup{}

Taking the above table of diallel cross data, total of each individual
parental line could be computed by summing over all the crossess
involving the common parent. Similarly, the grand totals could be
obtained by adding together all the individual parents' total. The
individual parents' sum and grand total is shown in the Table
\ref{tab:sum-over-ind} below.

\begin{table}

\caption{\label{tab:sum-over-ind}Totals of individual lines and grand total of diallel cross
 scheme using 10 parents}
\centering
\fontsize{10}{12}\selectfont
\begin{tabular}[t]{>{\bfseries}lr}
\toprule
Parents & Line total\\
\midrule
\rowcolor{gray!6}  P1 & 15\\
P2 & 40\\
\rowcolor{gray!6}  P3 & 45\\
P4 & 44\\
\rowcolor{gray!6}  P5 & 30\\
\addlinespace
P6 & 17\\
\rowcolor{gray!6}  P7 & 30\\
P8 & 31\\
\rowcolor{gray!6}  P9 & 40\\
P10 & 33\\
\addlinespace
\rowcolor{gray!6}  Total & 325\\
\bottomrule
\end{tabular}
\end{table}

\hypertarget{types-of-diallel-mating-design}{%
\subsection{Types of diallel mating
design}\label{types-of-diallel-mating-design}}

There are four types of popular diallel crossing designs:

\begin{enumerate}
\def\labelenumi{\arabic{enumi}.}
\tightlist
\item
  Full/Complete diallels: All the possible combinations of crosses among
  parents, including reciprocals and self-fertilization of the parents
  are made. For a sample of \(n\) parents, the full-diallel requires
  \(n \times n\) (\(n^2\)) progenies, a number that quickly becomes
  unmanageable as more parents are sampled (Table
  \ref{tab:full-diallel}).
\end{enumerate}

\begingroup\fontsize{8}{10}\selectfont

\begin{longtable}[t]{>{\raggedright\arraybackslash}p{3.5em}>{\raggedright\arraybackslash}p{3.5em}>{\raggedright\arraybackslash}p{3.5em}>{\raggedright\arraybackslash}p{3.5em}>{\raggedright\arraybackslash}p{3.5em}>{\raggedright\arraybackslash}p{3.5em}>{\raggedright\arraybackslash}p{3.5em}>{\raggedright\arraybackslash}p{3.5em}>{\raggedright\arraybackslash}p{3.5em}>{\raggedright\arraybackslash}p{3.5em}>{\raggedright\arraybackslash}p{3.5em}}
\caption{\label{tab:full-diallel}Full diallel mating scheme using 10 parents}\\
\toprule
Parents & P1 & P2 & P3 & P4 & P5 & P6 & P7 & P8 & P9 & P10\\
\midrule
\rowcolor{gray!6}  P1 & P1 x P1 & P1 x P2 & P1 x P3 & P1 x P4 & P1 x P5 & P1 x P6 & P1 x P7 & P1 x P8 & P1 x P9 & P1 x P10\\
P2 & P2 x P1 & P2 x P2 & P2 x P3 & P2 x P4 & P2 x P5 & P2 x P6 & P2 x P7 & P2 x P8 & P2 x P9 & P2 x P10\\
\rowcolor{gray!6}  P3 & P3 x P1 & P3 x P2 & P3 x P3 & P3 x P4 & P3 x P5 & P3 x P6 & P3 x P7 & P3 x P8 & P3 x P9 & P3 x P10\\
P4 & P4 x P1 & P4 x P2 & P4 x P3 & P4 x P4 & P4 x P5 & P4 x P6 & P4 x P7 & P4 x P8 & P4 x P9 & P4 x P10\\
\rowcolor{gray!6}  P5 & P5 x P1 & P5 x P2 & P5 x P3 & P5 x P4 & P5 x P5 & P5 x P6 & P5 x P7 & P5 x P8 & P5 x P9 & P5 x P10\\
\addlinespace
P6 & P6 x P1 & P6 x P2 & P6 x P3 & P6 x P4 & P6 x P5 & P6 x P6 & P6 x P7 & P6 x P8 & P6 x P9 & P6 x P10\\
\rowcolor{gray!6}  P7 & P7 x P1 & P7 x P2 & P7 x P3 & P7 x P4 & P7 x P5 & P7 x P6 & P7 x P7 & P7 x P8 & P7 x P9 & P7 x P10\\
P8 & P8 x P1 & P8 x P2 & P8 x P3 & P8 x P4 & P8 x P5 & P8 x P6 & P8 x P7 & P8 x P8 & P8 x P9 & P8 x P10\\
\rowcolor{gray!6}  P9 & P9 x P1 & P9 x P2 & P9 x P3 & P9 x P4 & P9 x P5 & P9 x P6 & P9 x P7 & P9 x P8 & P9 x P9 & P9 x P10\\
P10 & P10 x P1 & P10 x P2 & P10 x P3 & P10 x P4 & P10 x P5 & P10 x P6 & P10 x P7 & P10 x P8 & P10 x P9 & P10 x P10\\
\bottomrule
\end{longtable}
\endgroup{}

\begin{enumerate}
\def\labelenumi{\arabic{enumi}.}
\setcounter{enumi}{1}
\tightlist
\item
  Half diallels: Each parent is mated with every other parent, excluding
  selfs and reciprocals. This requires making \(\frac{n(n-1)}{2}\)
  crosses for n parents (Table \ref{tab:half-diallel}).
\end{enumerate}

\begingroup\fontsize{8}{10}\selectfont

\begin{longtable}[t]{>{\raggedright\arraybackslash}p{3.5em}>{\raggedright\arraybackslash}p{3.5em}>{\raggedright\arraybackslash}p{3.5em}>{\raggedright\arraybackslash}p{3.5em}>{\raggedright\arraybackslash}p{3.5em}>{\raggedright\arraybackslash}p{3.5em}>{\raggedright\arraybackslash}p{3.5em}>{\raggedright\arraybackslash}p{3.5em}>{\raggedright\arraybackslash}p{3.5em}>{\raggedright\arraybackslash}p{3.5em}>{\raggedright\arraybackslash}p{3.5em}}
\caption{\label{tab:half-diallel}Half diallel mating scheme using 10 parents}\\
\toprule
Parents & P1 & P2 & P3 & P4 & P5 & P6 & P7 & P8 & P9 & P10\\
\midrule
\endfirsthead
\caption[]{Half diallel mating scheme using 10 parents \textit{(continued)}}\\
\toprule
Parents & P1 & P2 & P3 & P4 & P5 & P6 & P7 & P8 & P9 & P10\\
\midrule
\endhead
\
\endfoot
\bottomrule
\endlastfoot
\rowcolor{gray!6}  P1 &  &  &  &  &  &  &  &  &  & \\
P2 & P2 x P1 &  &  &  &  &  &  &  &  & \\
\rowcolor{gray!6}  P3 & P3 x P1 & P3 x P2 &  &  &  &  &  &  &  & \\
P4 & P4 x P1 & P4 x P2 & P4 x P3 &  &  &  &  &  &  & \\
\rowcolor{gray!6}  P5 & P5 x P1 & P5 x P2 & P5 x P3 & P5 x P4 &  &  &  &  &  & \\
\addlinespace
P6 & P6 x P1 & P6 x P2 & P6 x P3 & P6 x P4 & P6 x P5 &  &  &  &  & \\
\rowcolor{gray!6}  P7 & P7 x P1 & P7 x P2 & P7 x P3 & P7 x P4 & P7 x P5 & P7 x P6 &  &  &  & \\
P8 & P8 x P1 & P8 x P2 & P8 x P3 & P8 x P4 & P8 x P5 & P8 x P6 & P8 x P7 &  &  & \\
\rowcolor{gray!6}  P9 & P9 x P1 & P9 x P2 & P9 x P3 & P9 x P4 & P9 x P5 & P9 x P6 & P9 x P7 & P9 x P8 &  & \\
P10 & P10 x P1 & P10 x P2 & P10 x P3 & P10 x P4 & P10 x P5 & P10 x P6 & P10 x P7 & P10 x P8 & P10 x P9 & \\*
\end{longtable}
\endgroup{}

\clearpage

\begin{enumerate}
\def\labelenumi{\arabic{enumi}.}
\setcounter{enumi}{2}
\tightlist
\item
  Partial diallel: Not all the crosses are made. There are no
  reciprocals or selfs. The goal is to reduce the breeding workload for
  a given sample of parents by making less than \(\frac{n(n-1)}{2}\)
  crosses for n parents (Table \ref{tab:partial-diallel}).
\end{enumerate}

With the partial scheme of diallel cross, same number of parents could
be tested in a framework with fewer crosses. The number of crosses in a
partial diallel scheme is given by,

\[
x = \frac{n \times s}{2}
\]

By analogy, we could have said that large number of lines could be
tested with lesser crossings under diallel scheme. Rearranging the above
relationship, \(\large n = \frac{2x}{s}\).

Further, there is restriction as to how \(s\) should is selected, for
example, is \(s = n -1\) it gives half diallel cross. \(s\) is a whole
number greater than or equal to 2, and \(k\) is a whole number --
\(k = \frac{n + 1 - s}{2}\).

Each parent occurs in \(s\) crosses, hence the number of crosses sampled
is \(ns/2\). For this to be true, For k to be a whole number, we do not
want both \(n\) and \(s\) to be odd or both even.

Now let's return to our previous example, with number of parents (\(n\))
to be 10. With \(n = 10\) we definitely want each parent to occur in at
least 2 sets of crosses. But, note that number of parents too is
positive (10). Hence, we even values of \(s\) is not allowed. Thus value
of \(s\) could be anywhere in \(\{3, 5, 7, 9\}\). We could also verify
that even values of \(s\) does not satisfy the requirement for \(k\) to
be a wholenumber.

For \(s = 2\),
\(k = \frac{n + 1 - s}{2} = \frac{10 + 1 - 2}{2} = \frac{9}{2} \neq \textrm{whole number}\)

For \(s = 4\),
\(k = \frac{n + 1 - s}{2} = \frac{10 + 1 - 4}{2} = \frac{7}{2} \neq \textrm{whole number}\),
and so on.

Here (Table \ref{tab:partial-diallel-10p-table}) is for example diallel
cross involving 10 parental lines and 7 set of cross involving each
parent,

\begingroup\fontsize{8}{10}\selectfont

\begin{longtable}[t]{rllllllllll}
\caption{\label{tab:partial-diallel-10p-table}Partial diallel cross involving 10 parents combined in 7 set of crossess each.}\\
\toprule
p & 1 & 2 & 3 & 4 & 5 & 6 & 7 & 8 & 9 & 10\\
\midrule
\rowcolor{gray!6}  1 &  &  &  & 4 x 1 & 5 x 1 & 6 x 1 & 7 x 1 & 8 x 1 &  & \\
2 &  &  &  &  & 5 x 2 & 6 x 2 & 7 x 2 & 8 x 2 & 9 x 2 & \\
\rowcolor{gray!6}  3 &  &  &  &  &  & 6 x 3 & 7 x 3 & 8 x 3 & 9 x 3 & 10 x 3\\
4 &  &  &  &  &  &  & 7 x 4 & 8 x 4 & 9 x 4 & 10 x 4\\
\rowcolor{gray!6}  5 &  &  &  &  &  &  &  & 8 x 5 & 9 x 5 & 10 x 5\\
\addlinespace
6 &  &  &  &  &  &  &  &  & 9 x 6 & 10 x 6\\
\rowcolor{gray!6}  7 &  &  &  &  &  &  &  &  &  & 10 x 7\\
8 &  &  &  &  &  &  &  &  &  & \\
\rowcolor{gray!6}  9 &  &  &  &  &  &  &  &  &  & \\
10 &  &  &  &  &  &  &  &  &  & \\
\bottomrule
\end{longtable}
\endgroup{}

\begin{enumerate}
\def\labelenumi{\arabic{enumi}.}
\setcounter{enumi}{3}
\tightlist
\item
  Connected diallels: Two groups (for example, \(1-6\) and \(7-12\)) of
  individuals are used to form two diallels (generally partial or other
  diallel scheme in each group) but they are connected by crossing
  \(4 \times 9\), \(7 \times 1\), \(9 \times 3\) and \(10 \times 2\). In
  the example below, the second diallel also includes some selfs \((S)\)
  and reciprocals \((R)\) (Table \ref{tab:connected-diallel}).
\end{enumerate}

\begingroup\fontsize{8}{10}\selectfont

\begin{longtable}[t]{>{\raggedright\arraybackslash}p{3.5em}>{\raggedright\arraybackslash}p{3.5em}>{\raggedright\arraybackslash}p{3.5em}>{\raggedright\arraybackslash}p{3.5em}>{\raggedright\arraybackslash}p{3.5em}>{\raggedright\arraybackslash}p{3.5em}>{\raggedright\arraybackslash}p{3.5em}>{\raggedright\arraybackslash}p{3.5em}>{\raggedright\arraybackslash}p{3.5em}>{\raggedright\arraybackslash}p{3.5em}>{\raggedright\arraybackslash}p{3.5em}ll}
\caption{\label{tab:connected-diallel}Connected diallel mating scheme using 12 parents}\\
\toprule
Parents & P1 & P2 & P3 & P4 & P5 & P6 & P7 & P8 & P9 & P10 & P11 & P12\\
\midrule
\rowcolor{gray!6}  P1 &  &  &  &  &  &  &  &  &  &  &  & \\
P2 & P2 x P1 &  &  &  &  &  &  &  &  &  &  & \\
\rowcolor{gray!6}  P3 & P3 x P1 &  &  &  &  &  &  &  &  &  &  & \\
P4 & P4 x P1 & P4 x P2 & P4 x P3 &  &  &  &  &  & P4 x P9 &  &  & \\
\rowcolor{gray!6}  P5 &  & P5 x P2 & P5 x P3 & P5 x P4 &  &  &  &  &  &  &  & \\
\addlinespace
P6 & P6 x P1 & P6 x P2 & P6 x P3 & P6 x P4 & P6 x P5 &  &  &  &  &  &  & \\
\rowcolor{gray!6}  P7 & P7 x P1 &  &  &  &  &  &  &  & P7 x P9 &  & P7 x P11 & \\
P8 &  &  &  &  &  &  &  &  &  &  &  & \\
\rowcolor{gray!6}  P9 &  &  & P9 x P3 &  &  &  &  &  &  &  &  & \\
P10 &  & P10 x P2 &  &  &  &  &  & P10 x P8 &  &  &  & \\
\addlinespace
\rowcolor{gray!6}  P11 &  &  &  &  &  &  & P11 x P7 &  &  & P11 x P10 & P11 x P11 & \\
P12 &  &  &  &  &  &  &  &  &  &  &  & P12 x P12\\
\bottomrule
\end{longtable}
\endgroup{}

With the convenience of using individuals as both male and female
parents, diallel mating designs are popular for plant breeding studies.
Certain diallel designs allow for estimation of reciprocal cross
effects. Diallels cannot be used in dioecious species (female and male
flowers occur in different plants). However, factorial designs can be
used in dioecious species to estimate dominance genetic variance.

If there are no connections between groups of parents, the design is a
diallel in sets. Diallel mating designs provide good evaluation of
parents and full-sib families. They also provide estimates of both
additive and dominance genetic effects, and genetic gains due to
additive and dominance genetic effects if we assume the sample of
parents used is sufficient to represent the reference population
\citep{baker1978issues, holland2003estimating}. One disadvantage of
diallels is that the breeding and progeny evaluations can be costly due
to large number of crosses required. As evident, For a full diallel with
6 parents, 36 crosses are required; with 12 parents the number of
crosses required is 144. On the other hand there is the looming
question, if the sample of 6 or 12 individuals from a population
provides useful estimate of the reference population genetic variances
\citep{baker1978issues}. \citet{white2007forest} and
\citet{hallauer1988quantitative} have described several other forms of
diallel mating designs in detail.

\hypertarget{analysis-of-diallel-cross}{%
\subsection{Analysis of diallel cross}\label{analysis-of-diallel-cross}}

There are mainly two approaches for analysis and interpretation of data
derived from diallel cross. They are:

\begin{enumerate}
\def\labelenumi{\arabic{enumi}.}
\item
  Analysis of general and specific combining ability. These methods are
  often referred to as Griffing's analyses, after B. Griffing who
  published his now famous paper \emph{Concept of general and specific
  combining ability in relation to diallel crossing systems}.
\item
  Analysis of array variances and covariances, often referred to as
  Hayman and Jinks' paper of 1953, \emph{The analysis of diallel
  crosses}
\end{enumerate}

\hypertarget{griffing-analysis}{%
\subsection{Griffing analysis}\label{griffing-analysis}}

Griffing's approach provides easy interpretation of results compared to
other analyses available. Parents used in diallel crosses can be
homozygous or heterozygous; for simplicity, diallel types are described
here in terms of homozygous (inbred) parents. Griffing's diallel
comprise of full diallel, half diallel (all possible combinations
without reciprocals but contains parental selfs), modified diallel (all
possible without parental selfs).

Griffing's analysis requires no assumptions and has been shown by many
researchers to provide reliable information on the combining potential
of parents. Once identifid, the ``best'' parental lines (those with the
highest general combining ability) can be crossed to identify optimum
hybrid combinations or to produce segregating progeny from which
superior cultivars would occur at a high frequency.

In simplest terms, the cross between two parents (i.e.~parent \(i\) and
parent \(j\)) in Griffing's analysis would be expressed as:

\[
X_{ij} = \mu + g_i + g_j + s_{ij}
\]

Where \(\mu\) is the overall mean of all entries in the diallel design,
\(g_i\) is the general combining ability of the \(i^{th}\) parent,
\(g_j\) is the general combining ability of the \(j^{th}\) parent, and
\(s_{ij}\) is the specific combining ability between the \(i_{th}\)
parent and the \(j_{th}\) parent.

General combining ability (GCA) measures the average performance of
parental lines in cross combination. GCA is therefore related to (but
not directly equal to) the proportion of variation that is genetically
additive in nature.

Specific combining ability (SCA) is the remaining part of the observed
phenotype that is not explained by the general combining ability of both
parents that constituted the progeny. By definition, SCA is the portion
of genetic variability which is not additive.

Griffing's analysis of a diallel is by analysis of variance, where the
total variance of all entries is partitioned into; and error variances.
In case where reciprocals are included, then reciprocals (or maternal
effects) are also partitioned. Error variances are estimated by
replication of families. To avoid excessive repetition, only Method 1
(complete diallel) and Method 2 (half diallel), both including parents,
will be considered further.

Degrees of freedom (df), sum of squares (SS) and mean squares (MSq) from
the analysis of variance for Method 1 for the assumption of model 1
(fixed effects) are shown in Table \ref{tab:complete-diallel-fixed}

\begin{table}

\caption{\label{tab:complete-diallel-fixed}Degrees of freedom, sum of squares and mean squares from the analysis of variance of a full diallel including parent selfs (Method 1) assuming fixed effects. Also shown are the expectations for the mean squares}
\centering
\begin{tabular}[t]{lllll}
\toprule
Source & df & SS & Msq & EMS\\
\midrule
GCA & p-1 & $S_g$ & $M_g$ & $\sigma^2 + 2p(1/(1-p))\sum g^2_i$\\
SCA & $\frac{p(p-1)}{2}$ & $S_s$ & $M_s$ & $\sigma^2 + \frac{2}{p(p-1)}\sum_{ij}s_{ij}^2$\\
Reciprocal & $\frac{p(p-1)}{2}$ & $S_r$ & $M_r$ & $\sigma^2 + 2\frac{2}{p(p-1)}\sum_{i<j}r_{ij}^2$\\
Error & $(r-1)p^2$ & $S_e$ & $M_e$ & $\sigma^2$\\
\bottomrule
\end{tabular}
\end{table}

\begin{table}

\caption{\label{tab:complete-diallel-random}Degrees of freedom, sum of squares and mean squares from the analysis of variance of a full diallel including parent selfs (Method 1) assuming random effects. Also shown are the expectations for the mean squares.}
\centering
\begin{tabular}[t]{lllll}
\toprule
Source & df & SS & Msq & EMS\\
\midrule
GCA & p-1 & $S_g$ & $M_g$ & $\sigma^2 + 2p(1/(1-p))\sigma^2_s + 2p\sigma^2_g$\\
SCA & $\frac{p(p-1)}{2}$ & $S_s$ & $M_s$ & $\sigma^2 + \frac{2(p^2-p+1)}{p^2 \sigma^2_s}\sum_{ij}s_{ij}^2$\\
Reciprocal & $\frac{p(p-1)}{2}$ & $S_r$ & $M_r$ & $\sigma^2 + 2\sigma^2_r$\\
Error & $(r-1)p^2$ & $S_e$ & $M_e$ & $\sigma^2$\\
\bottomrule
\end{tabular}
\end{table}

\begin{table}

\caption{\label{tab:half-diallel-fixed}Degrees of freedom, sum of squares and mean squares from the analysis of variance of a half diallel including parent selfs (Method 2), assuming fixed effects. Also shown are the expectations for the mean squares.}
\centering
\begin{tabular}[t]{lllll}
\toprule
Source & df & SS & Msq & EMS\\
\midrule
GCA & p-1 & $S_g$ & $M_g$ & $\sigma^2 + (p+2)(\frac{1}{1-p})\sum g_i^2$\\
SCA & $\frac{p(p-1)}{2}$ & $S_s$ & $M_s$ & $\sigma^2 + \frac{2p}{p-1}\sum_j s_{ij}^2$\\
Error & $(r-1)p\frac{p+1}{2}$ & $S_e$ & $M_e$ & $\sigma^2$\\
\bottomrule
\end{tabular}
\end{table}

\begin{table}

\caption{\label{tab:half-diallel-random}Degrees of freedom, sum of squares and mean squares from the analysis of variance of a half diallel including parent selfs (Method 2) assuming random effects. Also shown are the expectations for the mean squares.}
\centering
\begin{tabular}[t]{lllll}
\toprule
Source & df & SS & Msq & EMS\\
\midrule
GCA & p-1 & $S_g$ & $M_g$ & $\sigma^2 + \sigma^2_s + (p+2)\sigma^2_g$\\
SCA & $\frac{p(p-1)}{2}$ & $S_s$ & $M_s$ & $\sigma^2 + \sigma_s^2$\\
Error & $(r-1)\left(p\frac{p+1}{2}\right)$ & $S_e$ & $M_e$ & $\sigma^2$\\
\bottomrule
\end{tabular}
\end{table}

that is, sum over rows; \(X_j\) is
\(\sum_ix_{ij} = x_{1j} + x_{2j} + x_{3j} + ...,\) that is, sum over
columns; and \(X_{...}\) is \(\sum_{ij}x_{ij}\), the sum of all
observations. Where r is the number of replicates; p is the numeber of
parents; \(S_g\) is
\(\frac{1}{p+2}(\sum_i(X_i + x_ii)^2-\frac{4}{pX_{...}^2})\); \(S_s\) is
\(\sum_{i<j}x_{ij}^2-\frac{1}{p+2}\sum_i(X_i + x_{ii})^2 + \frac{2}{(p+1)(P+2)X_{...}^2}\)
and \(X_{i...}\) is \(\sum_j x_{ij} = x_{i1} + x_{i2} + x_{i3} + ...,\)
that is, the sum over rows; \(X_{...}\) is \(\sum_{ij}x_{ij}\), the sum
of all observations.

When SCA is relatively small in comparision with GCA, it should be
possible to predict the performance of particular cross combinations
based only on the values obtained for GCA of parents.

A realtively large SCA/GCA ratio implies the presence of dominance
and/or epistatic gene effects. It should be noted that if dominance x
additive effects are present, the GCA component will also contain some
of these effects in addition to pure additive effects.

For inbred lines, the closer that the following equations are equal to
one (i.e.~as SCA becomes small or very small compared with GCA), then
greater predictability based on GCA will be possible. The ratio
equations for each model are:

\[
\begin{aligned}
Model~1 &: \frac{2g_i^2}{[2g_i^2 + s_{ij}^2]} \\
Model~2 &: \frac{2\sigma_g^2}{[2\sigma_g^2 + \sigma_s^2}
\end{aligned}
\]

Where \(g_i^2\), \(\sigma_g^2\) are the general combining ability mean
square and variance, respectively and \(s_{ij}\) and \(\sigma_s^2\) are
specific combining ability mean square and variance, respectively.

The choice of Griffing method will depend on the plant breeder or
researcher's preference and on the characters of the crop and trial
under investigation. If, for example, there is suspicion that the
particular inheritance has a maternal or cytoplasmic effect then Method
1 or Method 3 may be the desired choice. If, however, there is no
evidence of reciprocal differences, then Method 2 or Method 4 would be
chosen. When the variance components are of major importance, then it
has been suggested that Method 1 will result in more accurate and
consistent variance estimation compared to other methods, available.
Conversely, it has been reported that the inclusion of the parental
genotypes in the diallel design can cause an upward bias in the
estimation of GCA and SCA variances.

Normally the \(F_1\) generation is considered in Griffing's analysis.
However, as no genetic assumptions are involved then there are no
reasons why \(F_2\) or indeed other segregating generations cannot be
analysed. This is a tremendous advantage in crop species where seeds per
hybridization event are few. For example in garbanzo bean (chick pea) a
single emasculation and pollination will result in 1 or 2 seeds, so
multiple hybridizations are needed to obtain quantities of seed suitable
for a proper \(F_1\) diallel analyses. In this case it is easier to
increase limited quantities of \(F_1\) seed to obtain larger quantities
of \(F_2\) seed (which can simply be obtained by bagging flowers to
prevent cross-pollination) for diallel analysis.

Despite the attraction and simplicity of Griffing's analysis, several
researchers have criticized the diallel technique. In open-pollinated
species such as corn, where GCA is the only parameter of interest, then
it has been suggested that other designs such as topcross or polycross
would yield equally reliable results with less effort, and that these
alternative methods provide the opportunity to test many more parental
lines. Similarly it has been argued that in many instances North
Carolina I designs (where a set of p parents to be tested are each
intercrossed with a set number of other parents, and where each parent
under test is not necessarily crossed to the same tester) or North
Carolina II designs (where a set of p parents are crossed to a common
set of n different parents and where each parent under test is crossed
to the same set of non-test parental (or tester) lines) would offer a
better alternative to diallel designs and Griffing's analysis.

Many studies have shown that the GCA values of parents from diallel
analyses are similar to actual phenotypic performance of the parents. It
has, therefore, been argued that it is not necessary to progeny test
potential parents in a plant breeding programme, but simply to ``cross
the best with the best''. Many practical plant breeders often add to
this statement, however, ``cross the best with the best, and hope for
the best'', but perhaps that is what we would be doing anyhow.

\hypertarget{example-of-griffings-analysis-of-half-diallel}{%
\subsection{Example of Griffing's analysis of half
diallel}\label{example-of-griffings-analysis-of-half-diallel}}

Let us consider now an example of a half diallel. A half diallel
crossing design between ten varieties of spring wheat ( \emph{Triticum
aestivum}) was carried out at the Agriculture and Forestry University,
Rampur, Chitwan, in the spring of 2018. The parental lines were: Bijay,
Swargadwari, WK-1204, Aditya, Gautam, Tilottama, HD-2967, Mayil,
Borlaug-100 and BL-4341. Bijay and Gautam are both popular and highly
scaled varieties, while the others are still in their initial stages of
being commercially tested. Crossing resulted in
\(\frac{n(n-1)}{2} = 45\) different \(F_1\) families. Over the following
winter, each of the 45 \(F_1\) families was grown in a two-replicate
randomized complete block design which also included the 10 parent
selfs, making a design with 55 entries \(\frac{n(n+1)}{2}\) and two
replicated (i.e.~110 plots).

Throughout the growth of this experiment a number of different traits
were recorded on each of the 110 plots. For the simplicity of use case
demonstration, only plant height character at the end of flowering will
be considered.

The average plant height of each of the 45\(F_1\)s and the 10 parents
are shown in Table \ref{tab:half-diallel-pht}. The data used were the
average of two plant heights (cms) as two representative plants were
measured in each of the replicate plots.

\begin{longtable}[t]{lrrrrrrrrrr}
\caption{\label{tab:half-diallel-pht}Average plant height (cm) of each of the 45 F1s and the 10 parents in a half diallel with selfs (replication 1)}\\
\toprule
  & Bijay & Swargadwari & WK-1204 & Aditya & Gautam & Tilottama & HD-2967 & Mayil & Borlaug-100 & BL-4341\\
\midrule
Bijay & 94.0 & 101.0 & 98.0 & 95 & 65.0 & 58.0 & 72.0 & 90.0 & 63 & 83.0\\
Swargadwari & 101.0 & 102.0 & 83.0 & 64 & 82.0 & 99.0 & 85.0 & 69.0 & 58 & 91.0\\
WK-1204 & 98.0 & 83.0 & 67.0 & 76 & 66.0 & 69.0 & 75.0 & 70.0 & 88 & 77.0\\
Aditya & 95.0 & 64.0 & 76.0 & 79 & 86.0 & 79.0 & 63.0 & 74.0 & 61 & 73.0\\
Gautam & 65.0 & 82.0 & 66.0 & 86 & 86.0 & 92.0 & 80.0 & 73.0 & 50 & 107.0\\
\addlinespace
Tilottama & 58.0 & 99.0 & 69.0 & 79 & 92.0 & 67.0 & 65.0 & 68.0 & 58 & 78.0\\
HD-2967 & 72.0 & 85.0 & 75.0 & 63 & 80.0 & 65.0 & 86.0 & 67.0 & 63 & 67.0\\
Mayil & 90.0 & 69.0 & 70.0 & 74 & 73.0 & 68.0 & 67.0 & 52.0 & 58 & 71.0\\
Borlaug-100 & 63.0 & 58.0 & 88.0 & 61 & 50.0 & 58.0 & 63.0 & 58.0 & 52 & 60.0\\
BL-4341 & 83.0 & 91.0 & 77.0 & 73 & 107.0 & 78.0 & 67.0 & 71.0 & 60 & 75.0\\
\addlinespace
GCA & 6.9 & 8.4 & 1.9 & 0 & 3.7 & -1.7 & -2.7 & -5.8 & -14 & 3.2\\
\bottomrule
\end{longtable}

From the data, the total variance (sum of squares) is partitioned into
differences between the two replicate blocks (Reps), general combining
ability, specific combining ability and an error term (based on
interactions between replicates and other factors). Sum of squares (SS)
and mean squares (MS) obtained are shown in Table
\ref{tab:half-diallel-pht-anova}.

\begin{table}

\caption{\label{tab:half-diallel-pht-gca}GCA of 10 parents in a half diallel with selfs}
\centering
\fontsize{8}{10}\selectfont
\begin{tabular}[t]{lr}
\toprule
Genotype & GCA\\
\midrule
\rowcolor{gray!6}  Aditya & 0.0\\
Bijay & 6.9\\
\rowcolor{gray!6}  BL-4341 & 3.2\\
Borlaug-100 & -13.9\\
\rowcolor{gray!6}  Gautam & 3.7\\
\addlinespace
HD-2967 & -2.7\\
\rowcolor{gray!6}  Mayil & -5.8\\
Swargadwari & 8.4\\
\rowcolor{gray!6}  Tilottama & -1.7\\
WK-1204 & 1.9\\
\bottomrule
\end{tabular}
\end{table}

Table \ref{tab:half-diallel-pht2-gca} shows the GCA values for second
replication of the experiment for plant height data (actually
simulated).

\begin{table}

\caption{\label{tab:half-diallel-pht2-gca}GCA of 10 parents in a half diallel with selfs}
\centering
\fontsize{8}{10}\selectfont
\begin{tabular}[t]{lr}
\toprule
Genotype & GCA\\
\midrule
\rowcolor{gray!6}  Aditya & -4.8\\
Bijay & 2.8\\
\rowcolor{gray!6}  BL-4341 & -3.6\\
Borlaug-100 & -20.4\\
\rowcolor{gray!6}  Gautam & -2.0\\
\addlinespace
HD-2967 & -5.9\\
\rowcolor{gray!6}  Mayil & -10.8\\
Swargadwari & 4.8\\
\rowcolor{gray!6}  Tilottama & -7.4\\
WK-1204 & -1.9\\
\bottomrule
\end{tabular}
\end{table}

\begin{table}

\caption{\label{tab:half-diallel-pht-anova}Degrees of freedom, sum of squares and mean squares from the analysis of variance of plant height of a half diallel including parent selfs. In the analysis, the total variance is partitioned into differences between the two replicate blocks (replication), general combining ability (Genotype male), specific combining ability (Genotype male:Genotype female) and an error term (based based on the replicate differences)}
\centering
\resizebox{\linewidth}{!}{
\fontsize{10}{12}\selectfont
\begin{tabular}[t]{lrrrrr}
\toprule
term & df & sumsq & meansq & statistic & p.value\\
\midrule
\rowcolor{gray!6}  Genotype male & 9 & 5842 & 649 & 61 & 0\\
replication & 1 & 909 & 909 & 85 & 0\\
\rowcolor{gray!6}  Genotype male:Genotype female & 45 & 16340 & 363 & 34 & 0\\
Residuals & 54 & 575 & 11 &  & \\
\bottomrule
\end{tabular}}
\end{table}

The basis assumption of this experiment was that the ten parental lines
were specifically chosen, although they were taken to be representative
of a wide range of spring wheat cultivar types. We are therefore
analysing a fixed effect model and all the mean squares in the analysis
are tested for significance (using the ``F'' test) against the error
mean square (i.e.~13).

From the analysis, the overall replicate block effect (i.e.~difference
between replicate one and replicate two) was not significant. An F-value
is obtained for specific combining ability as \(\frac{312}{13} = 24\).
This ``F'' value is compared to F-values found in statistical tables at
differing probability levels with 45 and 54 degrees of freedom. When
this is done, it is found that our observed F-value is greater than the
value from the statistical tables at the 5\% level. Therefore in this
example, specific combining ability is significant at the 5\% level, and
hence there should be presence of both variances contributing to the
overall phenotypes of progeny, this on the other hand implies a lowered
opportunity to predict progeny performance based on parent GCA values.

Consider now the variance ratio for general combining ability. The
appropriate F-value is \(\frac{689}{13} = 53\). When this value is
compared with the appropriate F-values in statistical tables with 9 and
54 degrees of freedom, we find that it exceeds the appropriate
expectation based on 99.9\% confidence (i.e.~approximately 3.54), and so
we say that general combining ability is highly significant. This, in
combination with the significant specific combining ability, suggests a
genetic model with both additive and interactive terms. This is opposed
to the model which would be composed of highly additive effects had
variation for the specific combining ability between genotypes been
lower.

Now the expected mean square for specific combining ability of a half
diallel with fixed effects is:

\[
\sigma^2 + \frac{2p}{p-1}\sum_is_i^2
\]

Therefore,

\[
\begin{aligned}
312-13 &= \frac{2\times 10}{10-1}\sum_is_i^2 \\
299 &= 2.2\sum_is_i^2
\end{aligned}
\]

So,

\[
\begin{aligned}
\sum_is_i^2 &= \frac{299}{2.2} \\
& = 136
\end{aligned}
\] Similarly for general combining ability, the expected mean square is:

\[
\begin{aligned}
689-13 &= (1 + 2)\left(\frac{1}{1-10}\right)\sum g_i^2 \\
676 &= 1.33 \sum g_i^2 
\end{aligned}
\]

So

\[
\begin{aligned}
\sum g_i^2 &= \frac{676}{1.33} \\
&= 508
\end{aligned}
\]

Now, from the equation above we can compare GCA and SCA effects, so as
noted earlier we have:

\[
\begin{aligned}
\frac{2g_i^2}{[2g_i^2 + s_{ij}]} &= \frac{2\times 508}{[(2\times 508) + 136]} \\
&= 0.88
\end{aligned}
\]

As this value is very close to one, it indicates that, as expected,
\(s_{ij}^2\) is relatively small compared to \(g_i^2\). Therefore
additive genetic effects predominate. This means there is a good chance
that plant height at the \(F_1\) stage in this \emph{B. napus} breeding
program can be predicted with good accuracy, depending upon the general
combining ability of the chosen parental lines.

\bibliography{skeleton.bib,../bibliographies.bib}



\end{document}
