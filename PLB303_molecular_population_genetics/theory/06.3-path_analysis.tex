\PassOptionsToPackage{unicode=true}{hyperref} % options for packages loaded elsewhere
\PassOptionsToPackage{hyphens}{url}
\documentclass[11pt,dvipsnames,ignorenonframetext,aspectratio=169]{beamer}
\IfFileExists{pgfpages.sty}{\usepackage{pgfpages}}{}
\setbeamertemplate{caption}[numbered]
\setbeamertemplate{caption label separator}{: }
\setbeamercolor{caption name}{fg=normal text.fg}
\beamertemplatenavigationsymbolsempty
\usepackage{lmodern}
\usepackage{amssymb,amsmath}
\usepackage{ifxetex,ifluatex}
\usepackage{fixltx2e} % provides \textsubscript
\ifnum 0\ifxetex 1\fi\ifluatex 1\fi=0 % if pdftex
  \usepackage[T1]{fontenc}
  \usepackage[utf8]{inputenc}
\else % if luatex or xelatex
  \ifxetex
    \usepackage{mathspec}
  \else
    \usepackage{fontspec}
\fi
\defaultfontfeatures{Ligatures=TeX,Scale=MatchLowercase}







\fi

  \usetheme[]{monash}

  \usecolortheme{monashwhite}


% A default size of 24 is set in beamerthememonash.sty

% Title page
\setbeamertemplate{title page}
{\placefig{-0.01}{-0.01}{width=1.01\paperwidth,height=1.01\paperheight}{deepseashrimp.jpg}
% \begin{textblock}{7.5}(1,2.8)\usebeamerfont{title} % original
\begin{textblock}{15}(1,2.2)\usebeamerfont{title}
{\color{white}\raggedright\par\inserttitle}
\end{textblock}
\begin{textblock}{10}(1,6)
\small
{\color{white}\raggedright{\insertauthor}\mbox{}\\[0.1cm]
\insertdate}
\end{textblock}}


  \useinnertheme{rounded}

  \useoutertheme{smoothtree}

% use upquote if available, for straight quotes in verbatim environments
\IfFileExists{upquote.sty}{\usepackage{upquote}}{}
% use microtype if available
\IfFileExists{microtype.sty}{%
  \usepackage{microtype}
  \UseMicrotypeSet[protrusion]{basicmath} % disable protrusion for tt fonts
}{}


\newif\ifbibliography


\hypersetup{
      pdftitle={Path Analysis},
            colorlinks=true,
    linkcolor=red,
    citecolor=Blue,
    urlcolor=lightgrayd,
    breaklinks=true}
%\urlstyle{same}  % Use monospace font for urls







% Prevent slide breaks in the middle of a paragraph:
\widowpenalties 1 10000
\raggedbottom

  \AtBeginPart{
    \let\insertpartnumber\relax
    \let\partname\relax
    \frame{\partpage}
  }
  \AtBeginSection{
    \ifbibliography
    \else
      \let\insertsectionnumber\relax
      \let\sectionname\relax
      \frame{\sectionpage}
    \fi
  }
  \AtBeginSubsection{
    \let\insertsubsectionnumber\relax
    \let\subsectionname\relax
    \frame{\subsectionpage}
  }



\setlength{\parindent}{0pt}
\setlength{\parskip}{6pt plus 2pt minus 1pt}
\setlength{\emergencystretch}{3em}  % prevent overfull lines
\providecommand{\tightlist}{%
  \setlength{\itemsep}{0pt}\setlength{\parskip}{0pt}}

  \setcounter{secnumdepth}{0}


%% Monash overrides
\AtBeginSection[]{
   \frame<beamer>{
   \frametitle{Outline}\vspace*{0.2cm}
   
   \tableofcontents[currentsection,hideallsubsections]
  }}

% Redefine shaded environment if it exists (to ensure text is black)
\ifcsname Shaded\endcsname
  \definecolor{shadecolor}{RGB}{225,225,225}
  \renewenvironment{Shaded}{\color{black}\begin{snugshade}\color{black}}{\end{snugshade}}
\fi
%%

  \usepackage{setspace}
  \usepackage{wasysym}
  % \usepackage{footnote} % don't use this this breaks all
  \usepackage{fontenc}
  \usepackage{fontawesome}
  \usepackage{booktabs,siunitx}
  \usepackage{longtable}
  \usepackage{array}
  \usepackage{multirow}
  \usepackage{wrapfig}
  \usepackage{float}
  \usepackage{colortbl}
  \usepackage{pdflscape}
  \usepackage{tabu}
  \usepackage{threeparttable}
  \usepackage{threeparttablex}
  \usepackage[normalem]{ulem}
  \usepackage{makecell}
  \usepackage{xcolor}
  \usepackage{tikz} % required for image opacity change
  \usepackage[absolute,overlay]{textpos} % for text formatting
  \usepackage{chemfig}
  \usepackage[skip=0.333\baselineskip]{caption}
  % \newcommand*{\AlignChar}[1]{\makebox[1ex][c]{\ensuremath{\scriptstyle#1}}}%

  % this font option is amenable for beamer
  \setbeamerfont{caption}{size=\tiny}
  \singlespacing
  \definecolor{lightgrayd}{gray}{0.95}
  \definecolor{skyblued}{rgb}{0.65, 0.6, 0.94}
  \definecolor{oranged}{RGB}{245, 145, 200}

  \newlength{\cslhangindent}
  \setlength{\cslhangindent}{1.5em}
  \newenvironment{cslreferences}%
    {\setlength{\parindent}{0pt}%
    \everypar{\setlength{\hangindent}{\cslhangindent}}\ignorespaces}%
    {\par}

  \title[]{Path Analysis}


  \author[
        Deependra Dhakal\\
College of Natural Resource Management\\
Agriculture and Forestry University\\
\textit{ddhakal.rookie@gmail.com}\\
\url{https://rookie.rbind.io}
    ]{Deependra Dhakal\\
College of Natural Resource Management\\
Agriculture and Forestry University\\
\textit{ddhakal.rookie@gmail.com}\\
\url{https://rookie.rbind.io}}


\date[
      
  ]{
    }

\begin{document}

% Hide progress bar and footline on titlepage
  \begin{frame}[plain]
  \titlepage
  \end{frame}


   \frame<beamer>{
   \frametitle{Outline}\vspace*{0.2cm}
   
   \tableofcontents[hideallsubsections]
  }

\hypertarget{path-analysis}{%
\section{Path analysis}\label{path-analysis}}

\begin{frame}{Introduction}
\protect\hypertarget{introduction}{}
\begin{itemize}
\tightlist
\item
  Path analysis, alike the family of related regression models, is a
  linear model framework that models regression equations simultaneously
  with the given observed variables. It can be thought of as a special
  case of Structural Equation Modeling (SEM).
\item
  Variations of SEMs:

  \begin{itemize}
  \tightlist
  \item
    Simple regression
  \item
    Multiple regression
  \item
    Path analysis
  \item
    Confirmatory factor analysis
  \item
    Measurement model (Latent - Observed)
  \item
    Structural model (Latent - Latent)
  \end{itemize}
\item
  Fit checks for SEMs

  \begin{itemize}
  \tightlist
  \item
    Model \(\chi^2\)
  \item
    RMSEA values
  \item
    Baseline model comparison
  \end{itemize}
\end{itemize}
\end{frame}

\begin{frame}{Example (fabricated) data}
\protect\hypertarget{example-fabricated-data}{}
\begin{table}

\caption{\label{tab:soydata-preview}A (fabricated) data showing variables having proposed roles in yield pathway.}
\centering
\fontsize{8}{10}\selectfont
\begin{tabular}[t]{lllllll}
\toprule
plants & pods\_plants & seeds\_pods & seed\_wt & pods\_branch & branch\_plant & grain\_yield\\
\midrule
14 & 102 & 2 & 75 & 9 & 11 & 1.1\\
14 & 92 & 2 & 62 & 10 & 10 & 1.01\\
18 & 83 & 3 & 86 & 6 & 14 & 1.44\\
... & ... & ... & ... & ... & ... & ...\\
15 & 72 & 2 & 76 & 6 & 9 & 1.17\\
\addlinespace
19 & 75 & 2 & 74 & 8 & 9 & 1.21\\
17 & 70 & 3 & 75 & 7 & 10 & 1.25\\
20 & 62 & 3 & 82 & 5 & 10 & 1.38\\
\bottomrule
\end{tabular}
\end{table}
\end{frame}

\begin{frame}{Conventions and symbology}
\protect\hypertarget{conventions-and-symbology}{}
\begin{itemize}
\item
  Circles are latent (unobserved) variables
\item
  Squares are manifest (observed) variables
\item
  Triangles can be used to interpret intercepts (part of square or
  circle but need to be turned on `specifically')
\item
  Except on specific model types (multigroup and latent growth), these
  are not estimated.
\end{itemize}
\end{frame}

\begin{frame}{}
\protect\hypertarget{section}{}
\begin{itemize}
\item
  Straight arrows are ``causal'' or directional
\item
  Non-standardized solution -\textgreater{} these are the \(b\) or slope
  values
\item
  Standardized solution -\textgreater{} these are beta values
  (standardized z score)
\item
  Curved arrows are non-directional
\item
  Non-standardized -\textgreater{} covariance
\item
  Standardized -\textgreater{} correlation
\item
  All endogenous variables (have arrows coming into them) have to have
  error terms.
\item
  Why is the arrow going into the variable ?

  \begin{itemize}
  \tightlist
  \item
    Because the error in the model is not explained by any other
    variable (otherwise it wouldn't be error).
  \end{itemize}
\end{itemize}
\end{frame}

\begin{frame}{}
\protect\hypertarget{section-1}{}
\includegraphics[width=0.85\linewidth]{06.3-path_analysis_files/figure-beamer/soybean-path-model-1}
\end{frame}

\hypertarget{bibliography}{%
\section{Bibliography}\label{bibliography}}

\begin{frame}{References}
\protect\hypertarget{references}{}
\end{frame}




\end{document}
