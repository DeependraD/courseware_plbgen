\documentclass[varwidth]{standalone}
\usepackage{tikz}
\usetikzlibrary{shapes,positioning,arrows.meta,decorations.pathmorphing,fit}
\usepackage{xcolor}

\tikzset{blue grid/.style={very thin,color=blue!40}}

\begin{document}

% % Recessive epistasis (Gene A is epistatic to Gene B (hypostatic) when in homozygous recessive (aa) state)

% 1st Phenotype

\begin{tikzpicture}[
  node distance=1.5cm,
  pre block/.style={draw,minimum height=0.8cm,minimum width=2cm},
  pro block/.style={draw,thick,minimum height=1cm,minimum width=2cm,decorate,decoration={random steps,segment length=5pt,amplitude=3pt}},
  gene2enzyme edge/.style={->,decorate,
  decoration={snake,amplitude=0.5mm,segment length=1mm,post length=1mm}},
  tcross/.append style={draw=#1, cross out},
  % gene block/.style={draw,circle,radius=1cm,inner sep=1pt},
  cross/.style={path picture={
  \draw[black]
  (path picture bounding box.south east) --
  (path picture bounding box.north west)
  (path picture bounding box.south west) --
  (path picture bounding box.north east);}}
  ]
\draw[blue grid] (-6,-2) grid (6,4);

% precursor, intermediate and product
\node[pre block] (precursor) at (-5,0) {Precursor};
\node[pre block,fill=red!40] (intermediate) [right=of precursor] {Intermediate compound};
\node[pro block,fill=yellow!50] (product) [right=of intermediate] {Product};

% edges joining pre and pro.
\draw[-stealth,thick] (precursor) -- node[above] (enza) {\textit{enz} a} (intermediate);
\draw[-stealth,thick] (intermediate) -- node[above] (enzb) {\textit{enz} b} (product);

\node[%gene block,
  above=2cm of enza] (geneA) {Gene A};
\node[%gene block,
  above=2cm of enzb] (geneB) {Gene B};

\node[
  above=of enza,
  % draw,
  rectangle,
  minimum height=0.5cm,
  minimum width=1.75cm,inner sep=1pt,text=black,path picture={
  \node[centered] at (path picture bounding box.center) {
  \includegraphics[width=2cm]{double\_hellix.pdf}
  };
}]
(geneA) {};

\node[
  above=of enzb,
  % draw,
  rectangle,
  minimum height=0.5cm,
  minimum width=1cm,inner sep=1pt,text=black,path picture={
  \node[centered] at (path picture bounding box.center) {
  \includegraphics[width=2cm]{double\_hellix.pdf}
  };
}] 
(geneB) {};


\draw[gene2enzyme edge] (geneA) -- node[right] {AA or Aa} (enza);
\draw[gene2enzyme edge] (geneB) -- node[right] {BB or Bb} (enzb);

\end{tikzpicture}

% 2nd Phenotype

\begin{tikzpicture}[
  node distance=1.5cm,
  pre block/.style={draw,minimum height=0.8cm,minimum width=2cm},
  pro block/.style={draw,thick,minimum height=1cm,minimum width=2cm,decorate,decoration={random steps,segment length=5pt,amplitude=3pt}},
  gene2enzyme edge/.style={->,decorate,
  decoration={snake,amplitude=0.5mm,segment length=1mm,post length=1mm}},
  tcross/.append style={draw=#1, cross out},
  % gene block/.style={draw,circle,radius=1cm,inner sep=1pt},
  cross/.style={path picture={
  \draw[black]
  (path picture bounding box.south east) --
  (path picture bounding box.north west)
  (path picture bounding box.south west) --
  (path picture bounding box.north east);}}
  ]
\draw[blue grid] (-6,-2) grid (6,4);

% precursor, intermediate and product
\node[pre block] (precursor) at (-5,0) {Precursor};
\node[pre block,fill=red!40] (intermediate) [right=of precursor] {Intermediate compound};
\node[pro block,fill=green!50] (product) [right=of intermediate] {Product};

% edges joining pre and pro.
\draw[-stealth,thick] (precursor) -- node[above] (enza) {\textit{enz} a} (intermediate);
\draw[-stealth,thick] (intermediate) -- node[above] (enzb) {\textit{enz} b} (product);

\node[%gene block,
  above=2cm of enza] (geneA) {Gene A};
\node[%gene block,
  above=2cm of enzb] (geneB) {Gene B};

\node[
  above=of enza,
  % draw,
  rectangle,
  minimum height=0.5cm,
  minimum width=1.75cm,inner sep=1pt,text=black,path picture={
  \node[centered] at (path picture bounding box.center) {
  \includegraphics[width=2cm]{double\_hellix.pdf}
  };
}]
(geneA) {};

\node[
  above=of enzb,
  % draw,
  rectangle,
  minimum height=0.5cm,
  minimum width=1cm,inner sep=1pt,text=black,path picture={
  \node[centered] at (path picture bounding box.center) {
  \includegraphics[width=2cm]{double\_hellix.pdf}
  };
}] 
(geneB) {};


\draw[gene2enzyme edge] (geneA) -- node[right] {AA or Aa} (enza);
\draw[gene2enzyme edge] (geneB) -- node[right] {bb} (enzb);

\end{tikzpicture}

% 3rd Phenotype

\begin{tikzpicture}[
  node distance=1.5cm,
  pre block/.style={draw,minimum height=0.8cm,minimum width=2cm},
  pro block/.style={draw,thick,minimum height=1cm,minimum width=2cm,decorate,decoration={random steps,segment length=5pt,amplitude=3pt}},
  gene2enzyme edge/.style={->,decorate,
  decoration={snake,amplitude=0.5mm,segment length=1mm,post length=1mm}},
  tcross/.append style={draw=#1, cross out},
  % gene block/.style={draw,circle,radius=1cm,inner sep=1pt},
  cross/.style={path picture={
  \draw[black]
  (path picture bounding box.south east) --
  (path picture bounding box.north west)
  (path picture bounding box.south west) --
  (path picture bounding box.north east);}}
  ]
\draw[blue grid] (-6,-2) grid (6,4);

% precursor, intermediate and product
\node[pre block] (precursor) at (-5,0) {Precursor};
\node[pre block,cross] (intermediate) [right=of precursor] {Intermediate compound};
\node[pro block,cross] (product) [right=of intermediate] {Product};

% edges joining pre and pro.
\draw[-stealth,thick] (precursor) -- node[above,tcross=red] (enza) {\textit{enz} a} (intermediate);
\draw[-stealth,thick] (intermediate) -- node[above] (enzb) {\textit{enz} b} (product);

\node[%gene block,
  above=2cm of enza] (geneA) {Gene A};
\node[%gene block,
  above=2cm of enzb] (geneB) {Gene B};

\node[
  above=of enza,
  % draw,
  rectangle,
  minimum height=0.5cm,
  minimum width=1.75cm,inner sep=1pt,text=black,path picture={
  \node[centered] at (path picture bounding box.center) {
  \includegraphics[width=2cm]{double\_hellix.pdf}
  };
}]
(geneA) {};

\node[
  above=of enzb,
  % draw,
  rectangle,
  minimum height=0.5cm,
  minimum width=1cm,inner sep=1pt,text=black,path picture={
  \node[centered] at (path picture bounding box.center) {
  \includegraphics[width=2cm]{double\_hellix.pdf}
  };
}] 
(geneB) {};


\draw[gene2enzyme edge] (geneA) -- node[right] {aa} (enza);
\draw[gene2enzyme edge] (geneB) -- node[right] {BB or Bb} (enzb);

\end{tikzpicture}

% 4th Phenotype

\begin{tikzpicture}[
  node distance=1.5cm,
  pre block/.style={draw,minimum height=0.8cm,minimum width=2cm},
  pro block/.style={draw,thick,minimum height=1cm,minimum width=2cm,decorate,decoration={random steps,segment length=5pt,amplitude=3pt}},
  gene2enzyme edge/.style={->,decorate,
  decoration={snake,amplitude=0.5mm,segment length=1mm,post length=1mm}},
  tcross/.append style={draw=#1, cross out},
  % gene block/.style={draw,circle,radius=1cm,inner sep=1pt},
  cross/.style={path picture={
  \draw[black]
  (path picture bounding box.south east) --
  (path picture bounding box.north west)
  (path picture bounding box.south west) --
  (path picture bounding box.north east);}}
  ]
\draw[blue grid] (-6,-2) grid (6,4);

% precursor, intermediate and product
\node[pre block] (precursor) at (-5,0) {Precursor};
\node[pre block,cross] (intermediate) [right=of precursor] {Intermediate compound};
\node[pro block,cross] (product) [right=of intermediate] {Product};

% edges joining pre and pro.
\draw[-stealth,thick] (precursor) -- node[above,tcross=red] (enza) {\textit{enz} a} (intermediate);
\draw[-stealth,thick] (intermediate) -- node[above] (enzb) {\textit{enz} b} (product);

\node[%gene block,
  above=2cm of enza] (geneA) {Gene A};
\node[%gene block,
  above=2cm of enzb] (geneB) {Gene B};

\node[
  above=of enza,
  % draw,
  rectangle,
  minimum height=0.5cm,
  minimum width=1.75cm,inner sep=1pt,text=black,path picture={
  \node[centered] at (path picture bounding box.center) {
  \includegraphics[width=2cm]{double\_hellix.pdf}
  };
}]
(geneA) {};

\node[
  above=of enzb,
  % draw,
  rectangle,
  minimum height=0.5cm,
  minimum width=1cm,inner sep=1pt,text=black,path picture={
  \node[centered] at (path picture bounding box.center) {
  \includegraphics[width=2cm]{double\_hellix.pdf}
  };
}] 
(geneB) {};


\draw[gene2enzyme edge] (geneA) -- node[right] {aa} (enza);
\draw[gene2enzyme edge] (geneB) -- node[right] {bb} (enzb);

\end{tikzpicture}

\end{document}