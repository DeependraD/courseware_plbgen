\documentclass[10pt,]{article}
\usepackage{lmodern}
\usepackage{amssymb,amsmath}
\usepackage{ifxetex,ifluatex}
\usepackage{fixltx2e} % provides \textsubscript
\ifnum 0\ifxetex 1\fi\ifluatex 1\fi=0 % if pdftex
  \usepackage[T1]{fontenc}
  \usepackage[utf8]{inputenc}
\else % if luatex or xelatex
  \ifxetex
    \usepackage{mathspec}
  \else
    \usepackage{fontspec}
  \fi
  \defaultfontfeatures{Ligatures=TeX,Scale=MatchLowercase}
\fi
% use upquote if available, for straight quotes in verbatim environments
\IfFileExists{upquote.sty}{\usepackage{upquote}}{}
% use microtype if available
\IfFileExists{microtype.sty}{%
\usepackage{microtype}
\UseMicrotypeSet[protrusion]{basicmath} % disable protrusion for tt fonts
}{}
\usepackage[margin=0.8cm]{geometry}
\usepackage{hyperref}
\hypersetup{unicode=true,
            pdfborder={0 0 0},
            breaklinks=true}
\urlstyle{same}  % don't use monospace font for urls
\usepackage{longtable,booktabs}
\usepackage{graphicx,grffile}
\makeatletter
\def\maxwidth{\ifdim\Gin@nat@width>\linewidth\linewidth\else\Gin@nat@width\fi}
\def\maxheight{\ifdim\Gin@nat@height>\textheight\textheight\else\Gin@nat@height\fi}
\makeatother
% Scale images if necessary, so that they will not overflow the page
% margins by default, and it is still possible to overwrite the defaults
% using explicit options in \includegraphics[width, height, ...]{}
\setkeys{Gin}{width=\maxwidth,height=\maxheight,keepaspectratio}
\IfFileExists{parskip.sty}{%
\usepackage{parskip}
}{% else
\setlength{\parindent}{0pt}
\setlength{\parskip}{6pt plus 2pt minus 1pt}
}
\setlength{\emergencystretch}{3em}  % prevent overfull lines
\providecommand{\tightlist}{%
  \setlength{\itemsep}{0pt}\setlength{\parskip}{0pt}}
\setcounter{secnumdepth}{5}
% Redefines (sub)paragraphs to behave more like sections
\ifx\paragraph\undefined\else
\let\oldparagraph\paragraph
\renewcommand{\paragraph}[1]{\oldparagraph{#1}\mbox{}}
\fi
\ifx\subparagraph\undefined\else
\let\oldsubparagraph\subparagraph
\renewcommand{\subparagraph}[1]{\oldsubparagraph{#1}\mbox{}}
\fi

%%% Use protect on footnotes to avoid problems with footnotes in titles
\let\rmarkdownfootnote\footnote%
\def\footnote{\protect\rmarkdownfootnote}

%%% Change title format to be more compact
\usepackage{titling}

% Create subtitle command for use in maketitle
\providecommand{\subtitle}[1]{
  \posttitle{
    \begin{center}\large#1\end{center}
    }
}

\setlength{\droptitle}{-2em}

  \title{\vspace{0.25cm} \Large{\textbf{TRIBHUVAN UNIVERSITY}}\\
\vspace{0.20cm} \large{GOKULESHWOR AGRICULTURE AND ANIMAL SCIENCE COLLEGE}\\
\vspace{0.20cm} \large{B. Sc. Ag., Internal Assessment, 2076}}
    \pretitle{\vspace{\droptitle}\centering\huge}
  \posttitle{\par}
    \author{}
    \preauthor{}\postauthor{}
    \date{}
    \predate{}\postdate{}
  
\usepackage{booktabs,siunitx}
\usepackage{listings}
\usepackage{xhfill}
\usepackage{framed}
\usepackage{xcolor}
\usepackage{etoolbox,refcount}
\usepackage{multicol}
\usepackage{enumitem}

\newcounter{countitems}
\newcounter{nextitemizecount}
\newcommand{\setupcountitems}{%
  \stepcounter{nextitemizecount}%
  \setcounter{countitems}{0}%
  \preto\item{\stepcounter{countitems}}%
}
\makeatletter
\newcommand{\computecountitems}{%
  \edef\@currentlabel{\number\c@countitems}%
  \label{countitems@\number\numexpr\value{nextitemizecount}-1\relax}%
}
\newcommand{\nextitemizecount}{%
  \getrefnumber{countitems@\number\c@nextitemizecount}%
}
\newcommand{\previtemizecount}{%
  \getrefnumber{countitems@\number\numexpr\value{nextitemizecount}-1\relax}%
}
\makeatother    
\newenvironment{AutoMultiColItemize}{%
\ifnumcomp{\nextitemizecount}{>}{3}{\begin{multicols}{2}}{}%
\setupcountitems\begin{itemize}}%
{\end{itemize}%
\unskip\computecountitems\ifnumcomp{\previtemizecount}{>}{3}{\end{multicols}}{}}

\newcommand*\wildcard[2][3cm]{\vspace*{1cm}\parbox{#1}{\centering\hrulefill\par#2\par}} %

\sisetup{per-mode=symbol}

\begin{document}
\maketitle

\begingroup

\makebox[4.2cm][l]{\textbf{Subject: Genetics of Populations}}
\hspace{0.5\textwidth}
\makebox[1.8cm][r]{\textbf{FM: 10}}

\par
\endgroup

\begingroup

\makebox[4.2cm][l]{\textbf{Time: 60 Minutes}}
\hspace{0.5\textwidth}
\makebox[1.8cm][r]{\textbf{PM: 4}}

\par
\endgroup

\textbf{Level:} \(5^{th}\) semester

\textbf{Answer all questions.}

Essay type (2 x 2 = 4)

\begin{enumerate}
\def\labelenumi{\arabic{enumi}.}
\tightlist
\item
  Write the significance of GxE interaction? Explain types of GxE interaction with diagrams.
\item
  Explain the model that best explains the structure of chromosome.
\end{enumerate}

Short answer type (1.5 x 4 = 6)

\begin{enumerate}
\def\labelenumi{\arabic{enumi}.}
\tightlist
\item
  What are restriction enzymes? Enlist any 5 restriction enzymes with their cleavage sites. Why is Type-II restriction enzyme family is more useful than others?
\item
  Explain ``one gene one polypeptide hypothesis'' with suitable examples if necessary.
\item
  Explain Johannsens pureline theory. What is meant by `variegation' in biological tissues?
\item
  What is recombinant DNA technology? Provide some examples.
\end{enumerate}

\vspace{2cm}

\begingroup

\makebox[4.2cm][l]{\textbf{Subject: Genetics of Populations}}
\hspace{0.5\textwidth}
\makebox[1.8cm][r]{\textbf{FM: 10}}

\par
\endgroup

\begingroup

\makebox[4.2cm][l]{\textbf{Time: 60 Minutes}}
\hspace{0.5\textwidth}
\makebox[1.8cm][r]{\textbf{PM: 4}}

\par
\endgroup

\textbf{Level:} \(5^{th}\) semester

\textbf{Answer all questions.}

Essay type (2 x 2 = 4)

\begin{enumerate}
\def\labelenumi{\arabic{enumi}.}
\tightlist
\item
  Write the significance of GxE interaction? Explain types of GxE interaction with diagrams.
\item
  Explain the model that best explains the structure of chromosome.
\end{enumerate}

Short answer type (1.5 x 4 = 6)

\begin{enumerate}
\def\labelenumi{\arabic{enumi}.}
\tightlist
\item
  What are restriction enzymes? Enlist any 5 restriction enzymes with their cleavage sites. Why is Type-II restriction enzyme family is more useful than others?
\item
  Explain ``one gene one polypeptide hypothesis'' with suitable examples if necessary.
\item
  Explain Johannsens pureline theory. What is meant by `variegation' in biological tissues?
\item
  What is recombinant DNA technology? Provide some examples.
\end{enumerate}


\end{document}
