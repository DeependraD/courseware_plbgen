\documentclass[10pt,]{article}
\usepackage{lmodern}
\usepackage{amssymb,amsmath}
\usepackage{ifxetex,ifluatex}
\usepackage{fixltx2e} % provides \textsubscript
\ifnum 0\ifxetex 1\fi\ifluatex 1\fi=0 % if pdftex
  \usepackage[T1]{fontenc}
  \usepackage[utf8]{inputenc}
\else % if luatex or xelatex
  \ifxetex
    \usepackage{mathspec}
  \else
    \usepackage{fontspec}
  \fi
  \defaultfontfeatures{Ligatures=TeX,Scale=MatchLowercase}
\fi
% use upquote if available, for straight quotes in verbatim environments
\IfFileExists{upquote.sty}{\usepackage{upquote}}{}
% use microtype if available
\IfFileExists{microtype.sty}{%
\usepackage{microtype}
\UseMicrotypeSet[protrusion]{basicmath} % disable protrusion for tt fonts
}{}
\usepackage[margin=0.8cm]{geometry}
\usepackage{hyperref}
\hypersetup{unicode=true,
            pdfborder={0 0 0},
            breaklinks=true}
\urlstyle{same}  % don't use monospace font for urls
\usepackage{longtable,booktabs}
\usepackage{graphicx,grffile}
\makeatletter
\def\maxwidth{\ifdim\Gin@nat@width>\linewidth\linewidth\else\Gin@nat@width\fi}
\def\maxheight{\ifdim\Gin@nat@height>\textheight\textheight\else\Gin@nat@height\fi}
\makeatother
% Scale images if necessary, so that they will not overflow the page
% margins by default, and it is still possible to overwrite the defaults
% using explicit options in \includegraphics[width, height, ...]{}
\setkeys{Gin}{width=\maxwidth,height=\maxheight,keepaspectratio}
\IfFileExists{parskip.sty}{%
\usepackage{parskip}
}{% else
\setlength{\parindent}{0pt}
\setlength{\parskip}{6pt plus 2pt minus 1pt}
}
\setlength{\emergencystretch}{3em}  % prevent overfull lines
\providecommand{\tightlist}{%
  \setlength{\itemsep}{0pt}\setlength{\parskip}{0pt}}
\setcounter{secnumdepth}{5}
% Redefines (sub)paragraphs to behave more like sections
\ifx\paragraph\undefined\else
\let\oldparagraph\paragraph
\renewcommand{\paragraph}[1]{\oldparagraph{#1}\mbox{}}
\fi
\ifx\subparagraph\undefined\else
\let\oldsubparagraph\subparagraph
\renewcommand{\subparagraph}[1]{\oldsubparagraph{#1}\mbox{}}
\fi

%%% Use protect on footnotes to avoid problems with footnotes in titles
\let\rmarkdownfootnote\footnote%
\def\footnote{\protect\rmarkdownfootnote}

%%% Change title format to be more compact
\usepackage{titling}

% Create subtitle command for use in maketitle
\providecommand{\subtitle}[1]{
  \posttitle{
    \begin{center}\large#1\end{center}
    }
}

\setlength{\droptitle}{-2em}

  \title{\vspace{0.25cm} \Large{\textbf{TRIBHUVAN UNIVERSITY}}\\
\vspace{0.20cm} \large{GOKULESHWOR AGRICULTURE AND ANIMAL SCIENCE COLLEGE}\\
\vspace{0.20cm} \large{B. Sc. Ag., Internal Assessment, 2076}}
    \pretitle{\vspace{\droptitle}\centering\huge}
  \posttitle{\par}
    \author{}
    \preauthor{}\postauthor{}
    \date{}
    \predate{}\postdate{}
  
\usepackage{booktabs,siunitx}
\usepackage{listings}
\usepackage{xhfill}
\usepackage{framed}
\usepackage{xcolor}
\usepackage{etoolbox,refcount}
\usepackage{multicol}
\usepackage{enumitem}

\newcounter{countitems}
\newcounter{nextitemizecount}
\newcommand{\setupcountitems}{%
  \stepcounter{nextitemizecount}%
  \setcounter{countitems}{0}%
  \preto\item{\stepcounter{countitems}}%
}
\makeatletter
\newcommand{\computecountitems}{%
  \edef\@currentlabel{\number\c@countitems}%
  \label{countitems@\number\numexpr\value{nextitemizecount}-1\relax}%
}
\newcommand{\nextitemizecount}{%
  \getrefnumber{countitems@\number\c@nextitemizecount}%
}
\newcommand{\previtemizecount}{%
  \getrefnumber{countitems@\number\numexpr\value{nextitemizecount}-1\relax}%
}
\makeatother    
\newenvironment{AutoMultiColItemize}{%
\ifnumcomp{\nextitemizecount}{>}{3}{\begin{multicols}{2}}{}%
\setupcountitems\begin{itemize}}%
{\end{itemize}%
\unskip\computecountitems\ifnumcomp{\previtemizecount}{>}{3}{\end{multicols}}{}}

\newcommand*\wildcard[2][3cm]{\vspace*{1cm}\parbox{#1}{\centering\hrulefill\par#2\par}} %

\sisetup{per-mode=symbol}

\begin{document}
\maketitle

\begingroup

\makebox[4.2cm][l]{\textbf{Subject: Introductory genetics}}
\hspace{0.5\textwidth}
\makebox[1.8cm][r]{\textbf{FM: 10}}

\par
\endgroup

\begingroup

\makebox[4.2cm][l]{\textbf{Time: 60 Minutes}}
\hspace{0.5\textwidth}
\makebox[1.8cm][r]{\textbf{PM: 4}}

\par
\endgroup

\textbf{Level:} \(3^{th}\) semester

\textbf{Answer all questions.}

Essay type (4)

\begin{enumerate}
\def\labelenumi{\arabic{enumi}.}
\tightlist
\item
  In maize, F1 heterozygous plants were test crossed with colourless, shrunken, waxy plants and the following types of progeny were obtained. CfS: 50, cFs: 46, Cfs: 383, cfS: 380, Dfs: 72, cFS: 68, CFS: 6, cfs: 5. Symbols: Colored = C, Colorless = c, Full = F, Shrunken = f, Starchy = S, waxy = s. \newline a. Are these genes linked ? Give reason. \newline b. Write the genes in correct order on the chromosome. \newline c. What are double crossover, non-crossover and single crossover types ? \newline d. Write the genotypes involved in the parental and test crosses. \newline e. Draw a linkage map showing map distances. \newline f. Calculate coefficient of coincidence (CC) and inference.
\end{enumerate}

Short answer type (1.5 x 4 = 6)

\begin{enumerate}
\def\labelenumi{\arabic{enumi}.}
\tightlist
\item
  A heterozygous plant for two genes is self-fertilized and in F2, the following seeds are observed.
\end{enumerate}

\begin{itemize}
\item
  Round yellow: 315,
\item
  Round green: 108,
\item
  Wrinkled yellow: 101,
\item
  Wrinkled green: 32

  Perform chi-square test for goodness of fit and interpret the result. (Tabulated chi-square value at 3d.f. = 7.81)
\end{itemize}

\begin{enumerate}
\def\labelenumi{\arabic{enumi}.}
\setcounter{enumi}{1}
\tightlist
\item
  What do you mean by extra nuclear inheritance? Explain with the help of example. Write down the characteristics of cytoplasmic inheritance.
\item
  What is sporogenesis and oogenesis? Diagrammatically show, how gamete formation takes place in life cycle of maize.
\item
  Explain in brief about Lac operon system in E. coli.
\end{enumerate}


\end{document}
