\documentclass[12pt]{article}\usepackage[]{graphicx}\usepackage[]{color}
% maxwidth is the original width if it is less than linewidth
% otherwise use linewidth (to make sure the graphics do not exceed the margin)
\makeatletter
\def\maxwidth{ %
  \ifdim\Gin@nat@width>\linewidth
    \linewidth
  \else
    \Gin@nat@width
  \fi
}
\makeatother

\definecolor{fgcolor}{rgb}{0.345, 0.345, 0.345}
\newcommand{\hlnum}[1]{\textcolor[rgb]{0.686,0.059,0.569}{#1}}%
\newcommand{\hlstr}[1]{\textcolor[rgb]{0.192,0.494,0.8}{#1}}%
\newcommand{\hlcom}[1]{\textcolor[rgb]{0.678,0.584,0.686}{\textit{#1}}}%
\newcommand{\hlopt}[1]{\textcolor[rgb]{0,0,0}{#1}}%
\newcommand{\hlstd}[1]{\textcolor[rgb]{0.345,0.345,0.345}{#1}}%
\newcommand{\hlkwa}[1]{\textcolor[rgb]{0.161,0.373,0.58}{\textbf{#1}}}%
\newcommand{\hlkwb}[1]{\textcolor[rgb]{0.69,0.353,0.396}{#1}}%
\newcommand{\hlkwc}[1]{\textcolor[rgb]{0.333,0.667,0.333}{#1}}%
\newcommand{\hlkwd}[1]{\textcolor[rgb]{0.737,0.353,0.396}{\textbf{#1}}}%
\let\hlipl\hlkwb

\usepackage{framed}
\makeatletter
\newenvironment{kframe}{%
 \def\at@end@of@kframe{}%
 \ifinner\ifhmode%
  \def\at@end@of@kframe{\end{minipage}}%
  \begin{minipage}{\columnwidth}%
 \fi\fi%
 \def\FrameCommand##1{\hskip\@totalleftmargin \hskip-\fboxsep
 \colorbox{shadecolor}{##1}\hskip-\fboxsep
     % There is no \\@totalrightmargin, so:
     \hskip-\linewidth \hskip-\@totalleftmargin \hskip\columnwidth}%
 \MakeFramed {\advance\hsize-\width
   \@totalleftmargin\z@ \linewidth\hsize
   \@setminipage}}%
 {\par\unskip\endMakeFramed%
 \at@end@of@kframe}
\makeatother

\definecolor{shadecolor}{rgb}{.97, .97, .97}
\definecolor{messagecolor}{rgb}{0, 0, 0}
\definecolor{warningcolor}{rgb}{1, 0, 1}
\definecolor{errorcolor}{rgb}{1, 0, 0}
\newenvironment{knitrout}{}{} % an empty environment to be redefined in TeX

\usepackage{alltt}
\usepackage[portrait,left=1.5cm,right=1.5cm,top=1.5cm,bottom=1.5cm,footskip=0.2in]{geometry}
\usepackage{booktabs}
\usepackage{longtable}
\usepackage{array}
\usepackage{multirow}
\usepackage{wrapfig}
\usepackage{float}
\usepackage{colortbl}
\usepackage{pdflscape}
\usepackage{tabu}
\usepackage{threeparttable}
\usepackage{threeparttablex}
\usepackage[normalem]{ulem}
\usepackage{makecell}
\usepackage{xcolor}
\setlength{\parindent}{0cm}
\IfFileExists{upquote.sty}{\usepackage{upquote}}{}
\begin{document}







{\centering \Large{\textbf{Gokuleshwor Agriculture and Animal Science College, Baitadi}} \\[0.25cm]
            \textbf{Institute}: GAASC, IAAS \\[0.2cm]
            \textbf{Internal Assessment} \\[0.2cm]} 
\textbf{Credit hours}: 3 + 1 \\ 
\textbf{Level}: B. Sc. Ag (4th Semester) \\
\textbf{Roll no}: 1 \\[0.5cm] 
\textbf{Essay type} (1 x 10 = 10 marks) \\
Question 1. Suppose two inbred lines A and B are crossed to produce F1 hybrid. The F1 is selfed and F2 is produced. The genotype of inbred line A is AAbb and the genotype of inbred line B is aaBB. A and B are the dominant genes and contribute 12 cm and 10 cm towards the spike length of the hybrid respectively. In the absence of dominants, each recessive gene contributes 4 cm towards the spike length of the hybrid. The spike length of the best commercial variety if 25 cm. a. Find the spike length of the parents, F1 and F2 progeny b. Calculate average heterosis, heterobeltiosis, economic heterosis and inbreeding depression. c. Interpret the results.\\
\textbf{Short answer type} (10 x 3 = 30 marks) \\
Question 2. How do you improve vegetatively propagated crops ? Describe a method of your choice with example.\\
Question 3. Briefly explain application of allopolyploids with the help of suitable examples.\\
Question 4. Explain briefly the various mechanism which promote self and cross pollination in crop plants.\\
Question 5. In a random mating poulation, the mean plant height and variance are 120 cm and 121 square cm respectively. A plant breeder selected the top 5\% plants from the base population and found mean plant height 110 cm in the next generation. Find the gentic gain, selection differential and heritability of this trait.\\
Question 6. What is male sterility ? List the various types of male sterility found in plants.\\
Question 7. Half sib and full sib selection breeding methods are used in crop improvement of cross pollinated crops like maize. Which one is the most effective and why?\\
Question 8. Enlist different breeding methods used in Rice and Maize crops.\\
Question 9. Enlist various types of breeding methods used in self-pollinated crops. And, how a disease resistant dominant gene is transferred from uncultivated genotype to cultivated cultivar ? Outline the breeding procedure.\\
Question 10. Outline the production of double cross hybrids using cytoplasmic-genetic male sterility.\\
Question 11. How do you develop single and double cross hybrids using cytoplasmic male sterility ? Explain.\\
\clearpage 
{\centering \Large{\textbf{Gokuleshwor Agriculture and Animal Science College, Baitadi}} \\[0.25cm]
            \textbf{Institute}: GAASC, IAAS \\[0.2cm]
            \textbf{Internal Assessment} \\[0.2cm]} 
\textbf{Credit hours}: 3 + 1 \\ 
\textbf{Level}: B. Sc. Ag (4th Semester) \\
\textbf{Roll no}: 2 \\[0.5cm] 
\textbf{Essay type} (1 x 10 = 10 marks) \\
Question 1. How do you transfer disease resistance gene from uncultivated genotype to cultivated cultivar, which is susceptible? Explain in detail.\\
\textbf{Short answer type} (10 x 3 = 30 marks) \\
Question 2. Outline the production of double cross hybrids using cytoplasmic-genetic male sterility.\\
Question 3. How do you produce double cross hybrid using genetic male sterility ? Outline the procedure.\\
Question 4. Write about the objectives of plant breeding.\\
Question 5. For a quantitative trait in a RMP, the mean is 100 and the variance is 240. The regression of the offspring on mid parent value is 0.25. Truncation selection is practiced with a selection differential of 32. What is the expected mean in the next generation? Also, find the heritability of the trait.\\
Question 6. State Hardy-Weinberg law. Compute gene and genotypic frequencies from the following data and mention how many plants are disease resistant. Note that the susceptible gen, R is dominant over resistant gene, r.\\ 
\begin{table}[H]
\centering\begingroup\fontsize{8}{10}\selectfont

\begin{tabular}[t]{llll}
\toprule
Genotype & RR & Rr & rr\\
\midrule
Frequency & 32 & 48 & 20\\
\bottomrule
\end{tabular}
\endgroup{}
\end{table}
Question 7. Differentiate between (any three): a. Inbreeders and outbreeders b. Race and pathotype c. Hybrids and synthetics d. Breeder's and certified seeds\\
Question 8. Write various breeding methods used in self-pollinated crops. How do you transfer a disease resistance dominant gene, R, from non-cultivated genotype to a commercially cultivated variety which is disease susceptible ? Outline the breeding procedure.\\
Question 9. Briefly explain the self pollination enforcing mechanisms. What are the genetic consequences of self-pollination ?\\
Question 10. Differentiate qualitative and quantitative traits.\\
Question 11. Give the conclusive remarks of pureline theory given by Johannsen on the basis of his study in French bean.\\
\clearpage 
{\centering \Large{\textbf{Gokuleshwor Agriculture and Animal Science College, Baitadi}} \\[0.25cm]
            \textbf{Institute}: GAASC, IAAS \\[0.2cm]
            \textbf{Internal Assessment} \\[0.2cm]} 
\textbf{Credit hours}: 3 + 1 \\ 
\textbf{Level}: B. Sc. Ag (4th Semester) \\
\textbf{Roll no}: 3 \\[0.5cm] 
\textbf{Essay type} (1 x 10 = 10 marks) \\
Question 1. What is G x E interaction ? Explain the various methods of estimating G x E interaction.\\
\textbf{Short answer type} (10 x 3 = 30 marks) \\
Question 2. Explain gene for gene relationship between host and a pathogen governing susceptible or resistance reaction.\\
Question 3. Present current research activities carried out by Nepal Agriculture Research Council (NARC) for the improvement of wheat crop.\\
Question 4. Give the conclusive remarks of pureline theory given by Johannsen on the basis of his study in French bean.\\
Question 5. In a random mating poulation containing 100 individuals, 25 are recessive genotypes. Find gene and genotypic frequencies for the trait in the population.\\
Question 6. State Hardy-Weinberg law. Compute gene and genotypic frequencies from the following data and mention how many plants are disease resistant. Note that the susceptible gen, R is dominant over resistant gene, r.\\ 
\begin{table}[H]
\centering\begingroup\fontsize{8}{10}\selectfont

\begin{tabular}[t]{llll}
\toprule
Genotype & RR & Rr & rr\\
\midrule
Frequency & 32 & 48 & 20\\
\bottomrule
\end{tabular}
\endgroup{}
\end{table}
Question 7. How do you produce single, double, and three way cross hybrids? Explain with the help of suitable figures.\\
Question 8. Briefly explain the self pollination enforcing mechanisms. What are the genetic consequences of self-pollination ?\\
Question 9. Write various breeding methods used in self-pollinated crops. How do you transfer a disease resistance dominant gene, R, from non-cultivated genotype to a commercially cultivated variety which is disease susceptible ? Outline the breeding procedure.\\
Question 10. Enlist different breeding methods used in Rice and Maize crops.\\
Question 11. Briefly explain about research activities that are being carried out by Nepal Agriculture Research Council for the improvement of Wheat (Triticum aestivum) in the context of Nepal.\\
\clearpage 
{\centering \Large{\textbf{Gokuleshwor Agriculture and Animal Science College, Baitadi}} \\[0.25cm]
            \textbf{Institute}: GAASC, IAAS \\[0.2cm]
            \textbf{Internal Assessment} \\[0.2cm]} 
\textbf{Credit hours}: 3 + 1 \\ 
\textbf{Level}: B. Sc. Ag (4th Semester) \\
\textbf{Roll no}: 4 \\[0.5cm] 
\textbf{Essay type} (1 x 10 = 10 marks) \\
Question 1. Suppose two inbred lines A and B are crossed to produce F1 hybrid. The F1 is selfed and F2 is produced. The genotype of inbred line A is AAbb and the genotype of inbred line B is aaBB. A and B are the dominant genes and contribute 12 cm and 10 cm towards the spike length of the hybrid respectively. In the absence of dominants, each recessive gene contributes 4 cm towards the spike length of the hybrid. The spike length of the best commercial variety if 25 cm. a. Find the spike length of the parents, F1 and F2 progeny b. Calculate average heterosis, heterobeltiosis, economic heterosis and inbreeding depression. c. Interpret the results.\\
\textbf{Short answer type} (10 x 3 = 30 marks) \\
Question 2. Write short notes on (any three): a. Center of origin b. Self incompatibility c. Chemical hybridizing agents d. Plant breeders' rights\\
Question 3. What is inbreeding depression ? Describe the effects of inbreeding.\\
Question 4. Differentiate between gametophytic and sporophytic systems of self-incompatibility with the help of well labelled diagrams.\\
Question 5. Differentiate between gametophytic and sporophytic systems of self-incompatibility.\\
Question 6. In a random mating poulation containing 100 individuals, 25 are recessive genotypes. Find gene and genotypic frequencies for the trait in the population.\\
Question 7. Mention current research activities carried out by NARC in the improvement of wheat crop.\\
Question 8. What different types of progeny will occur in gametophytic and sporophytic system of self-incompatibility from following cross ? In which system, homozygous progeny can occur and why? The dominance relationship operated like, S1 $>$ S2 $>$ S3 $>$ S4.\\ 
\begin{table}[H]
\centering\begingroup\fontsize{8}{10}\selectfont

\begin{tabular}[t]{llllll}
\toprule
Male & Female & Gametophytic SI reaction & Gametophytic SI progeny & Sporophytic SI reaction & Sporophytic SI progeny\\
\midrule
S1S3 & S1S2 &  &  &  & \\
S1S2 & S1S3 &  &  &  & \\
S2S3 & S1S4 &  &  &  & \\
\bottomrule
\end{tabular}
\endgroup{}
\end{table}
Question 9. What do you mean by Plant Breeder Right (PBR) ? What are the main points to be considered in getting PBR ?\\
Question 10. What are the breeding objectives for rice ? Discuss.\\
Question 11. How transgressive segregants are produced ? Present with the help of suitable figure.\\
\clearpage 
{\centering \Large{\textbf{Gokuleshwor Agriculture and Animal Science College, Baitadi}} \\[0.25cm]
            \textbf{Institute}: GAASC, IAAS \\[0.2cm]
            \textbf{Internal Assessment} \\[0.2cm]} 
\textbf{Credit hours}: 3 + 1 \\ 
\textbf{Level}: B. Sc. Ag (4th Semester) \\
\textbf{Roll no}: 5 \\[0.5cm] 
\textbf{Essay type} (1 x 10 = 10 marks) \\
Question 1. How do you transfer disease resistance gene from uncultivated genotype to cultivated cultivar, which is susceptible? Explain in detail.\\
\textbf{Short answer type} (10 x 3 = 30 marks) \\
Question 2. In a random mating poulation containing 100 individuals, 25 are recessive genotypes. Find gene and genotypic frequencies for the trait in the population.\\
Question 3. What is the frequency of the heterozygote (Bb) in random mating population, if the frequency of recessive phenotype (bb) is 0.04 ?\\
Question 4. From the following data, calculate heterosis, average heterosis and economic heterosis for grain yield of a wheat hybrid.\\ 
\begin{table}[H]
\centering\begingroup\fontsize{8}{10}\selectfont

\begin{tabular}[t]{ll}
\toprule
Parents & Grain yield (t/ha)\\
\midrule
A & 6.6\\
B & 4.2\\
F1(AxB) & 8.5\\
Best commercial variety in area (Gautam) & 7\\
\bottomrule
\end{tabular}
\endgroup{}
\end{table}
Question 5. Briefly explain the self pollination enforcing mechanisms. What are the genetic consequences of self-pollination ?\\
Question 6. Differentiate between: a. Self incompatibility and male sterility b. Self pollinated crops and cross pollinated crops\\
Question 7. How do you produce single, double and three way cross hybrids using cytoplasmic male sterility ? Explain.\\
Question 8. Explain briefly the various mechanism which promote self and cross pollination in crop plants.\\
Question 9. How do you produce double cross hybrid using genetic male sterility ? Outline the procedure.\\
Question 10. State Hardy-Weinberg law. Compute gene and genotypic frequencies from the following data and mention how many plants are disease resistant. Note that the susceptible gen, R is dominant over resistant gene, r.\\ 
\begin{table}[H]
\centering\begingroup\fontsize{8}{10}\selectfont

\begin{tabular}[t]{llll}
\toprule
Genotype & RR & Rr & rr\\
\midrule
Frequency & 32 & 48 & 20\\
\bottomrule
\end{tabular}
\endgroup{}
\end{table}
Question 11. How do you release a superior variety of rice (Oryza sativa) if the existing variety yields 3.5 tons per hectare and is 75\% disease resistant.\\
\clearpage 
{\centering \Large{\textbf{Gokuleshwor Agriculture and Animal Science College, Baitadi}} \\[0.25cm]
            \textbf{Institute}: GAASC, IAAS \\[0.2cm]
            \textbf{Internal Assessment} \\[0.2cm]} 
\textbf{Credit hours}: 3 + 1 \\ 
\textbf{Level}: B. Sc. Ag (4th Semester) \\
\textbf{Roll no}: 6 \\[0.5cm] 
\textbf{Essay type} (1 x 10 = 10 marks) \\
Question 1. Define plant breeding. State different breeding methods in self and cross pollinated crops. Explain with diagram breeding methods that are practiced in self-pollinated crops.\\
\textbf{Short answer type} (10 x 3 = 30 marks) \\
Question 2. What is inbreeding depression ? Describe the effects of inbreeding.\\
Question 3. For a quantitative trait in a RMP, the mean is 100 and the variance is 240. The regression of the offspring on mid parent value is 0.25. Truncation selection is practiced with a selection differential of 32. What is the expected mean in the next generation? Also, find the heritability of the trait.\\
Question 4. List different breeding methods used in wheat and maize crops. Give your logics why generally a long time is required to release a variety in self pollinated crop as compared to cross pollinated crops?\\
Question 5. What are the breeding objectives for rice ? Discuss.\\
Question 6. Differentiate between qualitative and quantitative traits.\\
Question 7. Define heterosis. Write the causes of heterosis and theories governing heterosis.\\
Question 8. Differentiate between: a. Self incompatibility and male sterility b. Self pollinated crops and cross pollinated crops\\
Question 9. How do you produce double cross hybrid using genetic male sterility ? Outline the procedure.\\
Question 10. Give the conclusive remarks of pureline theory given by Johannsen on the basis of his study in French bean.\\
Question 11. Mention current research activities carried out by NARC in the improvement of wheat crop.\\
\clearpage 
{\centering \Large{\textbf{Gokuleshwor Agriculture and Animal Science College, Baitadi}} \\[0.25cm]
            \textbf{Institute}: GAASC, IAAS \\[0.2cm]
            \textbf{Internal Assessment} \\[0.2cm]} 
\textbf{Credit hours}: 3 + 1 \\ 
\textbf{Level}: B. Sc. Ag (4th Semester) \\
\textbf{Roll no}: 7 \\[0.5cm] 
\textbf{Essay type} (1 x 10 = 10 marks) \\
Question 1. Suppose two inbred lines A and B are crossed to produce F1 hybrid. The F1 is selfed and F2 is produced. The genotype of inbred line A is AAbb and the genotype of inbred line B is aaBB. A and B are the dominant genes and contribute 12 cm towards the spike length of the hybrid. In the absence of dominants, each recessive gene contributes 8 cm towards the spike length of the hybrid. The spike length of the best commercial variety is 22 cm. a. Find the spike length of the parents, F1 and F2 progeny b. Find all kinds of Heterosis and Inbreeding depression c. Interpret the results.\\
\textbf{Short answer type} (10 x 3 = 30 marks) \\
Question 2. Write about the status of rice breeding in Nepal.\\
Question 3. Write various breeding methods used in self-pollinated crops. How do you transfer a disease resistance dominant gene, R, from non-cultivated genotype to a commercially cultivated variety which is disease susceptible ? Outline the breeding procedure.\\
Question 4. Give the conclusive remarks of pureline theory given by Johannsen on the basis of his study in French bean.\\
Question 5. Explain how genetic variation can be originated in a population.\\
Question 6. Explain the role of environment on quantitative character.\\
Question 7. Explain pureline theory on the basis of Johansen's experiment. What might be the application of this theory in breeding program?\\
Question 8. Differentiate between gametophytic and sporophytic systems of self-incompatibility.\\
Question 9. Enlist different breeding methods used in Rice and Maize crops.\\
Question 10. How do you produce single, double, and three way cross hybrids? Explain with the help of suitable figures.\\
Question 11. Differentiate between full sib and half sib selection. Which selection scheme is the most effective in breeding maize ? Logically explain.\\
\clearpage 
{\centering \Large{\textbf{Gokuleshwor Agriculture and Animal Science College, Baitadi}} \\[0.25cm]
            \textbf{Institute}: GAASC, IAAS \\[0.2cm]
            \textbf{Internal Assessment} \\[0.2cm]} 
\textbf{Credit hours}: 3 + 1 \\ 
\textbf{Level}: B. Sc. Ag (4th Semester) \\
\textbf{Roll no}: 8 \\[0.5cm] 
\textbf{Essay type} (1 x 10 = 10 marks) \\
Question 1. Suppose two inbred lines A and B are crossed to produce F1 hybrid. The F1 is selfed and F2 is produced. The genotype of inbred line A is AAbb and the genotype of inbred line B is aaBB. A and B are the dominant genes and contribute 12 cm towards the spike length of the hybrid. In the absence of dominants, each recessive gene contributes 8 cm towards the spike length of the hybrid. The spike length of the best commercial variety is 22 cm. a. Find the spike length of the parents, F1 and F2 progeny b. Find all kinds of Heterosis and Inbreeding depression c. Interpret the results.\\
\textbf{Short answer type} (10 x 3 = 30 marks) \\
Question 2. How transgressive segregants are produced ? Present with the help of suitable figure.\\
Question 3. What is patent ? Explain its requirements.\\
Question 4. How do you produce hybrid seed using one self-incompatible (P1) and another self-compatible (P2) parents ?\\
Question 5. Briefly explain the different defence mechanisms of host against pathogen/parasite. Which of the horizontal or vertical resitance is desirable in a commercial cutivar. Why?\\
Question 6. What is intellectual property right? Write the requirements of patent.\\
Question 7. What different types of progeny will occur in gametophytic and sporophytic system of self-incompatibility from following cross ? In which system, homozygous progeny can occur and why? The dominance relationship operated like, S1 $>$ S2 $>$ S3 $>$ S4.\\ 
\begin{table}[H]
\centering\begingroup\fontsize{8}{10}\selectfont

\begin{tabular}[t]{llllll}
\toprule
Male & Female & Gametophytic SI reaction & Gametophytic SI progeny & Sporophytic SI reaction & Sporophytic SI progeny\\
\midrule
S1S3 & S1S2 &  &  &  & \\
S1S2 & S1S3 &  &  &  & \\
S2S3 & S1S4 &  &  &  & \\
\bottomrule
\end{tabular}
\endgroup{}
\end{table}
Question 8. What is inbreeding depression ? Describe the effects of inbreeding.\\
Question 9. Explain pureline theory on the basis of Johansen's experiment. What might be the application of this theory in breeding program?\\
Question 10. What is gene for gene hypothesis ? Write compatible (+) and incompatible (-) reaction with the help of following information. Which are the most resistant and most susceptible reaction types and why ?\\ 
\begin{table}[H]
\centering\begingroup\fontsize{8}{10}\selectfont

\begin{tabular}[t]{llllll}
\toprule
Host genotypes & Pathogen G1 & Pathogen G2 & Pathogen G3 & Pathogen G4 & Pathogen G5\\
\midrule
 & a1a1a2a2a3a3 & A1A1a2a2a3a3 & A1A1A2A2a3a3 & a1a1a2a2A3A3 & A1A1A2A2A3A3\\
R1R1R2R2R3R3 &  &  &  &  & \\
R1R1r2r2R3R3 &  &  &  &  & \\
r1r1R2R2r3r3 &  &  &  &  & \\
r1r1r2r2R3R3 &  &  &  &  & \\
\addlinespace
r1r1r2r2r3r3 &  &  &  &  & \\
\bottomrule
\end{tabular}
\endgroup{}
\end{table}
Question 11. Define heterosis. Write the causes of heterosis and theories governing heterosis.\\
\clearpage 
{\centering \Large{\textbf{Gokuleshwor Agriculture and Animal Science College, Baitadi}} \\[0.25cm]
            \textbf{Institute}: GAASC, IAAS \\[0.2cm]
            \textbf{Internal Assessment} \\[0.2cm]} 
\textbf{Credit hours}: 3 + 1 \\ 
\textbf{Level}: B. Sc. Ag (4th Semester) \\
\textbf{Roll no}: 9 \\[0.5cm] 
\textbf{Essay type} (1 x 10 = 10 marks) \\
Question 1. What is G x E interaction ? Explain the various methods of estimating G x E interaction.\\
\textbf{Short answer type} (10 x 3 = 30 marks) \\
Question 2. What is gene for gene hypothesis ? Write compatible (+) and incompatible (-) reaction with the help of following information. Which are the most resistant and most susceptible reaction types and why ?\\ 
\begin{table}[H]
\centering\begingroup\fontsize{8}{10}\selectfont

\begin{tabular}[t]{llllll}
\toprule
Host genotypes & Pathogen G1 & Pathogen G2 & Pathogen G3 & Pathogen G4 & Pathogen G5\\
\midrule
 & a1a1a2a2a3a3 & A1A1a2a2a3a3 & A1A1A2A2a3a3 & a1a1a2a2A3A3 & A1A1A2A2A3A3\\
R1R1R2R2R3R3 &  &  &  &  & \\
R1R1r2r2R3R3 &  &  &  &  & \\
r1r1R2R2r3r3 &  &  &  &  & \\
r1r1r2r2R3R3 &  &  &  &  & \\
\addlinespace
r1r1r2r2r3r3 &  &  &  &  & \\
\bottomrule
\end{tabular}
\endgroup{}
\end{table}
Question 3. State law of homologous series in variation by NI Vavilov. Discuss the relationship between, primary, secondary and tertiary gene pools with respect to their combining ability.\\
Question 4. Make a partial diallel crossing scheme involving 11 parents and a three way cross involving 3 parents. Parents are represented like P1, P2, P3, …, P12.\\
Question 5. Write various breeding methods used in self-pollinated crops. How do you transfer a disease resistance dominant gene, R, from non-cultivated genotype to a commercially cultivated variety which is disease susceptible ? Outline the breeding procedure.\\
Question 6. What is male sterility ? List the various types of male sterility found in plants.\\
Question 7. Give the conclusive remarks of pureline theory given by Johannsen on the basis of his study in French bean.\\
Question 8. A cultivated variety of wheat became susceptible to a fungal disease which drastically reduced the yield. However, a wild variety is resistant to this fungus. If the resistance is a dominant trait governed by "R" gene, is it possible to transfer this trait to the cultivated variety? Give procedure with your logics.\\
Question 9. Explain how genetic variation can be originated in a population.\\
Question 10. Define heterosis. Write the causes of heterosis and theories governing heterosis.\\
Question 11. Explain pureline theory given by Johannsen.\\
\clearpage 
{\centering \Large{\textbf{Gokuleshwor Agriculture and Animal Science College, Baitadi}} \\[0.25cm]
            \textbf{Institute}: GAASC, IAAS \\[0.2cm]
            \textbf{Internal Assessment} \\[0.2cm]} 
\textbf{Credit hours}: 3 + 1 \\ 
\textbf{Level}: B. Sc. Ag (4th Semester) \\
\textbf{Roll no}: 10 \\[0.5cm] 
\textbf{Essay type} (1 x 10 = 10 marks) \\
Question 1. Suppose two inbred lines A and B are crossed to produce F1 hybrid. The F1 is selfed and F2 is produced. The genotype of inbred line A is AAbb and the genotype of inbred line B is aaBB. A and B are the dominant genes and contribute 12 cm and 10 cm towards the spike length of the hybrid respectively. In the absence of dominants, each recessive gene contributes 4 cm towards the spike length of the hybrid. The spike length of the best commercial variety if 25 cm. a. Find the spike length of the parents, F1 and F2 progeny b. Calculate average heterosis, heterobeltiosis, economic heterosis and inbreeding depression. c. Interpret the results.\\
\textbf{Short answer type} (10 x 3 = 30 marks) \\
Question 2. Write main achievements of plant breeding in the context of Nepal.\\
Question 3. The mean days to maturity and variance are 120 and 144 respectively. A plant breeder selected the top 5\% plants from base population and found mean days to maturity 110 in the next generation. Find the genetic gain and heritability of this trait.\\
Question 4. Differentiate between (any three): a. Inbreeders and outbreeders b. Race and pathotype c. Hybrids and synthetics d. Breeder's and certified seeds\\
Question 5. How do you produce double cross hybrid using genetic male sterility ? Outline the procedure.\\
Question 6. Briefly explain the different defence mechanisms of host against pathogen/parasite. Which of the horizontal or vertical resitance is desirable in a commercial cutivar. Why?\\
Question 7. Enlist different breeding methods used in Rice and Maize crops.\\
Question 8. Explain briefly the various mechanism which promote self and cross pollination in crop plants.\\
Question 9. State Hardy-Weinberg law. Compute gene and genotypic frequencies from the following data and mention how many plants are disease resistant. Note that the susceptible gen, R is dominant over resistant gene, r.\\ 
\begin{table}[H]
\centering\begingroup\fontsize{8}{10}\selectfont

\begin{tabular}[t]{llll}
\toprule
Genotype & RR & Rr & rr\\
\midrule
Frequency & 32 & 48 & 20\\
\bottomrule
\end{tabular}
\endgroup{}
\end{table}
Question 10. How do you release a superior variety of rice (Oryza sativa) if the existing variety yields 3.5 tons per hectare and is 75\% disease resistant.\\
Question 11. From a random mating population with mean of 200 units, individuals with mean of 260 are selected to be the parents of the next generation. If the heritability of the trait is 0.6, what is your expected mean in the next generation ?\\
\clearpage 
{\centering \Large{\textbf{Gokuleshwor Agriculture and Animal Science College, Baitadi}} \\[0.25cm]
            \textbf{Institute}: GAASC, IAAS \\[0.2cm]
            \textbf{Internal Assessment} \\[0.2cm]} 
\textbf{Credit hours}: 3 + 1 \\ 
\textbf{Level}: B. Sc. Ag (4th Semester) \\
\textbf{Roll no}: 11 \\[0.5cm] 
\textbf{Essay type} (1 x 10 = 10 marks) \\
Question 1. There is a wheat variety which is very good in terms of yield and other performances. It lacks trait, for instance, say disease resistance. A local landrace with recessive resistance is found. How do you incorporate it in the above wheat variety which is otherwise good ?\\
\textbf{Short answer type} (10 x 3 = 30 marks) \\
Question 2. How do you produce double cross hybrid using genetic male sterility ? Outline the procedure.\\
Question 3. Give the conclusive remarks of pureline theory given by Johannsen on the basis of his study in French bean.\\
Question 4. From a random mating population with mean of 200 units, individuals with mean of 260 are selected to be the parents of the next generation. If the heritability of the trait is 0.6, what is your expected mean in the next generation ?\\
Question 5. From the following data, calculate heterosis, average heterosis and economic heterosis for grain yield of a wheat hybrid.\\ 
\begin{table}[H]
\centering\begingroup\fontsize{8}{10}\selectfont

\begin{tabular}[t]{ll}
\toprule
Parents & Grain yield (t/ha)\\
\midrule
A & 6.6\\
B & 4.2\\
F1(AxB) & 8.5\\
Best commercial variety in area (Gautam) & 7\\
\bottomrule
\end{tabular}
\endgroup{}
\end{table}
Question 6. In a random mating poulation containing 100 individuals, 25 are recessive genotypes. Find gene and genotypic frequencies for the trait in the population.\\
Question 7. Explain the pureline theory of Johansen\\
Question 8. Differentiate between gametophytic and sporophytic systems of self-incompatibility with the help of well labelled diagrams.\\
Question 9. Differentiate between gametophytic and sporophytic systems of self-incompatibility.\\
Question 10. Differentiate between full sib and half sib selection. Which selection scheme is the most effective in breeding maize ? Logically explain.\\
Question 11. Outline the production of double cross hybrids using cytoplasmic-genetic male sterility.\\
\clearpage 
{\centering \Large{\textbf{Gokuleshwor Agriculture and Animal Science College, Baitadi}} \\[0.25cm]
            \textbf{Institute}: GAASC, IAAS \\[0.2cm]
            \textbf{Internal Assessment} \\[0.2cm]} 
\textbf{Credit hours}: 3 + 1 \\ 
\textbf{Level}: B. Sc. Ag (4th Semester) \\
\textbf{Roll no}: 12 \\[0.5cm] 
\textbf{Essay type} (1 x 10 = 10 marks) \\
Question 1. How do you transfer disease resistance gene from uncultivated genotype to cultivated cultivar, which is susceptible? Explain in detail.\\
\textbf{Short answer type} (10 x 3 = 30 marks) \\
Question 2. How transgressive segregants are produced ? Present with the help of suitable figure.\\
Question 3. The mean days to maturity and variance are 120 and 144 respectively. A plant breeder selected the top 5\% plants from base population and found mean days to maturity 110 in the next generation. Find the genetic gain and heritability of this trait.\\
Question 4. What are defense mechanisms of host against natural enemy ? Explain.\\
Question 5. Enlist various types of breeding methods used in self-pollinated crops. And, how a disease resistant dominant gene is transferred from uncultivated genotype to cultivated cultivar ? Outline the breeding procedure.\\
Question 6. Write various breeding methods used in self-pollinated crops. How do you transfer a disease resistance dominant gene, R, from non-cultivated genotype to a commercially cultivated variety which is disease susceptible ? Outline the breeding procedure.\\
Question 7. Write short notes on (any three): a. Center of origin b. Self incompatibility c. Chemical hybridizing agents d. Plant breeders' rights\\
Question 8. How do you improve vegetatively propagated crops ? Describe a method of your choice with example.\\
Question 9. How do you develop single and double cross hybrids using cytoplasmic male sterility ? Explain.\\
Question 10. Write main achievements of plant breeding in the context of Nepal.\\
Question 11. Explain pureline theory on the basis of Johansen's experiment. What might be the application of this theory in breeding program?\\
\clearpage 
{\centering \Large{\textbf{Gokuleshwor Agriculture and Animal Science College, Baitadi}} \\[0.25cm]
            \textbf{Institute}: GAASC, IAAS \\[0.2cm]
            \textbf{Internal Assessment} \\[0.2cm]} 
\textbf{Credit hours}: 3 + 1 \\ 
\textbf{Level}: B. Sc. Ag (4th Semester) \\
\textbf{Roll no}: 13 \\[0.5cm] 
\textbf{Essay type} (1 x 10 = 10 marks) \\
Question 1. What do you mean by heterosis and inbreeding depression? A plant breeder crossed two genotypes of wheat AAbb and aaBB to get F1. The F1 was selfed to obtain F2. On the basis of heterosis governing theories (dominance and over dominance), find all kinds of heterosis and inbreeding depression. Each of the dominant homozygote, heterozygote and recessive homozygote contributes 4 ton, 6 ton and 2 ton per hectare in yield respectively. The commercial variety of wheat yields 5 ton/ha.\\
\textbf{Short answer type} (10 x 3 = 30 marks) \\
Question 2. Explain gene for gene relationship between host and a pathogen governing susceptible or resistance reaction.\\
Question 3. Mention current research activities carried out by NARC in the improvement of wheat crop.\\
Question 4. For a quantitative trait in a RMP, mean is 100 and variation is 240. The regression of the offspring on mid-parent value is 0.25. Truncation selection is practiced with selection differential of 32. What is the expected mean in the next generation ?\\
Question 5. How do you improve vegetatively propagated crops ? Describe a method of your choice with example.\\
Question 6. Differentiate between full sib and half sib selection. Which selection scheme is the most effective in breeding maize ? Logically explain.\\
Question 7. How are haploids/monoploids produced and utilized in plant breeding ?\\
Question 8. How do you produce single, double and three way cross hybrids using cytoplasmic male sterility ? Explain.\\
Question 9. Briefly explain about research activities that are being carried out by Nepal Agriculture Research Council for the improvement of Wheat (Triticum aestivum) in the context of Nepal.\\
Question 10. State Hardy-Weinberg law. In a population consisting of 10000 individuals, 49 individuals are of "aa" genotype. If the population is in Hardy-Weinberg equilibrium, find the gene and genotype frequencies of that population.\\
Question 11. List different breeding methods used in wheat and maize crops. Give your logics why generally a long time is required to release a variety in self pollinated crop as compared to cross pollinated crops?\\
\clearpage 
{\centering \Large{\textbf{Gokuleshwor Agriculture and Animal Science College, Baitadi}} \\[0.25cm]
            \textbf{Institute}: GAASC, IAAS \\[0.2cm]
            \textbf{Internal Assessment} \\[0.2cm]} 
\textbf{Credit hours}: 3 + 1 \\ 
\textbf{Level}: B. Sc. Ag (4th Semester) \\
\textbf{Roll no}: 14 \\[0.5cm] 
\textbf{Essay type} (1 x 10 = 10 marks) \\
Question 1. Suppose two inbred lines A and B are crossed to produce F1 hybrid. The F1 is selfed and F2 is produced. The genotype of inbred line A is AAbb and the genotype of inbred line B is aaBB. A and B are the dominant genes and contribute 12 cm and 10 cm towards the spike length of the hybrid respectively. In the absence of dominants, each recessive gene contributes 4 cm towards the spike length of the hybrid. The spike length of the best commercial variety if 25 cm. a. Find the spike length of the parents, F1 and F2 progeny b. Calculate average heterosis, heterobeltiosis, economic heterosis and inbreeding depression. c. Interpret the results.\\
\textbf{Short answer type} (10 x 3 = 30 marks) \\
Question 2. What is the frequency of the heterozygote (Bb) in random mating population, if the frequency of recessive phenotype (bb) is 0.04 ?\\
Question 3. What is gene for gene hypothesis ? Write compatible (+) and incompatible (-) reaction with the help of following information. Which are the most resistant and most susceptible reaction types and why ?\\ 
\begin{table}[H]
\centering\begingroup\fontsize{8}{10}\selectfont

\begin{tabular}[t]{llllll}
\toprule
Host genotypes & Pathogen G1 & Pathogen G2 & Pathogen G3 & Pathogen G4 & Pathogen G5\\
\midrule
 & a1a1a2a2a3a3 & A1A1a2a2a3a3 & A1A1A2A2a3a3 & a1a1a2a2A3A3 & A1A1A2A2A3A3\\
R1R1R2R2R3R3 &  &  &  &  & \\
R1R1r2r2R3R3 &  &  &  &  & \\
r1r1R2R2r3r3 &  &  &  &  & \\
r1r1r2r2R3R3 &  &  &  &  & \\
\addlinespace
r1r1r2r2r3r3 &  &  &  &  & \\
\bottomrule
\end{tabular}
\endgroup{}
\end{table}
Question 4. Differentiate between gametophytic and sporophytic systems of self-incompatibility.\\
Question 5. Briefly explain about research activities that are being carried out by Nepal Agriculture Research Council for the improvement of Wheat (Triticum aestivum) in the context of Nepal.\\
Question 6. In a random mating poulation containing 100 individuals, 25 are recessive genotypes. Find gene and genotypic frequencies for the trait in the population.\\
Question 7. For a quantitative trait in a RMP, mean is 100 and variation is 240. The regression of the offspring on mid-parent value is 0.25. Truncation selection is practiced with selection differential of 32. What is the expected mean in the next generation ?\\
Question 8. State Hardy-Weinberg law. In a population consisting of 10000 individuals, 49 individuals are of "aa" genotype. If the population is in Hardy-Weinberg equilibrium, find the gene and genotype frequencies of that population.\\
Question 9. State Hardy-Weinberg law. Compute gene and genotypic frequencies from the following data and mention how many plants are disease resistant. Note that the susceptible gen, R is dominant over resistant gene, r.\\ 
\begin{table}[H]
\centering\begingroup\fontsize{8}{10}\selectfont

\begin{tabular}[t]{llll}
\toprule
Genotype & RR & Rr & rr\\
\midrule
Frequency & 32 & 48 & 20\\
\bottomrule
\end{tabular}
\endgroup{}
\end{table}
Question 10. Explain the pureline theory of Johansen\\
Question 11. How do you produce single, double, and three way cross hybrids? Explain with the help of suitable figures.\\
\clearpage 
{\centering \Large{\textbf{Gokuleshwor Agriculture and Animal Science College, Baitadi}} \\[0.25cm]
            \textbf{Institute}: GAASC, IAAS \\[0.2cm]
            \textbf{Internal Assessment} \\[0.2cm]} 
\textbf{Credit hours}: 3 + 1 \\ 
\textbf{Level}: B. Sc. Ag (4th Semester) \\
\textbf{Roll no}: 15 \\[0.5cm] 
\textbf{Essay type} (1 x 10 = 10 marks) \\
Question 1. Suppose two inbred lines A and B are crossed to produce F1 hybrid. The F1 is selfed and F2 is produced. The genotype of inbred line A is AAbb and the genotype of inbred line B is aaBB. A and B are the dominant genes and contribute 12 cm towards the spike length of the hybrid. In the absence of dominants, each recessive gene contributes 8 cm towards the spike length of the hybrid. The spike length of the best commercial variety is 22 cm. a. Find the spike length of the parents, F1 and F2 progeny b. Find all kinds of Heterosis and Inbreeding depression c. Interpret the results.\\
\textbf{Short answer type} (10 x 3 = 30 marks) \\
Question 2. State law of homologous series in variation by NI Vavilov. Discuss the relationship between, primary, secondary and tertiary gene pools with respect to their combining ability.\\
Question 3. The mean days to maturity and variance are 120 and 144 respectively. A plant breeder selected the top 5\% plants from base population and found mean days to maturity 110 in the next generation. Find the genetic gain and heritability of this trait.\\
Question 4. How do you produce hybrid seed using one self-incompatible (P1) and another self-compatible (P2) parents ?\\
Question 5. What is gene for gene hypothesis ? Write compatible (+) and incompatible (-) reaction with the help of following information. Which are the most resistant and most susceptible reaction types and why ?\\ 
\begin{table}[H]
\centering\begingroup\fontsize{8}{10}\selectfont

\begin{tabular}[t]{llllll}
\toprule
Host genotypes & Pathogen G1 & Pathogen G2 & Pathogen G3 & Pathogen G4 & Pathogen G5\\
\midrule
 & a1a1a2a2a3a3 & A1A1a2a2a3a3 & A1A1A2A2a3a3 & a1a1a2a2A3A3 & A1A1A2A2A3A3\\
R1R1R2R2R3R3 &  &  &  &  & \\
R1R1r2r2R3R3 &  &  &  &  & \\
r1r1R2R2r3r3 &  &  &  &  & \\
r1r1r2r2R3R3 &  &  &  &  & \\
\addlinespace
r1r1r2r2r3r3 &  &  &  &  & \\
\bottomrule
\end{tabular}
\endgroup{}
\end{table}
Question 6. On the basis of following table, answer the following questions: a. Which cultivar is the most susceptible and why ? b. Which cultivar is the most resistant and why ? c. Which cultivar is the most tolerant and why ? d. Which cultivar is the most sensitive and why ?\\ 
\begin{table}[H]
\centering\begingroup\fontsize{8}{10}\selectfont

\begin{tabular}[t]{lllll}
\toprule
Cultivar & Virus concentration & Yellowing & Yield with virus & Yield without virus\\
\midrule
A & 100 & 8 & 80 & 90\\
B & 60 & 0 & 97 & 100\\
C & 50 & 0 & 90 & 70\\
\bottomrule
\end{tabular}
\endgroup{}
\end{table}
Question 7. Enlist various types of breeding methods used in self-pollinated crops. And, how a disease resistant dominant gene is transferred from uncultivated genotype to cultivated cultivar ? Outline the breeding procedure.\\
Question 8. Differentiate between qualitative and quantitative traits.\\
Question 9. What is inbreeding depression ? Describe the effects of inbreeding.\\
Question 10. A cultivated variety of wheat became susceptible to a fungal disease which drastically reduced the yield. However, a wild variety is resistant to this fungus. If the resistance is a dominant trait governed by "R" gene, is it possible to transfer this trait to the cultivated variety? Give procedure with your logics.\\
Question 11. Differentiate between (any three): a. Inbreeders and outbreeders b. Race and pathotype c. Hybrids and synthetics d. Breeder's and certified seeds\\
\clearpage 
{\centering \Large{\textbf{Gokuleshwor Agriculture and Animal Science College, Baitadi}} \\[0.25cm]
            \textbf{Institute}: GAASC, IAAS \\[0.2cm]
            \textbf{Internal Assessment} \\[0.2cm]} 
\textbf{Credit hours}: 3 + 1 \\ 
\textbf{Level}: B. Sc. Ag (4th Semester) \\
\textbf{Roll no}: 16 \\[0.5cm] 
\textbf{Essay type} (1 x 10 = 10 marks) \\
Question 1. What is G x E interaction ? Explain the various methods of estimating G x E interaction.\\
\textbf{Short answer type} (10 x 3 = 30 marks) \\
Question 2. How are haploids/monoploids produced and utilized in plant breeding ?\\
Question 3. Explain gene for gene relationship between host and a pathogen governing susceptible or resistance reaction.\\
Question 4. How do you release a superior variety of rice (Oryza sativa) if the existing variety yields 3.5 tons per hectare and is 75\% disease resistant.\\
Question 5. Define intellectual property right. Explain its forms.\\
Question 6. Write about the objectives of plant breeding.\\
Question 7. What is gene for gene hypothesis ? Write compatible (+) and incompatible (-) reaction with the help of following information. Which are the most resistant and most susceptible reaction types and why ?\\ 
\begin{table}[H]
\centering\begingroup\fontsize{8}{10}\selectfont

\begin{tabular}[t]{llllll}
\toprule
Host genotypes & Pathogen G1 & Pathogen G2 & Pathogen G3 & Pathogen G4 & Pathogen G5\\
\midrule
 & a1a1a2a2a3a3 & A1A1a2a2a3a3 & A1A1A2A2a3a3 & a1a1a2a2A3A3 & A1A1A2A2A3A3\\
R1R1R2R2R3R3 &  &  &  &  & \\
R1R1r2r2R3R3 &  &  &  &  & \\
r1r1R2R2r3r3 &  &  &  &  & \\
r1r1r2r2R3R3 &  &  &  &  & \\
\addlinespace
r1r1r2r2r3r3 &  &  &  &  & \\
\bottomrule
\end{tabular}
\endgroup{}
\end{table}
Question 8. How do you produce single, double and three way cross hybrids using cytoplasmic male sterility ? Explain.\\
Question 9. What is the frequency of the heterozygote (Bb) in random mating population, if the frequency of recessive phenotype (bb) is 0.04 ?\\
Question 10. For a quantitative trait in a RMP, mean is 100 and variation is 240. The regression of the offspring on mid-parent value is 0.25. Truncation selection is practiced with selection differential of 32. What is the expected mean in the next generation ?\\
Question 11. Explain pureline theory given by Johannsen.\\
\clearpage 
{\centering \Large{\textbf{Gokuleshwor Agriculture and Animal Science College, Baitadi}} \\[0.25cm]
            \textbf{Institute}: GAASC, IAAS \\[0.2cm]
            \textbf{Internal Assessment} \\[0.2cm]} 
\textbf{Credit hours}: 3 + 1 \\ 
\textbf{Level}: B. Sc. Ag (4th Semester) \\
\textbf{Roll no}: 17 \\[0.5cm] 
\textbf{Essay type} (1 x 10 = 10 marks) \\
Question 1. How do you transfer disease resistance gene from uncultivated genotype to cultivated cultivar, which is susceptible? Explain in detail.\\
\textbf{Short answer type} (10 x 3 = 30 marks) \\
Question 2. List different breeding methods used in wheat and maize crops. Give your logics why generally a long time is required to release a variety in self pollinated crop as compared to cross pollinated crops?\\
Question 3. Write about the objectives of plant breeding.\\
Question 4. How do you produce single, double and three way cross hybrids using cytoplasmic male sterility ? Explain.\\
Question 5. Make a partial diallel crossing scheme involving 11 parents and a three way cross involving 3 parents. Parents are represented like P1, P2, P3, …, P12.\\
Question 6. Define plant breeding. What are the major objectives of plant breeding.\\
Question 7. Briefly explain the self pollination enforcing mechanisms. What are the genetic consequences of self-pollination ?\\
Question 8. Differentiate qualitative and quantitative traits.\\
Question 9. How do you produce double cross hybrid using genetic male sterility ? Outline the procedure.\\
Question 10. What is the frequency of the heterozygote (Bb) in random mating population, if the frequency of recessive phenotype (bb) is 0.04 ?\\
Question 11. For a quantitative trait in a RMP, the mean is 100 and the variance is 240. The regression of the offspring on mid parent value is 0.25. Truncation selection is practiced with a selection differential of 32. What is the expected mean in the next generation? Also, find the heritability of the trait.\\
\clearpage 
{\centering \Large{\textbf{Gokuleshwor Agriculture and Animal Science College, Baitadi}} \\[0.25cm]
            \textbf{Institute}: GAASC, IAAS \\[0.2cm]
            \textbf{Internal Assessment} \\[0.2cm]} 
\textbf{Credit hours}: 3 + 1 \\ 
\textbf{Level}: B. Sc. Ag (4th Semester) \\
\textbf{Roll no}: 18 \\[0.5cm] 
\textbf{Essay type} (1 x 10 = 10 marks) \\
Question 1. There is a wheat variety which is very good in terms of yield and other performances. It lacks trait, for instance, say disease resistance. A local landrace with recessive resistance is found. How do you incorporate it in the above wheat variety which is otherwise good ?\\
\textbf{Short answer type} (10 x 3 = 30 marks) \\
Question 2. Outline the production of double cross hybrids using cytoplasmic-genetic male sterility.\\
Question 3. Enlist various types of breeding methods used in self-pollinated crops. And, how a disease resistant dominant gene is transferred from uncultivated genotype to cultivated cultivar ? Outline the breeding procedure.\\
Question 4. Explain the pureline theory of Johansen\\
Question 5. For a quantitative trait in a RMP, mean is 100 and variation is 240. The regression of the offspring on mid-parent value is 0.25. Truncation selection is practiced with selection differential of 32. What is the expected mean in the next generation ?\\
Question 6. What is hybridization ? Describe the steps involved in hybridization, in brief.\\
Question 7. How do you produce single, double and three way cross hybrids using cytoplasmic male sterility ? Explain.\\
Question 8. Explain the role of environment on quantitative character.\\
Question 9. Differentiate between qualitative and quantitative traits.\\
Question 10. What is multiple factor hypothesis ? Explain it on the basis of seed color in wheat.\\
Question 11. In a random mating poulation containing 100 individuals, 25 are recessive genotypes. Find gene and genotypic frequencies for the trait in the population.\\
\clearpage 
{\centering \Large{\textbf{Gokuleshwor Agriculture and Animal Science College, Baitadi}} \\[0.25cm]
            \textbf{Institute}: GAASC, IAAS \\[0.2cm]
            \textbf{Internal Assessment} \\[0.2cm]} 
\textbf{Credit hours}: 3 + 1 \\ 
\textbf{Level}: B. Sc. Ag (4th Semester) \\
\textbf{Roll no}: 19 \\[0.5cm] 
\textbf{Essay type} (1 x 10 = 10 marks) \\
Question 1. Suppose two inbred lines A and B are crossed to produce F1 hybrid. The F1 is selfed and F2 is produced. The genotype of inbred line A is AAbb and the genotype of inbred line B is aaBB. A and B are the dominant genes and contribute 12 cm and 10 cm towards the spike length of the hybrid respectively. In the absence of dominants, each recessive gene contributes 4 cm towards the spike length of the hybrid. The spike length of the best commercial variety if 25 cm. a. Find the spike length of the parents, F1 and F2 progeny b. Calculate average heterosis, heterobeltiosis, economic heterosis and inbreeding depression. c. Interpret the results.\\
\textbf{Short answer type} (10 x 3 = 30 marks) \\
Question 2. Define cytoplasmic male sterility. Show the cross how male sterility line is maintained during production of single cross hybrid and double cross hybrid ?\\
Question 3. What do you mean by Plant Breeder Right (PBR) ? What are the main points to be considered in getting PBR ?\\
Question 4. How are haploids/monoploids produced and utilized in plant breeding ?\\
Question 5. Briefly explain the self pollination enforcing mechanisms. What are the genetic consequences of self-pollination ?\\
Question 6. List different breeding methods used in wheat and maize crops. Give your logics why generally a long time is required to release a variety in self pollinated crop as compared to cross pollinated crops?\\
Question 7. Half sib and full sib selection breeding methods are used in crop improvement of cross pollinated crops like maize. Which one is the most effective and why?\\
Question 8. Explain the pureline theory of Johansen\\
Question 9. Mention current research activities carried out by NARC in the improvement of wheat crop.\\
Question 10. For a quantitative trait in a RMP, mean is 100 and variation is 240. The regression of the offspring on mid-parent value is 0.25. Truncation selection is practiced with selection differential of 32. What is the expected mean in the next generation ?\\
Question 11. What is gene for gene hypothesis ? Write compatible (+) and incompatible (-) reaction with the help of following information. Which are the most resistant and most susceptible reaction types and why ?\\ 
\begin{table}[H]
\centering\begingroup\fontsize{8}{10}\selectfont

\begin{tabular}[t]{llllll}
\toprule
Host genotypes & Pathogen G1 & Pathogen G2 & Pathogen G3 & Pathogen G4 & Pathogen G5\\
\midrule
 & a1a1a2a2a3a3 & A1A1a2a2a3a3 & A1A1A2A2a3a3 & a1a1a2a2A3A3 & A1A1A2A2A3A3\\
R1R1R2R2R3R3 &  &  &  &  & \\
R1R1r2r2R3R3 &  &  &  &  & \\
r1r1R2R2r3r3 &  &  &  &  & \\
r1r1r2r2R3R3 &  &  &  &  & \\
\addlinespace
r1r1r2r2r3r3 &  &  &  &  & \\
\bottomrule
\end{tabular}
\endgroup{}
\end{table}
\clearpage 
{\centering \Large{\textbf{Gokuleshwor Agriculture and Animal Science College, Baitadi}} \\[0.25cm]
            \textbf{Institute}: GAASC, IAAS \\[0.2cm]
            \textbf{Internal Assessment} \\[0.2cm]} 
\textbf{Credit hours}: 3 + 1 \\ 
\textbf{Level}: B. Sc. Ag (4th Semester) \\
\textbf{Roll no}: 20 \\[0.5cm] 
\textbf{Essay type} (1 x 10 = 10 marks) \\
Question 1. Define plant breeding. State different breeding methods in self and cross pollinated crops. Explain with diagram breeding methods that are practiced in self-pollinated crops.\\
\textbf{Short answer type} (10 x 3 = 30 marks) \\
Question 2. Briefly explain about hypothesis governing heterosis.\\
Question 3. Define cytoplasmic male sterility. Show the cross how male sterility line is maintained during production of single cross hybrid and double cross hybrid ?\\
Question 4. Briefly explain application of allopolyploids with the help of suitable examples.\\
Question 5. Briefly explain about research activities that are being carried out by Nepal Agriculture Research Council for the improvement of Wheat (Triticum aestivum) in the context of Nepal.\\
Question 6. What is multiple factor hypothesis ? Explain it on the basis of seed color in wheat.\\
Question 7. For a quantitative trait in a RMP, mean is 100 and variation is 240. The regression of the offspring on mid-parent value is 0.25. Truncation selection is practiced with selection differential of 32. What is the expected mean in the next generation ?\\
Question 8. Enlist different breeding methods used in Rice and Maize crops.\\
Question 9. Differentiate qualitative and quantitative traits.\\
Question 10. Differentiate between gametophytic and sporophytic systems of self-incompatibility.\\
Question 11. Differentiate between gametophytic and sporophytic systems of self-incompatibility with the help of well labelled diagrams.\\
\clearpage 
{\centering \Large{\textbf{Gokuleshwor Agriculture and Animal Science College, Baitadi}} \\[0.25cm]
            \textbf{Institute}: GAASC, IAAS \\[0.2cm]
            \textbf{Internal Assessment} \\[0.2cm]} 
\textbf{Credit hours}: 3 + 1 \\ 
\textbf{Level}: B. Sc. Ag (4th Semester) \\
\textbf{Roll no}: 21 \\[0.5cm] 
\textbf{Essay type} (1 x 10 = 10 marks) \\
Question 1. Define plant breeding. State different breeding methods in self and cross pollinated crops. Explain with diagram breeding methods that are practiced in self-pollinated crops.\\
\textbf{Short answer type} (10 x 3 = 30 marks) \\
Question 2. Give the conclusive remarks of pureline theory given by Johannsen on the basis of his study in French bean.\\
Question 3. A cultivated variety of wheat became susceptible to a fungal disease which drastically reduced the yield. However, a wild variety is resistant to this fungus. If the resistance is a dominant trait governed by "R" gene, is it possible to transfer this trait to the cultivated variety? Give procedure with your logics.\\
Question 4. Enlist different breeding methods used in Rice and Maize crops.\\
Question 5. What is patent ? Explain its requirements.\\
Question 6. State Hardy-Weinberg law. In a population consisting of 10000 individuals, 49 individuals are of "aa" genotype. If the population is in Hardy-Weinberg equilibrium, find the gene and genotype frequencies of that population.\\
Question 7. Differentiate between (any three): a. Inbreeders and outbreeders b. Race and pathotype c. Hybrids and synthetics d. Breeder's and certified seeds\\
Question 8. What do you mean by Plant Breeder Right (PBR) ? What are the main points to be considered in getting PBR ?\\
Question 9. How do you produce single, double, and three way cross hybrids? Explain with the help of suitable figures.\\
Question 10. Differentiate between gametophytic and sporophytic systems of self-incompatibility with the help of well labelled diagrams.\\
Question 11. What are defense mechanisms of host against natural enemy ? Explain.\\
\clearpage 
{\centering \Large{\textbf{Gokuleshwor Agriculture and Animal Science College, Baitadi}} \\[0.25cm]
            \textbf{Institute}: GAASC, IAAS \\[0.2cm]
            \textbf{Internal Assessment} \\[0.2cm]} 
\textbf{Credit hours}: 3 + 1 \\ 
\textbf{Level}: B. Sc. Ag (4th Semester) \\
\textbf{Roll no}: 22 \\[0.5cm] 
\textbf{Essay type} (1 x 10 = 10 marks) \\
Question 1. Define plant breeding. State different breeding methods in self and cross pollinated crops. Explain with diagram breeding methods that are practiced in self-pollinated crops.\\
\textbf{Short answer type} (10 x 3 = 30 marks) \\
Question 2. What do you mean by Plant Breeder Right (PBR) ? What are the main points to be considered in getting PBR ?\\
Question 3. Briefly explain about research activities that are being carried out by Nepal Agriculture Research Council for the improvement of Wheat (Triticum aestivum) in the context of Nepal.\\
Question 4. Define heterosis. Write the causes of heterosis and theories governing heterosis.\\
Question 5. What is hybridization ? Describe the steps involved in hybridization, in brief.\\
Question 6. Explain how genetic variation can be originated in a population.\\
Question 7. For a quantitative trait in a RMP, the mean is 100 and the variance is 240. The regression of the offspring on mid parent value is 0.25. Truncation selection is practiced with a selection differential of 32. What is the expected mean in the next generation? Also, find the heritability of the trait.\\
Question 8. Define plant breeding. What are the major objectives of plant breeding.\\
Question 9. Differentiate between qualitative and quantitative traits.\\
Question 10. Write about the objectives of plant breeding.\\
Question 11. What different types of progeny will occur in gametophytic and sporophytic system of self-incompatibility from following cross ? In which system, homozygous progeny can occur and why? The dominance relationship operated like, S1 $>$ S2 $>$ S3 $>$ S4.\\ 
\begin{table}[H]
\centering\begingroup\fontsize{8}{10}\selectfont

\begin{tabular}[t]{llllll}
\toprule
Male & Female & Gametophytic SI reaction & Gametophytic SI progeny & Sporophytic SI reaction & Sporophytic SI progeny\\
\midrule
S1S3 & S1S2 &  &  &  & \\
S1S2 & S1S3 &  &  &  & \\
S2S3 & S1S4 &  &  &  & \\
\bottomrule
\end{tabular}
\endgroup{}
\end{table}
\clearpage 
{\centering \Large{\textbf{Gokuleshwor Agriculture and Animal Science College, Baitadi}} \\[0.25cm]
            \textbf{Institute}: GAASC, IAAS \\[0.2cm]
            \textbf{Internal Assessment} \\[0.2cm]} 
\textbf{Credit hours}: 3 + 1 \\ 
\textbf{Level}: B. Sc. Ag (4th Semester) \\
\textbf{Roll no}: 23 \\[0.5cm] 
\textbf{Essay type} (1 x 10 = 10 marks) \\
Question 1. What is G x E interaction ? Explain the various methods of estimating G x E interaction.\\
\textbf{Short answer type} (10 x 3 = 30 marks) \\
Question 2. Write in short about the different activities in plant breeding directed to release a superior cultivar.\\
Question 3. How do you produce single, double, and three way cross hybrids? Explain with the help of suitable figures.\\
Question 4. What is patent ? Explain its requirements.\\
Question 5. From a random mating population with mean of 200 units, individuals with mean of 260 are selected to be the parents of the next generation. If the heritability of the trait is 0.6, what is your expected mean in the next generation ?\\
Question 6. How do you develop single and double cross hybrids using cytoplasmic male sterility ? Explain.\\
Question 7. What is male sterility ? List the various types of male sterility found in plants.\\
Question 8. How do you produce double cross hybrid using genetic male sterility ? Outline the procedure.\\
Question 9. The mean days to maturity and variance are 120 and 144 respectively. A plant breeder selected the top 5\% plants from base population and found mean days to maturity 110 in the next generation. Find the genetic gain and heritability of this trait.\\
Question 10. Explain the pureline theory of Johansen\\
Question 11. Differentiate between (any three): a. Inbreeders and outbreeders b. Race and pathotype c. Hybrids and synthetics d. Breeder's and certified seeds\\
\clearpage 
{\centering \Large{\textbf{Gokuleshwor Agriculture and Animal Science College, Baitadi}} \\[0.25cm]
            \textbf{Institute}: GAASC, IAAS \\[0.2cm]
            \textbf{Internal Assessment} \\[0.2cm]} 
\textbf{Credit hours}: 3 + 1 \\ 
\textbf{Level}: B. Sc. Ag (4th Semester) \\
\textbf{Roll no}: 24 \\[0.5cm] 
\textbf{Essay type} (1 x 10 = 10 marks) \\
Question 1. What do you mean by heterosis and inbreeding depression? A plant breeder crossed two genotypes of wheat AAbb and aaBB to get F1. The F1 was selfed to obtain F2. On the basis of heterosis governing theories (dominance and over dominance), find all kinds of heterosis and inbreeding depression. Each of the dominant homozygote, heterozygote and recessive homozygote contributes 4 ton, 6 ton and 2 ton per hectare in yield respectively. The commercial variety of wheat yields 5 ton/ha.\\
\textbf{Short answer type} (10 x 3 = 30 marks) \\
Question 2. How do you release a superior variety of rice (Oryza sativa) if the existing variety yields 3.5 tons per hectare and is 75\% disease resistant.\\
Question 3. Briefly explain application of allopolyploids with the help of suitable examples.\\
Question 4. What is hybridization ? Describe the steps involved in hybridization, in brief.\\
Question 5. Explain gene for gene relationship between host and a pathogen governing susceptible or resistance reaction.\\
Question 6. Explain briefly the various mechanism which promote self and cross pollination in crop plants.\\
Question 7. What different types of progeny will occur in gametophytic and sporophytic system of self-incompatibility from following cross ? In which system, homozygous progeny can occur and why? The dominance relationship operated like, S1 $>$ S2 $>$ S3 $>$ S4.\\ 
\begin{table}[H]
\centering\begingroup\fontsize{8}{10}\selectfont

\begin{tabular}[t]{llllll}
\toprule
Male & Female & Gametophytic SI reaction & Gametophytic SI progeny & Sporophytic SI reaction & Sporophytic SI progeny\\
\midrule
S1S3 & S1S2 &  &  &  & \\
S1S2 & S1S3 &  &  &  & \\
S2S3 & S1S4 &  &  &  & \\
\bottomrule
\end{tabular}
\endgroup{}
\end{table}
Question 8. State Hardy-Weinberg law. Compute gene and genotypic frequencies from the following data and mention how many plants are disease resistant. Note that the susceptible gen, R is dominant over resistant gene, r.\\ 
\begin{table}[H]
\centering\begingroup\fontsize{8}{10}\selectfont

\begin{tabular}[t]{llll}
\toprule
Genotype & RR & Rr & rr\\
\midrule
Frequency & 32 & 48 & 20\\
\bottomrule
\end{tabular}
\endgroup{}
\end{table}
Question 9. Write about the status of rice breeding in Nepal.\\
Question 10. Differentiate between gametophytic and sporophytic systems of self-incompatibility with the help of well labelled diagrams.\\
Question 11. Give the conclusive remarks of pureline theory given by Johannsen on the basis of his study in French bean.\\
\clearpage 
{\centering \Large{\textbf{Gokuleshwor Agriculture and Animal Science College, Baitadi}} \\[0.25cm]
            \textbf{Institute}: GAASC, IAAS \\[0.2cm]
            \textbf{Internal Assessment} \\[0.2cm]} 
\textbf{Credit hours}: 3 + 1 \\ 
\textbf{Level}: B. Sc. Ag (4th Semester) \\
\textbf{Roll no}: 25 \\[0.5cm] 
\textbf{Essay type} (1 x 10 = 10 marks) \\
Question 1. Define plant breeding. State different breeding methods in self and cross pollinated crops. Explain with diagram breeding methods that are practiced in self-pollinated crops.\\
\textbf{Short answer type} (10 x 3 = 30 marks) \\
Question 2. Briefly explain about hypothesis governing heterosis.\\
Question 3. What is hybridization ? Describe the steps involved in hybridization, in brief.\\
Question 4. How transgressive segregants are produced ? Present with the help of suitable figure.\\
Question 5. For a quantitative trait in a RMP, the mean is 100 and the variance is 240. The regression of the offspring on mid parent value is 0.25. Truncation selection is practiced with a selection differential of 32. What is the expected mean in the next generation? Also, find the heritability of the trait.\\
Question 6. Explain briefly the various mechanism which promote self and cross pollination in crop plants.\\
Question 7. Half sib and full sib selection breeding methods are used in crop improvement of cross pollinated crops like maize. Which one is the most effective and why?\\
Question 8. State Hardy-Weinberg law. In a population consisting of 10000 individuals, 49 individuals are of "aa" genotype. If the population is in Hardy-Weinberg equilibrium, find the gene and genotype frequencies of that population.\\
Question 9. How do you develop single and double cross hybrids using cytoplasmic male sterility ? Explain.\\
Question 10. Define heterosis. Write the causes of heterosis and theories governing heterosis.\\
Question 11. Explain how genetic variation can be originated in a population.\\
\clearpage 
{\centering \Large{\textbf{Gokuleshwor Agriculture and Animal Science College, Baitadi}} \\[0.25cm]
            \textbf{Institute}: GAASC, IAAS \\[0.2cm]
            \textbf{Internal Assessment} \\[0.2cm]} 
\textbf{Credit hours}: 3 + 1 \\ 
\textbf{Level}: B. Sc. Ag (4th Semester) \\
\textbf{Roll no}: 26 \\[0.5cm] 
\textbf{Essay type} (1 x 10 = 10 marks) \\
Question 1. Define plant breeding. State different breeding methods in self and cross pollinated crops. Explain with diagram breeding methods that are practiced in self-pollinated crops.\\
\textbf{Short answer type} (10 x 3 = 30 marks) \\
Question 2. Explain pureline theory on the basis of Johansen's experiment. What might be the application of this theory in breeding program?\\
Question 3. Explain briefly the various mechanism which promote self and cross pollination in crop plants.\\
Question 4. How are haploids/monoploids produced and utilized in plant breeding ?\\
Question 5. What do you mean by Plant Breeder Right (PBR) ? What are the main points to be considered in getting PBR ?\\
Question 6. Outline the production of double cross hybrids using cytoplasmic-genetic male sterility.\\
Question 7. Differentiate between (any three): a. Inbreeders and outbreeders b. Race and pathotype c. Hybrids and synthetics d. Breeder's and certified seeds\\
Question 8. Define intellectual property right. Explain its forms.\\
Question 9. How do you produce single, double and three way cross hybrids using cytoplasmic male sterility ? Explain.\\
Question 10. Enlist various types of breeding methods used in self-pollinated crops. And, how a disease resistant dominant gene is transferred from uncultivated genotype to cultivated cultivar ? Outline the breeding procedure.\\
Question 11. How do you produce single, double, and three way cross hybrids? Explain with the help of suitable figures.\\
\clearpage 
{\centering \Large{\textbf{Gokuleshwor Agriculture and Animal Science College, Baitadi}} \\[0.25cm]
            \textbf{Institute}: GAASC, IAAS \\[0.2cm]
            \textbf{Internal Assessment} \\[0.2cm]} 
\textbf{Credit hours}: 3 + 1 \\ 
\textbf{Level}: B. Sc. Ag (4th Semester) \\
\textbf{Roll no}: 27 \\[0.5cm] 
\textbf{Essay type} (1 x 10 = 10 marks) \\
Question 1. How do you transfer disease resistance gene from uncultivated genotype to cultivated cultivar, which is susceptible? Explain in detail.\\
\textbf{Short answer type} (10 x 3 = 30 marks) \\
Question 2. Explain the role of environment on quantitative character.\\
Question 3. Make a partial diallel crossing scheme involving 11 parents and a three way cross involving 3 parents. Parents are represented like P1, P2, P3, …, P12.\\
Question 4. How do you produce double cross hybrid using genetic male sterility ? Outline the procedure.\\
Question 5. Briefly explain about hypothesis governing heterosis.\\
Question 6. What is patent ? Explain its requirements.\\
Question 7. What is hybridization ? Describe the steps involved in hybridization, in brief.\\
Question 8. What is gene for gene hypothesis ? Write compatible (+) and incompatible (-) reaction with the help of following information. Which are the most resistant and most susceptible reaction types and why ?\\ 
\begin{table}[H]
\centering\begingroup\fontsize{8}{10}\selectfont

\begin{tabular}[t]{llllll}
\toprule
Host genotypes & Pathogen G1 & Pathogen G2 & Pathogen G3 & Pathogen G4 & Pathogen G5\\
\midrule
 & a1a1a2a2a3a3 & A1A1a2a2a3a3 & A1A1A2A2a3a3 & a1a1a2a2A3A3 & A1A1A2A2A3A3\\
R1R1R2R2R3R3 &  &  &  &  & \\
R1R1r2r2R3R3 &  &  &  &  & \\
r1r1R2R2r3r3 &  &  &  &  & \\
r1r1r2r2R3R3 &  &  &  &  & \\
\addlinespace
r1r1r2r2r3r3 &  &  &  &  & \\
\bottomrule
\end{tabular}
\endgroup{}
\end{table}
Question 9. Write about the objectives of plant breeding.\\
Question 10. Differentiate between full sib and half sib selection. Which selection scheme is the most effective in breeding maize ? Logically explain.\\
Question 11. Define plant breeding. What are the major objectives of plant breeding.\\
\clearpage 
{\centering \Large{\textbf{Gokuleshwor Agriculture and Animal Science College, Baitadi}} \\[0.25cm]
            \textbf{Institute}: GAASC, IAAS \\[0.2cm]
            \textbf{Internal Assessment} \\[0.2cm]} 
\textbf{Credit hours}: 3 + 1 \\ 
\textbf{Level}: B. Sc. Ag (4th Semester) \\
\textbf{Roll no}: 28 \\[0.5cm] 
\textbf{Essay type} (1 x 10 = 10 marks) \\
Question 1. What is G x E interaction ? Explain the various methods of estimating G x E interaction.\\
\textbf{Short answer type} (10 x 3 = 30 marks) \\
Question 2. Make a partial diallel crossing scheme involving 11 parents and a three way cross involving 3 parents. Parents are represented like P1, P2, P3, …, P12.\\
Question 3. Enlist various types of breeding methods used in self-pollinated crops. And, how a disease resistant dominant gene is transferred from uncultivated genotype to cultivated cultivar ? Outline the breeding procedure.\\
Question 4. Differentiate between gametophytic and sporophytic systems of self-incompatibility.\\
Question 5. Differentiate between full sib and half sib selection. Which selection scheme is the most effective in breeding maize ? Logically explain.\\
Question 6. Define cytoplasmic male sterility. Show the cross how male sterility line is maintained during production of single cross hybrid and double cross hybrid ?\\
Question 7. Explain the pureline theory of Johansen\\
Question 8. Present current research activities carried out by Nepal Agriculture Research Council (NARC) for the improvement of wheat crop.\\
Question 9. Write main achievements of plant breeding in the context of Nepal.\\
Question 10. The mean days to maturity and variance are 120 and 144 respectively. A plant breeder selected the top 5\% plants from base population and found mean days to maturity 110 in the next generation. Find the genetic gain and heritability of this trait.\\
Question 11. What is the frequency of the heterozygote (Bb) in random mating population, if the frequency of recessive phenotype (bb) is 0.04 ?\\
\clearpage 
{\centering \Large{\textbf{Gokuleshwor Agriculture and Animal Science College, Baitadi}} \\[0.25cm]
            \textbf{Institute}: GAASC, IAAS \\[0.2cm]
            \textbf{Internal Assessment} \\[0.2cm]} 
\textbf{Credit hours}: 3 + 1 \\ 
\textbf{Level}: B. Sc. Ag (4th Semester) \\
\textbf{Roll no}: 29 \\[0.5cm] 
\textbf{Essay type} (1 x 10 = 10 marks) \\
Question 1. Suppose two inbred lines A and B are crossed to produce F1 hybrid. The F1 is selfed and F2 is produced. The genotype of inbred line A is AAbb and the genotype of inbred line B is aaBB. A and B are the dominant genes and contribute 12 cm towards the spike length of the hybrid. In the absence of dominants, each recessive gene contributes 8 cm towards the spike length of the hybrid. The spike length of the best commercial variety is 22 cm. a. Find the spike length of the parents, F1 and F2 progeny b. Find all kinds of Heterosis and Inbreeding depression c. Interpret the results.\\
\textbf{Short answer type} (10 x 3 = 30 marks) \\
Question 2. How do you produce single, double, and three way cross hybrids? Explain with the help of suitable figures.\\
Question 3. From a random mating population with mean of 200 units, individuals with mean of 260 are selected to be the parents of the next generation. If the heritability of the trait is 0.6, what is your expected mean in the next generation ?\\
Question 4. What is multiple factor hypothesis ? Explain it on the basis of seed color in wheat.\\
Question 5. Briefly explain application of allopolyploids with the help of suitable examples.\\
Question 6. Briefly explain about research activities that are being carried out by Nepal Agriculture Research Council for the improvement of Wheat (Triticum aestivum) in the context of Nepal.\\
Question 7. How do you release a superior variety of rice (Oryza sativa) if the existing variety yields 3.5 tons per hectare and is 75\% disease resistant.\\
Question 8. State law of homologous series in variation by NI Vavilov. Discuss the relationship between, primary, secondary and tertiary gene pools with respect to their combining ability.\\
Question 9. What is intellectual property right? Write the requirements of patent.\\
Question 10. What is patent ? Explain its requirements.\\
Question 11. How are haploids/monoploids produced and utilized in plant breeding ?\\
\clearpage 
{\centering \Large{\textbf{Gokuleshwor Agriculture and Animal Science College, Baitadi}} \\[0.25cm]
            \textbf{Institute}: GAASC, IAAS \\[0.2cm]
            \textbf{Internal Assessment} \\[0.2cm]} 
\textbf{Credit hours}: 3 + 1 \\ 
\textbf{Level}: B. Sc. Ag (4th Semester) \\
\textbf{Roll no}: 30 \\[0.5cm] 
\textbf{Essay type} (1 x 10 = 10 marks) \\
Question 1. There is a wheat variety which is very good in terms of yield and other performances. It lacks trait, for instance, say disease resistance. A local landrace with recessive resistance is found. How do you incorporate it in the above wheat variety which is otherwise good ?\\
\textbf{Short answer type} (10 x 3 = 30 marks) \\
Question 2. Write about the status of rice breeding in Nepal.\\
Question 3. Give the conclusive remarks of pureline theory given by Johannsen on the basis of his study in French bean.\\
Question 4. What different types of progeny will occur in gametophytic and sporophytic system of self-incompatibility from following cross ? In which system, homozygous progeny can occur and why? The dominance relationship operated like, S1 $>$ S2 $>$ S3 $>$ S4.\\ 
\begin{table}[H]
\centering\begingroup\fontsize{8}{10}\selectfont

\begin{tabular}[t]{llllll}
\toprule
Male & Female & Gametophytic SI reaction & Gametophytic SI progeny & Sporophytic SI reaction & Sporophytic SI progeny\\
\midrule
S1S3 & S1S2 &  &  &  & \\
S1S2 & S1S3 &  &  &  & \\
S2S3 & S1S4 &  &  &  & \\
\bottomrule
\end{tabular}
\endgroup{}
\end{table}
Question 5. From a random mating population with mean of 200 units, individuals with mean of 260 are selected to be the parents of the next generation. If the heritability of the trait is 0.6, what is your expected mean in the next generation ?\\
Question 6. Present current research activities carried out by Nepal Agriculture Research Council (NARC) for the improvement of wheat crop.\\
Question 7. Differentiate between gametophytic and sporophytic systems of self-incompatibility.\\
Question 8. How do you produce double cross hybrid using genetic male sterility ? Outline the procedure.\\
Question 9. Explain pureline theory on the basis of Johansen's experiment. What might be the application of this theory in breeding program?\\
Question 10. Explain the role of environment on quantitative character.\\
Question 11. Differentiate between: a. Self incompatibility and male sterility b. Self pollinated crops and cross pollinated crops\\
\clearpage 
{\centering \Large{\textbf{Gokuleshwor Agriculture and Animal Science College, Baitadi}} \\[0.25cm]
            \textbf{Institute}: GAASC, IAAS \\[0.2cm]
            \textbf{Internal Assessment} \\[0.2cm]} 
\textbf{Credit hours}: 3 + 1 \\ 
\textbf{Level}: B. Sc. Ag (4th Semester) \\
\textbf{Roll no}: 31 \\[0.5cm] 
\textbf{Essay type} (1 x 10 = 10 marks) \\
Question 1. Suppose two inbred lines A and B are crossed to produce F1 hybrid. The F1 is selfed and F2 is produced. The genotype of inbred line A is AAbb and the genotype of inbred line B is aaBB. A and B are the dominant genes and contribute 12 cm and 10 cm towards the spike length of the hybrid respectively. In the absence of dominants, each recessive gene contributes 4 cm towards the spike length of the hybrid. The spike length of the best commercial variety if 25 cm. a. Find the spike length of the parents, F1 and F2 progeny b. Calculate average heterosis, heterobeltiosis, economic heterosis and inbreeding depression. c. Interpret the results.\\
\textbf{Short answer type} (10 x 3 = 30 marks) \\
Question 2. How do you release a superior variety of rice (Oryza sativa) if the existing variety yields 3.5 tons per hectare and is 75\% disease resistant.\\
Question 3. In a random mating poulation containing 100 individuals, 25 are recessive genotypes. Find gene and genotypic frequencies for the trait in the population.\\
Question 4. Present current research activities carried out by Nepal Agriculture Research Council (NARC) for the improvement of wheat crop.\\
Question 5. For a quantitative trait in a RMP, the mean is 100 and the variance is 240. The regression of the offspring on mid parent value is 0.25. Truncation selection is practiced with a selection differential of 32. What is the expected mean in the next generation? Also, find the heritability of the trait.\\
Question 6. What are the breeding objectives for rice ? Discuss.\\
Question 7. A cultivated variety of wheat became susceptible to a fungal disease which drastically reduced the yield. However, a wild variety is resistant to this fungus. If the resistance is a dominant trait governed by "R" gene, is it possible to transfer this trait to the cultivated variety? Give procedure with your logics.\\
Question 8. What are defense mechanisms of host against natural enemy ? Explain.\\
Question 9. Briefly explain about hypothesis governing heterosis.\\
Question 10. How do you produce single, double and three way cross hybrids using cytoplasmic male sterility ? Explain.\\
Question 11. Write about the status of rice breeding in Nepal.\\
\clearpage 
{\centering \Large{\textbf{Gokuleshwor Agriculture and Animal Science College, Baitadi}} \\[0.25cm]
            \textbf{Institute}: GAASC, IAAS \\[0.2cm]
            \textbf{Internal Assessment} \\[0.2cm]} 
\textbf{Credit hours}: 3 + 1 \\ 
\textbf{Level}: B. Sc. Ag (4th Semester) \\
\textbf{Roll no}: 32 \\[0.5cm] 
\textbf{Essay type} (1 x 10 = 10 marks) \\
Question 1. Define plant breeding. State different breeding methods in self and cross pollinated crops. Explain with diagram breeding methods that are practiced in self-pollinated crops.\\
\textbf{Short answer type} (10 x 3 = 30 marks) \\
Question 2. How do you produce double cross hybrid using genetic male sterility ? Outline the procedure.\\
Question 3. Briefly explain application of allopolyploids with the help of suitable examples.\\
Question 4. In a random mating poulation, the mean plant height and variance are 120 cm and 121 square cm respectively. A plant breeder selected the top 5\% plants from the base population and found mean plant height 110 cm in the next generation. Find the gentic gain, selection differential and heritability of this trait.\\
Question 5. What is patent ? Explain its requirements.\\
Question 6. What do you mean by Plant Breeder Right (PBR) ? What are the main points to be considered in getting PBR ?\\
Question 7. Briefly explain the different defence mechanisms of host against pathogen/parasite. Which of the horizontal or vertical resitance is desirable in a commercial cutivar. Why?\\
Question 8. Explain briefly the various mechanism which promote self and cross pollination in crop plants.\\
Question 9. Define intellectual property right. Explain its forms.\\
Question 10. Define heterosis. Write the causes of heterosis and theories governing heterosis.\\
Question 11. Write about the objectives of plant breeding.\\
\clearpage 
{\centering \Large{\textbf{Gokuleshwor Agriculture and Animal Science College, Baitadi}} \\[0.25cm]
            \textbf{Institute}: GAASC, IAAS \\[0.2cm]
            \textbf{Internal Assessment} \\[0.2cm]} 
\textbf{Credit hours}: 3 + 1 \\ 
\textbf{Level}: B. Sc. Ag (4th Semester) \\
\textbf{Roll no}: 33 \\[0.5cm] 
\textbf{Essay type} (1 x 10 = 10 marks) \\
Question 1. Suppose two inbred lines A and B are crossed to produce F1 hybrid. The F1 is selfed and F2 is produced. The genotype of inbred line A is AAbb and the genotype of inbred line B is aaBB. A and B are the dominant genes and contribute 12 cm and 10 cm towards the spike length of the hybrid respectively. In the absence of dominants, each recessive gene contributes 4 cm towards the spike length of the hybrid. The spike length of the best commercial variety if 25 cm. a. Find the spike length of the parents, F1 and F2 progeny b. Calculate average heterosis, heterobeltiosis, economic heterosis and inbreeding depression. c. Interpret the results.\\
\textbf{Short answer type} (10 x 3 = 30 marks) \\
Question 2. How do you produce single, double and three way cross hybrids using cytoplasmic male sterility ? Explain.\\
Question 3. Write about the objectives of plant breeding.\\
Question 4. How do you develop single and double cross hybrids using cytoplasmic male sterility ? Explain.\\
Question 5. Explain briefly the various mechanism which promote self and cross pollination in crop plants.\\
Question 6. Write various breeding methods used in self-pollinated crops. How do you transfer a disease resistance dominant gene, R, from non-cultivated genotype to a commercially cultivated variety which is disease susceptible ? Outline the breeding procedure.\\
Question 7. Write about the status of rice breeding in Nepal.\\
Question 8. Briefly explain application of allopolyploids with the help of suitable examples.\\
Question 9. Define intellectual property right. Explain its forms.\\
Question 10. Define heterosis. Write the causes of heterosis and theories governing heterosis.\\
Question 11. Enlist different breeding methods used in Rice and Maize crops.\\
\clearpage 
{\centering \Large{\textbf{Gokuleshwor Agriculture and Animal Science College, Baitadi}} \\[0.25cm]
            \textbf{Institute}: GAASC, IAAS \\[0.2cm]
            \textbf{Internal Assessment} \\[0.2cm]} 
\textbf{Credit hours}: 3 + 1 \\ 
\textbf{Level}: B. Sc. Ag (4th Semester) \\
\textbf{Roll no}: 34 \\[0.5cm] 
\textbf{Essay type} (1 x 10 = 10 marks) \\
Question 1. Suppose two inbred lines A and B are crossed to produce F1 hybrid. The F1 is selfed and F2 is produced. The genotype of inbred line A is AAbb and the genotype of inbred line B is aaBB. A and B are the dominant genes and contribute 12 cm and 10 cm towards the spike length of the hybrid respectively. In the absence of dominants, each recessive gene contributes 4 cm towards the spike length of the hybrid. The spike length of the best commercial variety if 25 cm. a. Find the spike length of the parents, F1 and F2 progeny b. Calculate average heterosis, heterobeltiosis, economic heterosis and inbreeding depression. c. Interpret the results.\\
\textbf{Short answer type} (10 x 3 = 30 marks) \\
Question 2. State law of homologous series in variation by NI Vavilov. Discuss the relationship between, primary, secondary and tertiary gene pools with respect to their combining ability.\\
Question 3. Make a partial diallel crossing scheme involving 11 parents and a three way cross involving 3 parents. Parents are represented like P1, P2, P3, …, P12.\\
Question 4. How do you produce single, double, and three way cross hybrids? Explain with the help of suitable figures.\\
Question 5. Briefly explain the self pollination enforcing mechanisms. What are the genetic consequences of self-pollination ?\\
Question 6. Explain briefly the various mechanism which promote self and cross pollination in crop plants.\\
Question 7. What do you mean by Plant Breeder Right (PBR) ? What are the main points to be considered in getting PBR ?\\
Question 8. What is patent ? Explain its requirements.\\
Question 9. Explain pureline theory on the basis of Johansen's experiment. What might be the application of this theory in breeding program?\\
Question 10. How do you produce hybrid seed using one self-incompatible (P1) and another self-compatible (P2) parents ?\\
Question 11. State Hardy-Weinberg law. Compute gene and genotypic frequencies from the following data and mention how many plants are disease resistant. Note that the susceptible gen, R is dominant over resistant gene, r.\\ 
\begin{table}[H]
\centering\begingroup\fontsize{8}{10}\selectfont

\begin{tabular}[t]{llll}
\toprule
Genotype & RR & Rr & rr\\
\midrule
Frequency & 32 & 48 & 20\\
\bottomrule
\end{tabular}
\endgroup{}
\end{table}
\clearpage 
{\centering \Large{\textbf{Gokuleshwor Agriculture and Animal Science College, Baitadi}} \\[0.25cm]
            \textbf{Institute}: GAASC, IAAS \\[0.2cm]
            \textbf{Internal Assessment} \\[0.2cm]} 
\textbf{Credit hours}: 3 + 1 \\ 
\textbf{Level}: B. Sc. Ag (4th Semester) \\
\textbf{Roll no}: 35 \\[0.5cm] 
\textbf{Essay type} (1 x 10 = 10 marks) \\
Question 1. There is a wheat variety which is very good in terms of yield and other performances. It lacks trait, for instance, say disease resistance. A local landrace with recessive resistance is found. How do you incorporate it in the above wheat variety which is otherwise good ?\\
\textbf{Short answer type} (10 x 3 = 30 marks) \\
Question 2. What is male sterility ? List the various types of male sterility found in plants.\\
Question 3. Differentiate between gametophytic and sporophytic systems of self-incompatibility with the help of well labelled diagrams.\\
Question 4. How do you improve vegetatively propagated crops ? Describe a method of your choice with example.\\
Question 5. Write about the objectives of plant breeding.\\
Question 6. What is hybridization ? Describe the steps involved in hybridization, in brief.\\
Question 7. Write in short about the different activities in plant breeding directed to release a superior cultivar.\\
Question 8. Briefly explain the self pollination enforcing mechanisms. What are the genetic consequences of self-pollination ?\\
Question 9. The mean days to maturity and variance are 120 and 144 respectively. A plant breeder selected the top 5\% plants from base population and found mean days to maturity 110 in the next generation. Find the genetic gain and heritability of this trait.\\
Question 10. Differentiate qualitative and quantitative traits.\\
Question 11. What do you mean by Plant Breeder Right (PBR) ? What are the main points to be considered in getting PBR ?\\
\clearpage 
{\centering \Large{\textbf{Gokuleshwor Agriculture and Animal Science College, Baitadi}} \\[0.25cm]
            \textbf{Institute}: GAASC, IAAS \\[0.2cm]
            \textbf{Internal Assessment} \\[0.2cm]} 
\textbf{Credit hours}: 3 + 1 \\ 
\textbf{Level}: B. Sc. Ag (4th Semester) \\
\textbf{Roll no}: 36 \\[0.5cm] 
\textbf{Essay type} (1 x 10 = 10 marks) \\
Question 1. Define plant breeding. State different breeding methods in self and cross pollinated crops. Explain with diagram breeding methods that are practiced in self-pollinated crops.\\
\textbf{Short answer type} (10 x 3 = 30 marks) \\
Question 2. From the following data, calculate heterosis, average heterosis and economic heterosis for grain yield of a wheat hybrid.\\ 
\begin{table}[H]
\centering\begingroup\fontsize{8}{10}\selectfont

\begin{tabular}[t]{ll}
\toprule
Parents & Grain yield (t/ha)\\
\midrule
A & 6.6\\
B & 4.2\\
F1(AxB) & 8.5\\
Best commercial variety in area (Gautam) & 7\\
\bottomrule
\end{tabular}
\endgroup{}
\end{table}
Question 3. Differentiate between full sib and half sib selection. Which selection scheme is the most effective in breeding maize ? Logically explain.\\
Question 4. State Hardy-Weinberg law. Compute gene and genotypic frequencies from the following data and mention how many plants are disease resistant. Note that the susceptible gen, R is dominant over resistant gene, r.\\ 
\begin{table}[H]
\centering\begingroup\fontsize{8}{10}\selectfont

\begin{tabular}[t]{llll}
\toprule
Genotype & RR & Rr & rr\\
\midrule
Frequency & 32 & 48 & 20\\
\bottomrule
\end{tabular}
\endgroup{}
\end{table}
Question 5. Write main achievements of plant breeding in the context of Nepal.\\
Question 6. How do you develop single and double cross hybrids using cytoplasmic male sterility ? Explain.\\
Question 7. Write about the objectives of plant breeding.\\
Question 8. What is hybridization ? Describe the steps involved in hybridization, in brief.\\
Question 9. Briefly explain the different defence mechanisms of host against pathogen/parasite. Which of the horizontal or vertical resitance is desirable in a commercial cutivar. Why?\\
Question 10. Define cytoplasmic male sterility. Show the cross how male sterility line is maintained during production of single cross hybrid and double cross hybrid ?\\
Question 11. What do you mean by Plant Breeder Right (PBR) ? What are the main points to be considered in getting PBR ?\\
\clearpage 
{\centering \Large{\textbf{Gokuleshwor Agriculture and Animal Science College, Baitadi}} \\[0.25cm]
            \textbf{Institute}: GAASC, IAAS \\[0.2cm]
            \textbf{Internal Assessment} \\[0.2cm]} 
\textbf{Credit hours}: 3 + 1 \\ 
\textbf{Level}: B. Sc. Ag (4th Semester) \\
\textbf{Roll no}: 37 \\[0.5cm] 
\textbf{Essay type} (1 x 10 = 10 marks) \\
Question 1. There is a wheat variety which is very good in terms of yield and other performances. It lacks trait, for instance, say disease resistance. A local landrace with recessive resistance is found. How do you incorporate it in the above wheat variety which is otherwise good ?\\
\textbf{Short answer type} (10 x 3 = 30 marks) \\
Question 2. From a random mating population with mean of 200 units, individuals with mean of 260 are selected to be the parents of the next generation. If the heritability of the trait is 0.6, what is your expected mean in the next generation ?\\
Question 3. Explain gene for gene relationship between host and a pathogen governing susceptible or resistance reaction.\\
Question 4. Briefly explain about hypothesis governing heterosis.\\
Question 5. In a random mating poulation, the mean plant height and variance are 120 cm and 121 square cm respectively. A plant breeder selected the top 5\% plants from the base population and found mean plant height 110 cm in the next generation. Find the gentic gain, selection differential and heritability of this trait.\\
Question 6. Write short notes on (any three): a. Center of origin b. Self incompatibility c. Chemical hybridizing agents d. Plant breeders' rights\\
Question 7. How do you develop single and double cross hybrids using cytoplasmic male sterility ? Explain.\\
Question 8. Explain how genetic variation can be originated in a population.\\
Question 9. Give the conclusive remarks of pureline theory given by Johannsen on the basis of his study in French bean.\\
Question 10. How do you produce single, double, and three way cross hybrids? Explain with the help of suitable figures.\\
Question 11. What is the frequency of the heterozygote (Bb) in random mating population, if the frequency of recessive phenotype (bb) is 0.04 ?\\
\clearpage 
{\centering \Large{\textbf{Gokuleshwor Agriculture and Animal Science College, Baitadi}} \\[0.25cm]
            \textbf{Institute}: GAASC, IAAS \\[0.2cm]
            \textbf{Internal Assessment} \\[0.2cm]} 
\textbf{Credit hours}: 3 + 1 \\ 
\textbf{Level}: B. Sc. Ag (4th Semester) \\
\textbf{Roll no}: 38 \\[0.5cm] 
\textbf{Essay type} (1 x 10 = 10 marks) \\
Question 1. Define plant breeding. State different breeding methods in self and cross pollinated crops. Explain with diagram breeding methods that are practiced in self-pollinated crops.\\
\textbf{Short answer type} (10 x 3 = 30 marks) \\
Question 2. How are haploids/monoploids produced and utilized in plant breeding ?\\
Question 3. Mention current research activities carried out by NARC in the improvement of wheat crop.\\
Question 4. Explain the pureline theory of Johansen\\
Question 5. What are defense mechanisms of host against natural enemy ? Explain.\\
Question 6. Explain how genetic variation can be originated in a population.\\
Question 7. Differentiate qualitative and quantitative traits.\\
Question 8. Define plant breeding. What are the major objectives of plant breeding.\\
Question 9. From the following data, calculate heterosis, average heterosis and economic heterosis for grain yield of a wheat hybrid.\\ 
\begin{table}[H]
\centering\begingroup\fontsize{8}{10}\selectfont

\begin{tabular}[t]{ll}
\toprule
Parents & Grain yield (t/ha)\\
\midrule
A & 6.6\\
B & 4.2\\
F1(AxB) & 8.5\\
Best commercial variety in area (Gautam) & 7\\
\bottomrule
\end{tabular}
\endgroup{}
\end{table}
Question 10. Differentiate between (any three): a. Inbreeders and outbreeders b. Race and pathotype c. Hybrids and synthetics d. Breeder's and certified seeds\\
Question 11. What is the frequency of the heterozygote (Bb) in random mating population, if the frequency of recessive phenotype (bb) is 0.04 ?\\
\clearpage 
{\centering \Large{\textbf{Gokuleshwor Agriculture and Animal Science College, Baitadi}} \\[0.25cm]
            \textbf{Institute}: GAASC, IAAS \\[0.2cm]
            \textbf{Internal Assessment} \\[0.2cm]} 
\textbf{Credit hours}: 3 + 1 \\ 
\textbf{Level}: B. Sc. Ag (4th Semester) \\
\textbf{Roll no}: 39 \\[0.5cm] 
\textbf{Essay type} (1 x 10 = 10 marks) \\
Question 1. Define plant breeding. State different breeding methods in self and cross pollinated crops. Explain with diagram breeding methods that are practiced in self-pollinated crops.\\
\textbf{Short answer type} (10 x 3 = 30 marks) \\
Question 2. Briefly explain about research activities that are being carried out by Nepal Agriculture Research Council for the improvement of Wheat (Triticum aestivum) in the context of Nepal.\\
Question 3. Write about the status of rice breeding in Nepal.\\
Question 4. How are haploids/monoploids produced and utilized in plant breeding ?\\
Question 5. Write main achievements of plant breeding in the context of Nepal.\\
Question 6. Define plant breeding. What are the major objectives of plant breeding.\\
Question 7. Briefly explain the different defence mechanisms of host against pathogen/parasite. Which of the horizontal or vertical resitance is desirable in a commercial cutivar. Why?\\
Question 8. How do you produce hybrid seed using one self-incompatible (P1) and another self-compatible (P2) parents ?\\
Question 9. Differentiate qualitative and quantitative traits.\\
Question 10. Explain the role of environment on quantitative character.\\
Question 11. Write in short about the different activities in plant breeding directed to release a superior cultivar.\\
\clearpage 
{\centering \Large{\textbf{Gokuleshwor Agriculture and Animal Science College, Baitadi}} \\[0.25cm]
            \textbf{Institute}: GAASC, IAAS \\[0.2cm]
            \textbf{Internal Assessment} \\[0.2cm]} 
\textbf{Credit hours}: 3 + 1 \\ 
\textbf{Level}: B. Sc. Ag (4th Semester) \\
\textbf{Roll no}: 40 \\[0.5cm] 
\textbf{Essay type} (1 x 10 = 10 marks) \\
Question 1. Suppose two inbred lines A and B are crossed to produce F1 hybrid. The F1 is selfed and F2 is produced. The genotype of inbred line A is AAbb and the genotype of inbred line B is aaBB. A and B are the dominant genes and contribute 12 cm and 10 cm towards the spike length of the hybrid respectively. In the absence of dominants, each recessive gene contributes 4 cm towards the spike length of the hybrid. The spike length of the best commercial variety if 25 cm. a. Find the spike length of the parents, F1 and F2 progeny b. Calculate average heterosis, heterobeltiosis, economic heterosis and inbreeding depression. c. Interpret the results.\\
\textbf{Short answer type} (10 x 3 = 30 marks) \\
Question 2. Explain how genetic variation can be originated in a population.\\
Question 3. Differentiate between: a. Self incompatibility and male sterility b. Self pollinated crops and cross pollinated crops\\
Question 4. State law of homologous series in variation by NI Vavilov. Discuss the relationship between, primary, secondary and tertiary gene pools with respect to their combining ability.\\
Question 5. Define intellectual property right. Explain its forms.\\
Question 6. What are the breeding objectives for rice ? Discuss.\\
Question 7. Explain the pureline theory of Johansen\\
Question 8. What different types of progeny will occur in gametophytic and sporophytic system of self-incompatibility from following cross ? In which system, homozygous progeny can occur and why? The dominance relationship operated like, S1 $>$ S2 $>$ S3 $>$ S4.\\ 
\begin{table}[H]
\centering\begingroup\fontsize{8}{10}\selectfont

\begin{tabular}[t]{llllll}
\toprule
Male & Female & Gametophytic SI reaction & Gametophytic SI progeny & Sporophytic SI reaction & Sporophytic SI progeny\\
\midrule
S1S3 & S1S2 &  &  &  & \\
S1S2 & S1S3 &  &  &  & \\
S2S3 & S1S4 &  &  &  & \\
\bottomrule
\end{tabular}
\endgroup{}
\end{table}
Question 9. Give the conclusive remarks of pureline theory given by Johannsen on the basis of his study in French bean.\\
Question 10. From a random mating population with mean of 200 units, individuals with mean of 260 are selected to be the parents of the next generation. If the heritability of the trait is 0.6, what is your expected mean in the next generation ?\\
Question 11. The mean days to maturity and variance are 120 and 144 respectively. A plant breeder selected the top 5\% plants from base population and found mean days to maturity 110 in the next generation. Find the genetic gain and heritability of this trait.\\
\clearpage 
{\centering \Large{\textbf{Gokuleshwor Agriculture and Animal Science College, Baitadi}} \\[0.25cm]
            \textbf{Institute}: GAASC, IAAS \\[0.2cm]
            \textbf{Internal Assessment} \\[0.2cm]} 
\textbf{Credit hours}: 3 + 1 \\ 
\textbf{Level}: B. Sc. Ag (4th Semester) \\
\textbf{Roll no}: 41 \\[0.5cm] 
\textbf{Essay type} (1 x 10 = 10 marks) \\
Question 1. How do you transfer disease resistance gene from uncultivated genotype to cultivated cultivar, which is susceptible? Explain in detail.\\
\textbf{Short answer type} (10 x 3 = 30 marks) \\
Question 2. Briefly explain application of allopolyploids with the help of suitable examples.\\
Question 3. How transgressive segregants are produced ? Present with the help of suitable figure.\\
Question 4. Explain the role of environment on quantitative character.\\
Question 5. What is the frequency of the heterozygote (Bb) in random mating population, if the frequency of recessive phenotype (bb) is 0.04 ?\\
Question 6. What is hybridization ? Describe the steps involved in hybridization, in brief.\\
Question 7. A cultivated variety of wheat became susceptible to a fungal disease which drastically reduced the yield. However, a wild variety is resistant to this fungus. If the resistance is a dominant trait governed by "R" gene, is it possible to transfer this trait to the cultivated variety? Give procedure with your logics.\\
Question 8. Differentiate between gametophytic and sporophytic systems of self-incompatibility.\\
Question 9. Briefly explain about hypothesis governing heterosis.\\
Question 10. Define intellectual property right. Explain its forms.\\
Question 11. How do you produce single, double, and three way cross hybrids? Explain with the help of suitable figures.\\
\clearpage 
{\centering \Large{\textbf{Gokuleshwor Agriculture and Animal Science College, Baitadi}} \\[0.25cm]
            \textbf{Institute}: GAASC, IAAS \\[0.2cm]
            \textbf{Internal Assessment} \\[0.2cm]} 
\textbf{Credit hours}: 3 + 1 \\ 
\textbf{Level}: B. Sc. Ag (4th Semester) \\
\textbf{Roll no}: 42 \\[0.5cm] 
\textbf{Essay type} (1 x 10 = 10 marks) \\
Question 1. Define plant breeding. State different breeding methods in self and cross pollinated crops. Explain with diagram breeding methods that are practiced in self-pollinated crops.\\
\textbf{Short answer type} (10 x 3 = 30 marks) \\
Question 2. Explain the pureline theory of Johansen\\
Question 3. Briefly explain application of allopolyploids with the help of suitable examples.\\
Question 4. What is patent ? Explain its requirements.\\
Question 5. What are defense mechanisms of host against natural enemy ? Explain.\\
Question 6. On the basis of following table, answer the following questions: a. Which cultivar is the most susceptible and why ? b. Which cultivar is the most resistant and why ? c. Which cultivar is the most tolerant and why ? d. Which cultivar is the most sensitive and why ?\\ 
\begin{table}[H]
\centering\begingroup\fontsize{8}{10}\selectfont

\begin{tabular}[t]{lllll}
\toprule
Cultivar & Virus concentration & Yellowing & Yield with virus & Yield without virus\\
\midrule
A & 100 & 8 & 80 & 90\\
B & 60 & 0 & 97 & 100\\
C & 50 & 0 & 90 & 70\\
\bottomrule
\end{tabular}
\endgroup{}
\end{table}
Question 7. How do you produce single, double and three way cross hybrids using cytoplasmic male sterility ? Explain.\\
Question 8. From the following data, calculate heterosis, average heterosis and economic heterosis for grain yield of a wheat hybrid.\\ 
\begin{table}[H]
\centering\begingroup\fontsize{8}{10}\selectfont

\begin{tabular}[t]{ll}
\toprule
Parents & Grain yield (t/ha)\\
\midrule
A & 6.6\\
B & 4.2\\
F1(AxB) & 8.5\\
Best commercial variety in area (Gautam) & 7\\
\bottomrule
\end{tabular}
\endgroup{}
\end{table}
Question 9. What is intellectual property right? Write the requirements of patent.\\
Question 10. What is gene for gene hypothesis ? Write compatible (+) and incompatible (-) reaction with the help of following information. Which are the most resistant and most susceptible reaction types and why ?\\ 
\begin{table}[H]
\centering\begingroup\fontsize{8}{10}\selectfont

\begin{tabular}[t]{llllll}
\toprule
Host genotypes & Pathogen G1 & Pathogen G2 & Pathogen G3 & Pathogen G4 & Pathogen G5\\
\midrule
 & a1a1a2a2a3a3 & A1A1a2a2a3a3 & A1A1A2A2a3a3 & a1a1a2a2A3A3 & A1A1A2A2A3A3\\
R1R1R2R2R3R3 &  &  &  &  & \\
R1R1r2r2R3R3 &  &  &  &  & \\
r1r1R2R2r3r3 &  &  &  &  & \\
r1r1r2r2R3R3 &  &  &  &  & \\
\addlinespace
r1r1r2r2r3r3 &  &  &  &  & \\
\bottomrule
\end{tabular}
\endgroup{}
\end{table}
Question 11. How transgressive segregants are produced ? Present with the help of suitable figure.\\
\clearpage 
{\centering \Large{\textbf{Gokuleshwor Agriculture and Animal Science College, Baitadi}} \\[0.25cm]
            \textbf{Institute}: GAASC, IAAS \\[0.2cm]
            \textbf{Internal Assessment} \\[0.2cm]} 
\textbf{Credit hours}: 3 + 1 \\ 
\textbf{Level}: B. Sc. Ag (4th Semester) \\
\textbf{Roll no}: 43 \\[0.5cm] 
\textbf{Essay type} (1 x 10 = 10 marks) \\
Question 1. Suppose two inbred lines A and B are crossed to produce F1 hybrid. The F1 is selfed and F2 is produced. The genotype of inbred line A is AAbb and the genotype of inbred line B is aaBB. A and B are the dominant genes and contribute 12 cm and 10 cm towards the spike length of the hybrid respectively. In the absence of dominants, each recessive gene contributes 4 cm towards the spike length of the hybrid. The spike length of the best commercial variety if 25 cm. a. Find the spike length of the parents, F1 and F2 progeny b. Calculate average heterosis, heterobeltiosis, economic heterosis and inbreeding depression. c. Interpret the results.\\
\textbf{Short answer type} (10 x 3 = 30 marks) \\
Question 2. Briefly explain application of allopolyploids with the help of suitable examples.\\
Question 3. Define intellectual property right. Explain its forms.\\
Question 4. What do you mean by Plant Breeder Right (PBR) ? What are the main points to be considered in getting PBR ?\\
Question 5. What is inbreeding depression ? Describe the effects of inbreeding.\\
Question 6. Differentiate between full sib and half sib selection. Which selection scheme is the most effective in breeding maize ? Logically explain.\\
Question 7. Differentiate between gametophytic and sporophytic systems of self-incompatibility.\\
Question 8. How do you release a superior variety of rice (Oryza sativa) if the existing variety yields 3.5 tons per hectare and is 75\% disease resistant.\\
Question 9. From a random mating population with mean of 200 units, individuals with mean of 260 are selected to be the parents of the next generation. If the heritability of the trait is 0.6, what is your expected mean in the next generation ?\\
Question 10. What is hybridization ? Describe the steps involved in hybridization, in brief.\\
Question 11. How do you produce single, double and three way cross hybrids using cytoplasmic male sterility ? Explain.\\
\clearpage 
{\centering \Large{\textbf{Gokuleshwor Agriculture and Animal Science College, Baitadi}} \\[0.25cm]
            \textbf{Institute}: GAASC, IAAS \\[0.2cm]
            \textbf{Internal Assessment} \\[0.2cm]} 
\textbf{Credit hours}: 3 + 1 \\ 
\textbf{Level}: B. Sc. Ag (4th Semester) \\
\textbf{Roll no}: 44 \\[0.5cm] 
\textbf{Essay type} (1 x 10 = 10 marks) \\
Question 1. What do you mean by heterosis and inbreeding depression? A plant breeder crossed two genotypes of wheat AAbb and aaBB to get F1. The F1 was selfed to obtain F2. On the basis of heterosis governing theories (dominance and over dominance), find all kinds of heterosis and inbreeding depression. Each of the dominant homozygote, heterozygote and recessive homozygote contributes 4 ton, 6 ton and 2 ton per hectare in yield respectively. The commercial variety of wheat yields 5 ton/ha.\\
\textbf{Short answer type} (10 x 3 = 30 marks) \\
Question 2. Mention current research activities carried out by NARC in the improvement of wheat crop.\\
Question 3. Explain the role of environment on quantitative character.\\
Question 4. Define heterosis. Write the causes of heterosis and theories governing heterosis.\\
Question 5. Define cytoplasmic male sterility. Show the cross how male sterility line is maintained during production of single cross hybrid and double cross hybrid ?\\
Question 6. For a quantitative trait in a RMP, mean is 100 and variation is 240. The regression of the offspring on mid-parent value is 0.25. Truncation selection is practiced with selection differential of 32. What is the expected mean in the next generation ?\\
Question 7. Explain the pureline theory of Johansen\\
Question 8. Differentiate qualitative and quantitative traits.\\
Question 9. How do you improve vegetatively propagated crops ? Describe a method of your choice with example.\\
Question 10. The mean days to maturity and variance are 120 and 144 respectively. A plant breeder selected the top 5\% plants from base population and found mean days to maturity 110 in the next generation. Find the genetic gain and heritability of this trait.\\
Question 11. What is hybridization ? Describe the steps involved in hybridization, in brief.\\
\clearpage 
{\centering \Large{\textbf{Gokuleshwor Agriculture and Animal Science College, Baitadi}} \\[0.25cm]
            \textbf{Institute}: GAASC, IAAS \\[0.2cm]
            \textbf{Internal Assessment} \\[0.2cm]} 
\textbf{Credit hours}: 3 + 1 \\ 
\textbf{Level}: B. Sc. Ag (4th Semester) \\
\textbf{Roll no}: 45 \\[0.5cm] 
\textbf{Essay type} (1 x 10 = 10 marks) \\
Question 1. There is a wheat variety which is very good in terms of yield and other performances. It lacks trait, for instance, say disease resistance. A local landrace with recessive resistance is found. How do you incorporate it in the above wheat variety which is otherwise good ?\\
\textbf{Short answer type} (10 x 3 = 30 marks) \\
Question 2. What different types of progeny will occur in gametophytic and sporophytic system of self-incompatibility from following cross ? In which system, homozygous progeny can occur and why? The dominance relationship operated like, S1 $>$ S2 $>$ S3 $>$ S4.\\ 
\begin{table}[H]
\centering\begingroup\fontsize{8}{10}\selectfont

\begin{tabular}[t]{llllll}
\toprule
Male & Female & Gametophytic SI reaction & Gametophytic SI progeny & Sporophytic SI reaction & Sporophytic SI progeny\\
\midrule
S1S3 & S1S2 &  &  &  & \\
S1S2 & S1S3 &  &  &  & \\
S2S3 & S1S4 &  &  &  & \\
\bottomrule
\end{tabular}
\endgroup{}
\end{table}
Question 3. Differentiate between gametophytic and sporophytic systems of self-incompatibility.\\
Question 4. Explain pureline theory on the basis of Johansen's experiment. What might be the application of this theory in breeding program?\\
Question 5. Write main achievements of plant breeding in the context of Nepal.\\
Question 6. How do you produce double cross hybrid using genetic male sterility ? Outline the procedure.\\
Question 7. Define intellectual property right. Explain its forms.\\
Question 8. Differentiate between full sib and half sib selection. Which selection scheme is the most effective in breeding maize ? Logically explain.\\
Question 9. Explain pureline theory given by Johannsen.\\
Question 10. How do you improve vegetatively propagated crops ? Describe a method of your choice with example.\\
Question 11. What is patent ? Explain its requirements.\\
\clearpage 
{\centering \Large{\textbf{Gokuleshwor Agriculture and Animal Science College, Baitadi}} \\[0.25cm]
            \textbf{Institute}: GAASC, IAAS \\[0.2cm]
            \textbf{Internal Assessment} \\[0.2cm]} 
\textbf{Credit hours}: 3 + 1 \\ 
\textbf{Level}: B. Sc. Ag (4th Semester) \\
\textbf{Roll no}: 46 \\[0.5cm] 
\textbf{Essay type} (1 x 10 = 10 marks) \\
Question 1. What do you mean by heterosis and inbreeding depression? A plant breeder crossed two genotypes of wheat AAbb and aaBB to get F1. The F1 was selfed to obtain F2. On the basis of heterosis governing theories (dominance and over dominance), find all kinds of heterosis and inbreeding depression. Each of the dominant homozygote, heterozygote and recessive homozygote contributes 4 ton, 6 ton and 2 ton per hectare in yield respectively. The commercial variety of wheat yields 5 ton/ha.\\
\textbf{Short answer type} (10 x 3 = 30 marks) \\
Question 2. How do you develop single and double cross hybrids using cytoplasmic male sterility ? Explain.\\
Question 3. What is intellectual property right? Write the requirements of patent.\\
Question 4. In a random mating poulation, the mean plant height and variance are 120 cm and 121 square cm respectively. A plant breeder selected the top 5\% plants from the base population and found mean plant height 110 cm in the next generation. Find the gentic gain, selection differential and heritability of this trait.\\
Question 5. How do you release a superior variety of rice (Oryza sativa) if the existing variety yields 3.5 tons per hectare and is 75\% disease resistant.\\
Question 6. Differentiate between: a. Self incompatibility and male sterility b. Self pollinated crops and cross pollinated crops\\
Question 7. How do you produce hybrid seed using one self-incompatible (P1) and another self-compatible (P2) parents ?\\
Question 8. Enlist different breeding methods used in Rice and Maize crops.\\
Question 9. From the following data, calculate heterosis, average heterosis and economic heterosis for grain yield of a wheat hybrid.\\ 
\begin{table}[H]
\centering\begingroup\fontsize{8}{10}\selectfont

\begin{tabular}[t]{ll}
\toprule
Parents & Grain yield (t/ha)\\
\midrule
A & 6.6\\
B & 4.2\\
F1(AxB) & 8.5\\
Best commercial variety in area (Gautam) & 7\\
\bottomrule
\end{tabular}
\endgroup{}
\end{table}
Question 10. Present current research activities carried out by Nepal Agriculture Research Council (NARC) for the improvement of wheat crop.\\
Question 11. How do you improve vegetatively propagated crops ? Describe a method of your choice with example.\\
\clearpage 
{\centering \Large{\textbf{Gokuleshwor Agriculture and Animal Science College, Baitadi}} \\[0.25cm]
            \textbf{Institute}: GAASC, IAAS \\[0.2cm]
            \textbf{Internal Assessment} \\[0.2cm]} 
\textbf{Credit hours}: 3 + 1 \\ 
\textbf{Level}: B. Sc. Ag (4th Semester) \\
\textbf{Roll no}: 47 \\[0.5cm] 
\textbf{Essay type} (1 x 10 = 10 marks) \\
Question 1. What is G x E interaction ? Explain the various methods of estimating G x E interaction.\\
\textbf{Short answer type} (10 x 3 = 30 marks) \\
Question 2. Present current research activities carried out by Nepal Agriculture Research Council (NARC) for the improvement of wheat crop.\\
Question 3. How do you improve vegetatively propagated crops ? Describe a method of your choice with example.\\
Question 4. Differentiate between qualitative and quantitative traits.\\
Question 5. State Hardy-Weinberg law. In a population consisting of 10000 individuals, 49 individuals are of "aa" genotype. If the population is in Hardy-Weinberg equilibrium, find the gene and genotype frequencies of that population.\\
Question 6. Define cytoplasmic male sterility. Show the cross how male sterility line is maintained during production of single cross hybrid and double cross hybrid ?\\
Question 7. How transgressive segregants are produced ? Present with the help of suitable figure.\\
Question 8. How do you produce double cross hybrid using genetic male sterility ? Outline the procedure.\\
Question 9. In a random mating poulation, the mean plant height and variance are 120 cm and 121 square cm respectively. A plant breeder selected the top 5\% plants from the base population and found mean plant height 110 cm in the next generation. Find the gentic gain, selection differential and heritability of this trait.\\
Question 10. Differentiate between (any three): a. Inbreeders and outbreeders b. Race and pathotype c. Hybrids and synthetics d. Breeder's and certified seeds\\
Question 11. Define intellectual property right. Explain its forms.\\
\clearpage 
{\centering \Large{\textbf{Gokuleshwor Agriculture and Animal Science College, Baitadi}} \\[0.25cm]
            \textbf{Institute}: GAASC, IAAS \\[0.2cm]
            \textbf{Internal Assessment} \\[0.2cm]} 
\textbf{Credit hours}: 3 + 1 \\ 
\textbf{Level}: B. Sc. Ag (4th Semester) \\
\textbf{Roll no}: 48 \\[0.5cm] 
\textbf{Essay type} (1 x 10 = 10 marks) \\
Question 1. How do you transfer disease resistance gene from uncultivated genotype to cultivated cultivar, which is susceptible? Explain in detail.\\
\textbf{Short answer type} (10 x 3 = 30 marks) \\
Question 2. For a quantitative trait in a RMP, mean is 100 and variation is 240. The regression of the offspring on mid-parent value is 0.25. Truncation selection is practiced with selection differential of 32. What is the expected mean in the next generation ?\\
Question 3. Briefly explain application of allopolyploids with the help of suitable examples.\\
Question 4. How are haploids/monoploids produced and utilized in plant breeding ?\\
Question 5. From the following data, calculate heterosis, average heterosis and economic heterosis for grain yield of a wheat hybrid.\\ 
\begin{table}[H]
\centering\begingroup\fontsize{8}{10}\selectfont

\begin{tabular}[t]{ll}
\toprule
Parents & Grain yield (t/ha)\\
\midrule
A & 6.6\\
B & 4.2\\
F1(AxB) & 8.5\\
Best commercial variety in area (Gautam) & 7\\
\bottomrule
\end{tabular}
\endgroup{}
\end{table}
Question 6. Write about the objectives of plant breeding.\\
Question 7. What do you mean by Plant Breeder Right (PBR) ? What are the main points to be considered in getting PBR ?\\
Question 8. From a random mating population with mean of 200 units, individuals with mean of 260 are selected to be the parents of the next generation. If the heritability of the trait is 0.6, what is your expected mean in the next generation ?\\
Question 9. Define heterosis. Write the causes of heterosis and theories governing heterosis.\\
Question 10. Define cytoplasmic male sterility. Show the cross how male sterility line is maintained during production of single cross hybrid and double cross hybrid ?\\
Question 11. Explain the pureline theory of Johansen\\
\clearpage 
{\centering \Large{\textbf{Gokuleshwor Agriculture and Animal Science College, Baitadi}} \\[0.25cm]
            \textbf{Institute}: GAASC, IAAS \\[0.2cm]
            \textbf{Internal Assessment} \\[0.2cm]} 
\textbf{Credit hours}: 3 + 1 \\ 
\textbf{Level}: B. Sc. Ag (4th Semester) \\
\textbf{Roll no}: 49 \\[0.5cm] 
\textbf{Essay type} (1 x 10 = 10 marks) \\
Question 1. What do you mean by heterosis and inbreeding depression? A plant breeder crossed two genotypes of wheat AAbb and aaBB to get F1. The F1 was selfed to obtain F2. On the basis of heterosis governing theories (dominance and over dominance), find all kinds of heterosis and inbreeding depression. Each of the dominant homozygote, heterozygote and recessive homozygote contributes 4 ton, 6 ton and 2 ton per hectare in yield respectively. The commercial variety of wheat yields 5 ton/ha.\\
\textbf{Short answer type} (10 x 3 = 30 marks) \\
Question 2. How do you develop single and double cross hybrids using cytoplasmic male sterility ? Explain.\\
Question 3. Present current research activities carried out by Nepal Agriculture Research Council (NARC) for the improvement of wheat crop.\\
Question 4. Briefly explain the self pollination enforcing mechanisms. What are the genetic consequences of self-pollination ?\\
Question 5. Half sib and full sib selection breeding methods are used in crop improvement of cross pollinated crops like maize. Which one is the most effective and why?\\
Question 6. How are haploids/monoploids produced and utilized in plant breeding ?\\
Question 7. Explain how genetic variation can be originated in a population.\\
Question 8. What is intellectual property right? Write the requirements of patent.\\
Question 9. State law of homologous series in variation by NI Vavilov. Discuss the relationship between, primary, secondary and tertiary gene pools with respect to their combining ability.\\
Question 10. Enlist various types of breeding methods used in self-pollinated crops. And, how a disease resistant dominant gene is transferred from uncultivated genotype to cultivated cultivar ? Outline the breeding procedure.\\
Question 11. List different breeding methods used in wheat and maize crops. Give your logics why generally a long time is required to release a variety in self pollinated crop as compared to cross pollinated crops?\\
\clearpage 
{\centering \Large{\textbf{Gokuleshwor Agriculture and Animal Science College, Baitadi}} \\[0.25cm]
            \textbf{Institute}: GAASC, IAAS \\[0.2cm]
            \textbf{Internal Assessment} \\[0.2cm]} 
\textbf{Credit hours}: 3 + 1 \\ 
\textbf{Level}: B. Sc. Ag (4th Semester) \\
\textbf{Roll no}: 50 \\[0.5cm] 
\textbf{Essay type} (1 x 10 = 10 marks) \\
Question 1. What do you mean by heterosis and inbreeding depression? A plant breeder crossed two genotypes of wheat AAbb and aaBB to get F1. The F1 was selfed to obtain F2. On the basis of heterosis governing theories (dominance and over dominance), find all kinds of heterosis and inbreeding depression. Each of the dominant homozygote, heterozygote and recessive homozygote contributes 4 ton, 6 ton and 2 ton per hectare in yield respectively. The commercial variety of wheat yields 5 ton/ha.\\
\textbf{Short answer type} (10 x 3 = 30 marks) \\
Question 2. Explain briefly the various mechanism which promote self and cross pollination in crop plants.\\
Question 3. Write about the objectives of plant breeding.\\
Question 4. Mention current research activities carried out by NARC in the improvement of wheat crop.\\
Question 5. For a quantitative trait in a RMP, mean is 100 and variation is 240. The regression of the offspring on mid-parent value is 0.25. Truncation selection is practiced with selection differential of 32. What is the expected mean in the next generation ?\\
Question 6. Write main achievements of plant breeding in the context of Nepal.\\
Question 7. A cultivated variety of wheat became susceptible to a fungal disease which drastically reduced the yield. However, a wild variety is resistant to this fungus. If the resistance is a dominant trait governed by "R" gene, is it possible to transfer this trait to the cultivated variety? Give procedure with your logics.\\
Question 8. What different types of progeny will occur in gametophytic and sporophytic system of self-incompatibility from following cross ? In which system, homozygous progeny can occur and why? The dominance relationship operated like, S1 $>$ S2 $>$ S3 $>$ S4.\\ 
\begin{table}[H]
\centering\begingroup\fontsize{8}{10}\selectfont

\begin{tabular}[t]{llllll}
\toprule
Male & Female & Gametophytic SI reaction & Gametophytic SI progeny & Sporophytic SI reaction & Sporophytic SI progeny\\
\midrule
S1S3 & S1S2 &  &  &  & \\
S1S2 & S1S3 &  &  &  & \\
S2S3 & S1S4 &  &  &  & \\
\bottomrule
\end{tabular}
\endgroup{}
\end{table}
Question 9. Briefly explain application of allopolyploids with the help of suitable examples.\\
Question 10. Write about the status of rice breeding in Nepal.\\
Question 11. How are haploids/monoploids produced and utilized in plant breeding ?\\
\clearpage 


\end{document}
