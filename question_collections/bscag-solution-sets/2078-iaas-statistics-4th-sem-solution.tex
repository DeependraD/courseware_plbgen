% Options for packages loaded elsewhere
\PassOptionsToPackage{unicode}{hyperref}
\PassOptionsToPackage{hyphens}{url}
%
\documentclass[
  answers,addpoints,12pt]{exam}
\usepackage{lmodern}
\usepackage{amssymb,amsmath}
\usepackage{ifxetex,ifluatex}
\ifnum 0\ifxetex 1\fi\ifluatex 1\fi=0 % if pdftex
  \usepackage[T1]{fontenc}
  \usepackage[utf8]{inputenc}
  \usepackage{textcomp} % provide euro and other symbols
\else % if luatex or xetex
  \usepackage{unicode-math}
  \defaultfontfeatures{Scale=MatchLowercase}
  \defaultfontfeatures[\rmfamily]{Ligatures=TeX,Scale=1}
\fi
% Use upquote if available, for straight quotes in verbatim environments
\IfFileExists{upquote.sty}{\usepackage{upquote}}{}
\IfFileExists{microtype.sty}{% use microtype if available
  \usepackage[]{microtype}
  \UseMicrotypeSet[protrusion]{basicmath} % disable protrusion for tt fonts
}{}
\makeatletter
\@ifundefined{KOMAClassName}{% if non-KOMA class
  \IfFileExists{parskip.sty}{%
    \usepackage{parskip}
  }{% else
    \setlength{\parindent}{0pt}
    \setlength{\parskip}{6pt plus 2pt minus 1pt}}
}{% if KOMA class
  \KOMAoptions{parskip=half}}
\makeatother
\usepackage{xcolor}
\IfFileExists{xurl.sty}{\usepackage{xurl}}{} % add URL line breaks if available
\IfFileExists{bookmark.sty}{\usepackage{bookmark}}{\usepackage{hyperref}}
\hypersetup{
  pdftitle={Question Set Solution},
  hidelinks,
  pdfcreator={LaTeX via pandoc}}
\urlstyle{same} % disable monospaced font for URLs
\usepackage[top=1.5cm,bottom=1.5cm,left=1.5cm,right=1.5cm]{geometry}
\usepackage{longtable,booktabs}
% Correct order of tables after \paragraph or \subparagraph
\usepackage{etoolbox}
\makeatletter
\patchcmd\longtable{\par}{\if@noskipsec\mbox{}\fi\par}{}{}
\makeatother
% Allow footnotes in longtable head/foot
\IfFileExists{footnotehyper.sty}{\usepackage{footnotehyper}}{\usepackage{footnote}}
\makesavenoteenv{longtable}
\usepackage{graphicx}
\makeatletter
\def\maxwidth{\ifdim\Gin@nat@width>\linewidth\linewidth\else\Gin@nat@width\fi}
\def\maxheight{\ifdim\Gin@nat@height>\textheight\textheight\else\Gin@nat@height\fi}
\makeatother
% Scale images if necessary, so that they will not overflow the page
% margins by default, and it is still possible to overwrite the defaults
% using explicit options in \includegraphics[width, height, ...]{}
\setkeys{Gin}{width=\maxwidth,height=\maxheight,keepaspectratio}
% Set default figure placement to htbp
\makeatletter
\def\fps@figure{htbp}
\makeatother
\setlength{\emergencystretch}{3em} % prevent overfull lines
\providecommand{\tightlist}{%
  \setlength{\itemsep}{0pt}\setlength{\parskip}{0pt}}
\setcounter{secnumdepth}{5}
% \usepackage{exam}

\newcommand{\bquestions}{\begin{questions}}
\newcommand{\equestions}{\end{questions}}
\newcommand{\bsolution}{\begin{solution}}
\newcommand{\esolution}{\end{solution}}
\newcommand{\bparts}{\begin{parts}}
\newcommand{\eparts}{\end{parts}}
\newcommand{\stpart}{\part} % this is absolutely necessary to make new part command
\newcommand{\bsubparts}{\begin{subparts}}
\newcommand{\esubparts}{\end{subparts}}
\newcommand{\stsubpart}{\subpart}

% solution environment is a minipage and cannot support float so, always use "HOLD_position" 
\usepackage{float} % this package is essential for solution with code
% eqnarray is better avoided, most suggestion lead to \align provided in amsmath
% also split is used when there are very long lines of equation
% for different expression of the same equation use line separator \\ and use \notag or \nonumber
\usepackage{eqnarray, amsmath}

\usepackage{background}
\usepackage{eso-pic}
\usepackage{contour}

\backgroundsetup{
  angle=90,
  opacity=0.8,
  scale=1.2,
  color=red,
  nodeanchor=south west,
  position={current page.south east},
  contents={}{{\footnotesize Generated by }{\Large Deependra Dhakal }{(\footnotesize to be shared and distributed freely!)}},
  hshift=40pt,% to move the text vertically
  vshift=+10pt% to move the text horizontally
}

\pagestyle{empty}
\usepackage{booktabs}
\usepackage{longtable}
\usepackage{array}
\usepackage{multirow}
\usepackage{wrapfig}
\usepackage{float}
\usepackage{colortbl}
\usepackage{pdflscape}
\usepackage{tabu}
\usepackage{threeparttable}
\usepackage{threeparttablex}
\usepackage[normalem]{ulem}
\usepackage{makecell}
\usepackage{xcolor}

\title{Question Set Solution}
\usepackage{etoolbox}
\makeatletter
\providecommand{\subtitle}[1]{% add subtitle to \maketitle
  \apptocmd{\@title}{\par {\large #1 \par}}{}{}
}
\makeatother
\subtitle{Agricultural Statistics, 4th Semester, IAAS, 2078}
\author{Deependra Dhakal}
\date{November, 2021}

\begin{document}
\maketitle

\hypertarget{essay-type-question}{%
\section{Essay Type Question}\label{essay-type-question}}

\bquestions

\question[10] \label{quest:first}

A set of data involving four feed stuffs used A, B, C and D tried on certain 20 mice is given in Table \ref{tab:mice-weight-gain} below. All the mice are treated alike in all respects except the following treatment and each feeding is given to 5 mice.

\begin{longtable}[t]{llllll}
\caption{\label{tab:mice-weight-gain}Feeding treatment and weight gain in mice}\\
\toprule
\multicolumn{1}{c}{ } & \multicolumn{5}{c}{Weight gain} \\
\cmidrule(l{3pt}r{3pt}){2-6}
Feed &   &   &   &   &  \\
\midrule
A & 45 & 56 & 54 &  & \\
B & 45 & 65 & 67 & 23 & \\
C & 23 & 45 & 18 & 67 & 78\\
D & 45 & 34 &  &  & \\
\bottomrule
\end{longtable}

Test the hypothesis that the mean effect of feeds is same or not.

\bsolution (Question \ref{quest:first})

Given information in the Table \ref{tab:mice-weight-gain} from feeding experiment involves missing data for some mice (Only x mice are represented despite involving a total of 20).

Since we are interested in whether or not the weight gain due to different feed (\(t = 4\)) are different, a widely used technique for comparison of group difference involving more that 2 groups is F-test. The F-test can be employed most easily with analysis of variance (ANOVA) setting. Although, it should be noted, due to unequal group sizes (ie. there are 3 mice in Feed A group, 4 mice in Feed B group, and so on), validity of group difference comparison is limited only to the data available.

\begin{longtable}[t]{ll}
\caption{\label{tab:mice-weight-gain-reshaped}Mean weight gain of mice in each treatment (feeding) group}\\
\toprule
Feed & Mean weight gain\\
\midrule
A & 51.67\\
B & 50.00\\
C & 46.20\\
D & 39.50\\
\bottomrule
\end{longtable}

To detect the difference in treatment means, we formulate the following Null hypothesis, of which if no sufficient evidence is found for acceptance we assert that difference between mean of Feeding groups exist.

\[\mu_{A} = \mu_{B} = \mu_C = \mu_D = \mu_E\]

Total variance due to feeding treatment can be disaggregated into effects due to Feed and the noise (residual). Respective variances components, degrees of freedom, and the F-value for treatment factor are computed and shown in the ANOVA (Table \ref{tab:mice-weight-gain-anova}).

\begingroup\fontsize{10}{12}\selectfont

\begin{longtable}[t]{>{\raggedright\arraybackslash}p{5em}>{\raggedright\arraybackslash}p{2em}>{\raggedright\arraybackslash}p{8em}>{\raggedright\arraybackslash}p{8em}>{\raggedright\arraybackslash}p{6em}>{\raggedright\arraybackslash}p{4em}}
\caption{\label{tab:mice-weight-gain-anova}ANOVA of study on mice weight gain due to various feeding treatment}\\
\toprule
Source of variation & DF & Sum of squares & Mean sum of squares & F-value & P-value\\
\midrule
Feed & 3 & 213.53 & 71.18 & 0.17 & 0.91\\
Residuals & 10 & 4175.97 & 417.60 &  & \\
\bottomrule
\end{longtable}
\endgroup{}

\begin{longtable}[t]{lll}
\caption{\label{tab:mice-weight-gain-anova}Difference of mean between treatment groups using LSD method. Group difference is indicated by lowercase letters (all 'a' for every Feed group).}\\
\toprule
Feed & Weight gain & Groups\\
\midrule
A & 51.67 & a\\
B & 50.00 & a\\
C & 46.20 & a\\
D & 39.50 & a\\
\bottomrule
\end{longtable}

It can be concluded based on Null Hypothesis Significance Testing (NHST) that, no Feeding groups are different.

\esolution

\equestions

\hypertarget{short-questions}{%
\section{Short Questions}\label{short-questions}}

\bquestions

\question[4] \label{quest:second}

Wheat do you mean by the random sampling ? Differentiate between stratified sampling and cluster sampling.

\bsolution

Write anything but not the correct answer.

\esolution

\equestions

\end{document}
