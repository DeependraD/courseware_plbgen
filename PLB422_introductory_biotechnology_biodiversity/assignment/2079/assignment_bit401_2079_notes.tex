% Options for packages loaded elsewhere
\PassOptionsToPackage{unicode}{hyperref}
\PassOptionsToPackage{hyphens}{url}
%
\documentclass[
]{article}
\usepackage{amsmath,amssymb}
\usepackage{lmodern}
\usepackage{iftex}
\ifPDFTeX
  \usepackage[T1]{fontenc}
  \usepackage[utf8]{inputenc}
  \usepackage{textcomp} % provide euro and other symbols
\else % if luatex or xetex
  \usepackage{unicode-math}
  \defaultfontfeatures{Scale=MatchLowercase}
  \defaultfontfeatures[\rmfamily]{Ligatures=TeX,Scale=1}
\fi
% Use upquote if available, for straight quotes in verbatim environments
\IfFileExists{upquote.sty}{\usepackage{upquote}}{}
\IfFileExists{microtype.sty}{% use microtype if available
  \usepackage[]{microtype}
  \UseMicrotypeSet[protrusion]{basicmath} % disable protrusion for tt fonts
}{}
\makeatletter
\@ifundefined{KOMAClassName}{% if non-KOMA class
  \IfFileExists{parskip.sty}{%
    \usepackage{parskip}
  }{% else
    \setlength{\parindent}{0pt}
    \setlength{\parskip}{6pt plus 2pt minus 1pt}}
}{% if KOMA class
  \KOMAoptions{parskip=half}}
\makeatother
\usepackage{xcolor}
\usepackage{longtable,booktabs,array}
\usepackage{calc} % for calculating minipage widths
% Correct order of tables after \paragraph or \subparagraph
\usepackage{etoolbox}
\makeatletter
\patchcmd\longtable{\par}{\if@noskipsec\mbox{}\fi\par}{}{}
\makeatother
% Allow footnotes in longtable head/foot
\IfFileExists{footnotehyper.sty}{\usepackage{footnotehyper}}{\usepackage{footnote}}
\makesavenoteenv{longtable}
\usepackage{graphicx}
\makeatletter
\def\maxwidth{\ifdim\Gin@nat@width>\linewidth\linewidth\else\Gin@nat@width\fi}
\def\maxheight{\ifdim\Gin@nat@height>\textheight\textheight\else\Gin@nat@height\fi}
\makeatother
% Scale images if necessary, so that they will not overflow the page
% margins by default, and it is still possible to overwrite the defaults
% using explicit options in \includegraphics[width, height, ...]{}
\setkeys{Gin}{width=\maxwidth,height=\maxheight,keepaspectratio}
% Set default figure placement to htbp
\makeatletter
\def\fps@figure{htbp}
\makeatother
\setlength{\emergencystretch}{3em} % prevent overfull lines
\providecommand{\tightlist}{%
  \setlength{\itemsep}{0pt}\setlength{\parskip}{0pt}}
\setcounter{secnumdepth}{-\maxdimen} % remove section numbering
\usepackage{multicol}
\usepackage{enumitem}
\ifLuaTeX
  \usepackage{selnolig}  % disable illegal ligatures
\fi
\IfFileExists{bookmark.sty}{\usepackage{bookmark}}{\usepackage{hyperref}}
\IfFileExists{xurl.sty}{\usepackage{xurl}}{} % add URL line breaks if available
\urlstyle{same} % disable monospaced font for URLs
\hypersetup{
  pdftitle={Assignment on Biotechnology and Biodiversity (BIT 401), 2079},
  pdfauthor={Deependra Dhakal},
  hidelinks,
  pdfcreator={LaTeX via pandoc}}

\title{Assignment on Biotechnology and Biodiversity (BIT 401), 2079}
\author{Deependra Dhakal}
\date{March 31 2023}

\begin{document}
\maketitle

\hypertarget{task}{%
\section{Task}\label{task}}

Construct a database of ``Genetic diversity of wild and related species/genera of important cultivated crops with notes on molecular genetic studies''.

\hypertarget{guidelines}{%
\section{Guidelines}\label{guidelines}}

\begin{itemize}
\item
  Database need to be submitted in `xlsx' (Microsoft excel) and ``pdf'' format. (Both contain same data, but `pdf' version may be polished to look better. Graphs may be included in the `pdf' version.)
\item
  Each row displays extended information about the relevant ``Lower taxa'' or ``taxa identifier''.
\item
  Database should (at the minimum contain) following attributes (columns) of information\footnote{Attributes in \textbf{Boldface} deserve citation of each information with corresponding reference in the ``References'' column.}:

  \begin{itemize}
  \tightlist
  \item
    SN
  \item
    Crop
  \item
    Scientific name
  \item
    Family
  \item
    Higher taxa 1
  \item
    Higher taxa notes
  \item
    Lower taxa 1 (Subspecies/Landrace/Variety/Population/Genotype/Cultivar/\ldots)
  \item
    Lower taxa 2 (Subspecies/Landrace/Variety/Population/Genotype/Cultivar/\ldots)
  \item
    Lower taxa notes
  \item
    Geographical distribution
  \item
    Nomenclature synonyms
  \item
    Nomenclature references
  \item
    \textbf{Major characters relevant for use in crop improvement}
  \item
    Number of chromosomes
  \item
    Genome constitution
  \item
    \textbf{Ancestry/Evolutionary relationship with cultivated species group}
  \item
    \textbf{Notes on genome (ploidy, domestication process)}
  \item
    \textbf{Major findings and references for genetic studies}
  \item
    \textbf{Notes about specific genes and its functions}
  \item
    References
  \end{itemize}
\item
  Crops

  \begin{multicols}{2}
  \begin{enumerate}[noitemsep,topsep=0pt]
  \item Rice
  \item Wheat
  \item Maize
  \item Fingermillet
  \item Minor millets
  \item Buckwheat
  \item Barley
  \item Soybean
  \item Blackgram
  \item Lentil
  \item Groundnut
  \item Pea
  \item Chickpea
  \item Pigeonpea
  \item Potato
  \item Sugarcane
  \item Jute
  \item Cotton
  \item Brassica (with focus on rapeseed, broad leaf mustard and radish)
  \item Sunflower
  \item Sesame
  \item Tea
  \item Coffee
  \item Chilly/pepper
  \item Tomato
  \item Garlic
  \item Onion
  \item Taro
  \item Mango
  \item Apple
  \item Banana
  \item Grape
  \item Pear
  \item Citruses
  \item Pomegranate
  \item Chestnut
  \item Strawberry
  \item Raspberries and blueberries
  \item Guava
  \item Pumpkin
  \item Carrot
  \item Cucumber
  \item Gourds (Bottlegourd, spongegourd, bittergourd, pointedgourd, snakegourd)
  \item Amaranthus
  \item Oat
  \end{enumerate}
  \end{multicols}
\item
  Maintain the image (self photographed or acquired from the internet, of high quality: \textgreater400x400 pixels and \textless1200x1200) of respective variant in distinct folder for each crop whenever possible. Rename image file to include identification of corresponding taxa under which it's description is recorded.
\end{itemize}

\hypertarget{allotment}{%
\subsection{Allotment}\label{allotment}}

Each student is required to explore a unique crop, maintain a database information on it. The allotment of crop to student is made in random.

\hypertarget{assignment-due-and-submission}{%
\subsection{Assignment due and submission}\label{assignment-due-and-submission}}

The assignment is due April 15, 2023 (12:00 PM). It is required to be submitted through email (\href{mailto:ddhakal.rookie@gmail.com}{\nolinkurl{ddhakal.rookie@gmail.com}}). Each assignment should accompany an xlsx file, a pdf file and a zip file (contains all the properly named images collected). A template `xlsx' file called `database\_template.xlsx' to be completed for each crop is provided herewith. While submitting over the email the `xlsx' file should be named `student\_name-database-cropname.xlsx' (for example: `john\_doe-database-mustard.xlsx'), the `pdf' file should be named `student\_name-database-cropname\_extended.pdf' and the `zip' file should be named `student\_name-database-cropname-images.zip'.

\end{document}
